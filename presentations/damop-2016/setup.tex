\documentclass{beamer}
\usepackage[english]{babel}
\usepackage[latin1]{inputenc}
\usepackage{times}
\usepackage[T1]{fontenc}
\usepackage{amsmath}
\usepackage{amssymb}
\usepackage{esint}
\usepackage{hyperref}
\usepackage{tikz}
\usepackage{xkeyval}
\usepackage{xargs}
\usepackage{verbatim}
\usepackage{listings}
\usepackage{multimedia}
\usepackage{pgfplots}
\usepgfplotslibrary{colormaps}
\usetikzlibrary{
  arrows,
  calc,
  decorations.pathmorphing,
  decorations.pathreplacing,
  decorations.markings,
  fadings,
  positioning,
  shapes
}

\pgfdeclareradialshading{glow}{\pgfpoint{0cm}{0cm}}{
  color(0mm)=(white);
  color(3mm)=(white);
  color(7mm)=(black);
  color(10mm)=(black)
}
\pgfdeclareverticalshading{beam}{2cm}{
  % manual 1082-1083; later - shading is assumed to be 100bp diameter ??
  color(0cm)=(black);
  color(0.3cm)=(black);
  color(1cm)=(white);
  color(1.7cm)=(black);
  color(2cm)=(black)
}

\begin{tikzfadingfrompicture}[name=glow fading]
  \shade [shading=glow] (0,0) circle (1);
\end{tikzfadingfrompicture}

\begin{tikzfadingfrompicture}[name=beam fading]
  \shade [shading=beam] (-1,-1) rectangle (1, 1);
\end{tikzfadingfrompicture}

\beamertemplatenavigationsymbolsempty

\begin{document}

\begin{frame}
  \begin{tikzpicture}
    \visible<2-3>{
      \shade[shading=radial,rotate=90,yscale=1,fill opacity=0.6,
      inner color=red]
      plot[draw,samples=200,domain=-3:3] function {sqrt(0.0025 + x**2 / 10)}
      -- plot[draw,samples=200,domain=3:-3] function {-sqrt(0.0025 + x**2 / 10)};
      \path (0, 3) node[below] {Tweezer};
    }
    \visible<1-2>{
      \fill[orange,opacity=0.8,path fading=glow fading] (0,0) circle (1.5);
    }
    \visible<3>{
      \path (2, 0) node[align=center] {Imaging\\beams};
      \fill[orange,opacity=0.7,path fading=beam fading,transform canvas={rotate=25}]
      (-2, -0.2) rectangle (2, 0.2);
      \fill[orange,opacity=0.7,path fading=beam fading,transform canvas={rotate=-25}]
      (-2, -0.2) rectangle (2, 0.2);
    }
    \visible<2-3>{
      \fill [blue,path fading=glow fading] (0,0) circle (0.08);
    }
  \end{tikzpicture}
\end{frame}

\end{document}
