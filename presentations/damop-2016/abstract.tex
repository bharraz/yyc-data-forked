\documentclass{article}
\begin{document}
In order to create a diatomic molecule with a large electric dipole moment,
it is generally necessary to use atoms with very different electronegativities.
In the context of bi-alkali molecules,
this means combining a light alkali atom with a heavy one.
This is the reason why we use NaCs in our molecule assembler experiment;
NaCs has the largest induced dipole moment in few kV/cm lab fields.
However, the use of sodium atoms also poses challenges.
\begin{itemize}
\item The higher Doppler temperature and lack of efficient D2 polarization
  gradient cooling increases the necessary depth of our optical dipole
  (tweezer) traps.
\item The lack of a convenient magic wavelength for the dipole trap
  creates a large AC stark shift on the optical transition as well as
  additional heating mechanisms.
\item The light mass of the sodium, and therefore larger Lamb-Dicke
  parameter and higher recoil temperature, makes it more difficult to
  perform efficient Raman sideband cooling on the atom in the trap.
\end{itemize}
I will discuss the techniques we use to overcome these challenges,
in particular a method to eliminate the light shifts and associated heating
mechanisms in tight optical traps.
\end{document}
