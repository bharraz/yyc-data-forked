\documentclass[border=0pt]{standalone}
\usepackage{tikz}
\usetikzlibrary{decorations.pathreplacing,
  fadings
}
\tikzfading[name=fade img right,left color=transparent!100, right color=transparent!0]
\tikzfading[name=fade img left,right color=transparent!100, left color=transparent!0]
\usepackage{graphicx}
\begin{document}
\begin{tikzpicture}
  % 0.0 - 1.6 axial (1.6)
  % 1.6 - 1.9 OP (0.3)
  % 1.9 - 2.5 radial 2 (0.6)
  % 2.5 - 2.8 OP (0.3)
  % 2.8 - 4.4 axial (1.6)
  % 4.4 - 4.7 OP (0.3)
  % 4.7 - 5.3 radial 3 (0.6)
  % 5.3 - 5.6 OP (0.3)
  % 5.6 - 7.2 axial (1.6)
  % 7.2 - 7.5 OP (0.3)
  % 7.5 - 8.1 radial 2 (0.6)
  % 8.1 - 8.4 OP (0.3)
  % 8.4 - 10.0 axial (1.6)
  % 10.0 - 10.3 OP (0.3)
  % 10.3 - 10.9 radial 3 (0.6)
  % 10.9 - 11.2 OP (0.3)

  \begin{scope}
    \clip (4.6, -2.1) rectangle (17.8, 3.5);
    %% Pulse mark
    \foreach \i in {0,...,7} {
      \draw[gray!20, line width=2] (2.8*\i, 1.5) -- (2.8*\i, -2.0);
      \draw[gray!20, line width=2] (2.8*\i+1.9, 1.5) -- (2.8*\i+1.9, -2.0);
    }

    %% Tweezer
    \draw[line width=1.5,purple] (0, 2.4) -- (22.4, 2.4);
    \foreach \i in {0,...,99} {
      \filldraw[line width=1,purple] (\i*22.4/100, 2.4) rectangle (\i*22.4/100+0.06, 3.5);
    }

    %% Axial
    \foreach \i in {0,...,7} {
      \fill[blue!30]
      plot[draw,samples=200,domain={2.8*\i}:{2.8*\i+1.6},variable=\x] ({\x}, {sin((\x-2.8*\i) * 112.5)^4})
      -- cycle;
      \draw[line width=1,blue] plot[domain={2.8*\i}:{2.8*\i+1.6},variable=\x]
      ({\x}, {sin((\x-2.8*\i) * 112.5)^4}) -- (2.8*\i + 2.8, 0);
    }
    \path[blue] (2.8*2+0.8, 0) node[below,align=center] {\textbf{Axis 1}\\\textit{Axial}};
    \foreach \i in {0,...,3} {
      \path (5.6*\i + 0.8, 1) node[above,blue,align=center] {\textbf{$n$-th}\\\textbf{order}};
      \path (5.6*\i + 3.6, 1) node[above,blue,align=center] {\textbf{$(n-1)$-th}\\\textbf{order}};
    }

    %% Radial 2
    \foreach \i in {0,...,3} {
      \fill[red!30]
      plot[draw,samples=200,domain={5.6*\i+1.9}:{5.6*\i+2.5},variable=\x] ({\x}, {1.5*sin((\x-5.6*\i-1.9) * 300)^4})
      -- cycle;
      \draw[line width=1,red] (5.6*\i, 0) -- plot[domain={5.6*\i+1.9}:{5.6*\i+2.5},variable=\x]
      ({\x}, {1.5*sin((\x-5.6*\i-1.9) * 300)^4}) -- (5.6*\i+5.6, 0);
    }
    \path[red] (5.6*1+2.2, 1.5) node[above,align=center] {\textbf{Axis 2}\\\textit{Radial}};

    %% Radial 3
    \foreach \i in {0,...,3} {
      \fill[black!30]
      plot[draw,samples=200,domain={5.6*\i+4.7}:{5.6*\i+5.3},variable=\x] ({\x}, {1.5*sin((\x-5.6*\i-4.7) * 300)^4})
      -- cycle;
      \draw[line width=1,black] (5.6*\i, 0) -- plot[domain={5.6*\i+4.7}:{5.6*\i+5.3},variable=\x]
      ({\x}, {1.5*sin((\x-5.6*\i-4.7) * 300)^4}) -- (5.6*\i+5.6, 0);
    }
    \path[black] (5.6*1+5.0, 1.5) node[above,align=center] {\textbf{Axis 3}\\\textit{Radial}};

    %% OP
    \foreach \i in {0,...,7} {
      \filldraw[fill=orange!30,draw=orange,line width=1.2]
      (2.8*\i+1.58, -2) rectangle (2.8*\i+1.91, -0.7);
      \filldraw[fill=orange!30,draw=orange,line width=1.2]
      (2.8*\i+2.48, -2) rectangle (2.8*\i+2.81, -0.7);
    }
    \draw[line width=2,orange] (0,-2) -- (22.4,-2);
  \end{scope}

  \fill[white,path fading=fade img right] (16.8, -2.2) rectangle (17.81, 3.51);
  \fill[white,path fading=fade img right] (16.8, -2.2) rectangle (17.81, 3.51);
  \fill[white,path fading=fade img right] (16.8, -2.2) rectangle (17.81, 3.51);

  \fill[white,path fading=fade img left] (5.6, -2.2) rectangle (4.59, 3.51);
  \fill[white,path fading=fade img left] (5.6, -2.2) rectangle (4.59, 3.51);
  \fill[white,path fading=fade img left] (5.6, -2.2) rectangle (4.59, 3.51);

  \path (5.0, 2.3) node[left,purple,align=center] {\textbf{Tweezer}};
  \path (4.8, 0.5) node[left,align=center] {\textbf{Raman}};
  \path (5.15, -1.5) node[left,orange,align=center] {\textbf{Optical}\\\textbf{pumping}};
  \draw[line width=2,orange,->] (5.6,-2) -- (17.8,-2) node[right, below] {\scalebox{2}{$t$}};

  %% Cycle mark
  \draw[decoration={brace,mirror,raise=3pt, amplitude=10pt},decorate,line width=2]
  (5.6,-2) -- node[below=15pt] {\scalebox{1.5}{One cooling cycle}} (16.8,-2);
\end{tikzpicture}
\end{document}
