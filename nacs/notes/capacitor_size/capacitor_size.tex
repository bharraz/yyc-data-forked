\documentclass[10pt,fleqn]{article}
% \usepackage[journal=rsc]{chemstyle}
% \usepackage{mhchem}
\usepackage{amsmath}
\usepackage{amssymb}
\usepackage{amsfonts}
\usepackage{esint}
\usepackage{bbm}
\usepackage{amscd}
\usepackage{picinpar}
\usepackage{graphicx}
\usepackage{tikz}
\usepackage{tikz-3dplot}
\usepackage{indentfirst}
\usepackage{wrapfig}
\usepackage{units}
\usepackage{textcomp}
\usepackage[utf8x]{inputenc}
% \usepackage{feyn}
\usepackage{feynmp}
\usepackage{xkeyval}
\usepackage{xargs}
\usepackage{verbatim}
\usepackage{pgfplots}
\usepackage{hyperref}
\usetikzlibrary{
  arrows,
  calc,
  decorations.pathmorphing,
  decorations.pathreplacing,
  decorations.markings,
  fadings,
  positioning,
  shapes
}

\DeclareGraphicsRule{*}{mps}{*}{}
\newcommand{\ud}{\mathrm{d}}
\newcommand{\ue}{\mathrm{e}}
\newcommand{\ui}{\mathrm{i}}
\newcommand{\res}{\mathrm{Res}}
\newcommand{\Tr}{\mathrm{Tr}}
\newcommand{\dsum}{\displaystyle\sum}
\newcommand{\dprod}{\displaystyle\prod}
\newcommand{\dlim}{\displaystyle\lim}
\newcommand{\dint}{\displaystyle\int}
\newcommand{\fsno}[1]{{\!\not\!{#1}}}
\newcommand{\eqar}[1]
{
  \begin{align*}
    #1
  \end{align*}
}
\newcommand{\texp}[2]{\ensuremath{{#1}\times10^{#2}}}
\newcommand{\dexp}[2]{\ensuremath{{#1}\cdot10^{#2}}}
\newcommand{\eval}[2]{{\left.{#1}\right|_{#2}}}
\newcommand{\paren}[1]{{\left({#1}\right)}}
\newcommand{\lparen}[1]{{\left({#1}\right.}}
\newcommand{\rparen}[1]{{\left.{#1}\right)}}
\newcommand{\abs}[1]{{\left|{#1}\right|}}
\newcommand{\sqr}[1]{{\left[{#1}\right]}}
\newcommand{\crly}[1]{{\left\{{#1}\right\}}}
\newcommand{\angl}[1]{{\left\langle{#1}\right\rangle}}
\newcommand{\tpdiff}[4][{}]{{\paren{\frac{\partial^{#1} {#2}}{\partial {#3}{}^{#1}}}_{#4}}}
\newcommand{\tpsdiff}[4][{}]{{\paren{\frac{\partial^{#1}}{\partial {#3}{}^{#1}}{#2}}_{#4}}}
\newcommand{\pdiff}[3][{}]{{\frac{\partial^{#1} {#2}}{\partial {#3}{}^{#1}}}}
\newcommand{\diff}[3][{}]{{\frac{\ud^{#1} {#2}}{\ud {#3}{}^{#1}}}}
\newcommand{\psdiff}[3][{}]{{\frac{\partial^{#1}}{\partial {#3}{}^{#1}} {#2}}}
\newcommand{\sdiff}[3][{}]{{\frac{\ud^{#1}}{\ud {#3}{}^{#1}} {#2}}}
\newcommand{\tpddiff}[4][{}]{{\left(\dfrac{\partial^{#1} {#2}}{\partial {#3}{}^{#1}}\right)_{#4}}}
\newcommand{\tpsddiff}[4][{}]{{\paren{\dfrac{\partial^{#1}}{\partial {#3}{}^{#1}}{#2}}_{#4}}}
\newcommand{\pddiff}[3][{}]{{\dfrac{\partial^{#1} {#2}}{\partial {#3}{}^{#1}}}}
\newcommand{\ddiff}[3][{}]{{\dfrac{\ud^{#1} {#2}}{\ud {#3}{}^{#1}}}}
\newcommand{\psddiff}[3][{}]{{\frac{\partial^{#1}}{\partial{}^{#1} {#3}} {#2}}}
\newcommand{\sddiff}[3][{}]{{\frac{\ud^{#1}}{\ud {#3}{}^{#1}} {#2}}}
\usepackage{fancyhdr}
\usepackage{multirow}
\usepackage{fontenc}
% \usepackage{tipa}
\usepackage{ulem}
\usepackage{color}
\usepackage{cancel}
\newcommand{\hcancel}[2][black]{\setbox0=\hbox{#2}%
  \rlap{\raisebox{.45\ht0}{\textcolor{#1}{\rule{\wd0}{1pt}}}}#2}
\pagestyle{fancy}
\setlength{\headheight}{67pt}
\fancyhead{}
\fancyfoot{}
\fancyfoot[C]{\thepage}
\fancyhead[R]{}
\renewcommand{\footruleskip}{0pt}
\renewcommand{\headrulewidth}{0.4pt}
\renewcommand{\footrulewidth}{0pt}

\newcommand\pgfmathsinandcos[3]{%
  \pgfmathsetmacro#1{sin(#3)}%
  \pgfmathsetmacro#2{cos(#3)}%
}
\newcommand\LongitudePlane[3][current plane]{%
  \pgfmathsinandcos\sinEl\cosEl{#2} % elevation
  \pgfmathsinandcos\sint\cost{#3} % azimuth
  \tikzset{#1/.estyle={cm={\cost,\sint*\sinEl,0,\cosEl,(0,0)}}}
}
\newcommand\LatitudePlane[3][current plane]{%
  \pgfmathsinandcos\sinEl\cosEl{#2} % elevation
  \pgfmathsinandcos\sint\cost{#3} % latitude
  \pgfmathsetmacro\yshift{\cosEl*\sint}
  \tikzset{#1/.estyle={cm={\cost,0,0,\cost*\sinEl,(0,\yshift)}}} %
}
\newcommand\DrawLongitudeCircle[2][1]{
  \LongitudePlane{\angEl}{#2}
  \tikzset{current plane/.prefix style={scale=#1}}
  % angle of "visibility"
  \pgfmathsetmacro\angVis{atan(sin(#2)*cos(\angEl)/sin(\angEl))} %
  \draw[current plane] (\angVis:1) arc (\angVis:\angVis+180:1);
  \draw[current plane,dashed] (\angVis-180:1) arc (\angVis-180:\angVis:1);
}
\newcommand\DrawLatitudeCircleArrow[2][1]{
  \LatitudePlane{\angEl}{#2}
  \tikzset{current plane/.prefix style={scale=#1}}
  \pgfmathsetmacro\sinVis{sin(#2)/cos(#2)*sin(\angEl)/cos(\angEl)}
  % angle of "visibility"
  \pgfmathsetmacro\angVis{asin(min(1,max(\sinVis,-1)))}
  \draw[current plane,decoration={markings, mark=at position 0.6 with {\arrow{<}}},postaction={decorate},line width=.6mm] (\angVis:1) arc (\angVis:-\angVis-180:1);
  \draw[current plane,dashed,line width=.6mm] (180-\angVis:1) arc (180-\angVis:\angVis:1);
}
\newcommand\DrawLatitudeCircle[2][1]{
  \LatitudePlane{\angEl}{#2}
  \tikzset{current plane/.prefix style={scale=#1}}
  \pgfmathsetmacro\sinVis{sin(#2)/cos(#2)*sin(\angEl)/cos(\angEl)}
  % angle of "visibility"
  \pgfmathsetmacro\angVis{asin(min(1,max(\sinVis,-1)))}
  \draw[current plane] (\angVis:1) arc (\angVis:-\angVis-180:1);
  \draw[current plane,dashed] (180-\angVis:1) arc (180-\angVis:\angVis:1);
}
\newcommand\coil[1]{
  {\rh * cos(\t * pi r)}, {\apart * (2 * #1 + \t) + \rv * sin(\t * pi r)}
}
\makeatletter
\define@key{DrawFromCenter}{style}[{->}]{
  \tikzset{DrawFromCenterPlane/.style={#1}}
}
\define@key{DrawFromCenter}{r}[1]{
  \def\@R{#1}
}
\define@key{DrawFromCenter}{center}[(0, 0)]{
  \def\@Center{#1}
}
\define@key{DrawFromCenter}{theta}[0]{
  \def\@Theta{#1}
}
\define@key{DrawFromCenter}{phi}[0]{
  \def\@Phi{#1}
}
\presetkeys{DrawFromCenter}{style, r, center, theta, phi}{}
\newcommand*\DrawFromCenter[1][]{
  \setkeys{DrawFromCenter}{#1}{
    \pgfmathsinandcos\sint\cost{\@Theta}
    \pgfmathsinandcos\sinp\cosp{\@Phi}
    \pgfmathsinandcos\sinA\cosA{\angEl}
    \pgfmathsetmacro\DX{\@R*\cost*\cosp}
    \pgfmathsetmacro\DY{\@R*(\cost*\sinp*\sinA+\sint*\cosA)}
    \draw[DrawFromCenterPlane] \@Center -- ++(\DX, \DY);
  }
}
\newcommand*\DrawFromCenterText[2][]{
  \setkeys{DrawFromCenter}{#1}{
    \pgfmathsinandcos\sint\cost{\@Theta}
    \pgfmathsinandcos\sinp\cosp{\@Phi}
    \pgfmathsinandcos\sinA\cosA{\angEl}
    \pgfmathsetmacro\DX{\@R*\cost*\cosp}
    \pgfmathsetmacro\DY{\@R*(\cost*\sinp*\sinA+\sint*\cosA)}
    \draw[DrawFromCenterPlane] \@Center -- ++(\DX, \DY) node {#2};
  }
}
\makeatother
\tikzstyle{snakearrow} = [decorate, decoration={pre length=0.2cm,
  post length=0.2cm, snake, amplitude=.4mm,
  segment length=2mm},thick, ->]
%% document-wide tikz options and styles
\tikzset{%
  >=latex, % option for nice arrows
  inner sep=0pt,%
  outer sep=2pt,%
  mark coordinate/.style={inner sep=0pt,outer sep=0pt,minimum size=3pt,
    fill=black,circle}%
}
\addtolength{\hoffset}{-1.3cm}
\addtolength{\voffset}{-2cm}
\addtolength{\textwidth}{3cm}
\addtolength{\textheight}{2.5cm}
\renewcommand{\footskip}{10pt}
\setlength{\headwidth}{\textwidth}
\setlength{\headsep}{20pt}
\setlength{\marginparwidth}{0pt}
\parindent=0pt
\title{Why do capacitors usually only have a small value}
\author{Yichao Yu}
\begin{document}

\maketitle

In Lab 1 of PS3, there is a question about why capacitors usually has a small value.
AFAIK, the student is supposed to relate this to the smallness of the electron charge in
SI unit which I find to be problematic.
This is a note that list all reasons I think such a question is problematic and some
better answers that I can come up with.

\section{Why such a explanation is problematic}

To start, super-capacitors exist, and may have a capacitance of hundreds of $F$
within a small volume.
This alone suggests that this question is more about material properties that
are not within the scope of this class.

Now about the expected answer. Electron charge is about $1.602\times10^{-19}C$,
if we were to use this as an explanation for why $C$ is such a large unit
(which in turn means that $F = C / V$ is a large unit) we would expect capacitors to
have a capacitance value on the order of $10^{-19}$, which is wrong for at least two reasons,

\begin{enumerate}
\item They are not.\\
  In fact, due to parasitic capacitance, it is very hard to get stand alone capacitance smaller
  than few $pF$.
\item $1e$ is much smaller than the amount of charge we expect a normal capacitor to store.\\
  And I don't believe there is an obvious way for the students to do a correct estimation
  of how much charge the conductor in a capacitor can hold. (One is given below).
\end{enumerate}

Therefore, while electron charge being small indeed put some limit on how much charge a normal
capacitor can store.
It is non-trivial to actual do so and certainly not for students in this class.
This is not even mentioning (see below) that this limit isn't the limiting factors in
capacitor manufacturing.

It is good to give students an idea of the magnitude of different quantities, electron charge,
capacitor values, safe voltages, etc.
However, it is better to do so with more direct arguments. Even if the argument related to
electron charge is what limits the value of conventional capacitors, it's not,
it shouldn't be used as an example if the student cannot fully understand the argument,
and they can't.
Doing so would encourage them to skip the logic and make incorrect connections between
quantities, which IMHO is a bad habit that is very hard to correct later on.

\section{The actual reason}

\subsection{TL;DR}

It's complicated. Since units are arbitrarily defined, this question is essentially asking why
the unit systems are designed in the way they are. Additionally, since the unit system
was more-or-less defined based on the quantity that makes sense on human scale,
we necessarily need to consider what is the correct human scale historically chosen.
Below I present a few of the limiting factors based on the best I know of the unit system
and material science.
Note that I do not believe any of the following explanations are understandable for
students taking PS3 (at this stage, or ever) and they are given mainly to show
how complicated the issue is. It should hopefully also be an interesting read.

\subsection{Limit from breakdown}
In principle, this does not really limit the capacitor size since one can in principle always
make a larger capacitor with lower breakdown voltage until running into manufacture limit,
something I'll definitely \textbf{NOT} get into.
Nevertheless, these are included here since the breakdown is the only thing
we could remotely relate the electron charge value to.

\subsubsection{The amount of charge one can store in a conductor}
This is the closest limitation I (and Nick)
could come up with that is related to electron charge.

The smallness of the electron charge means that a lot of electrons needs to be added
or removed from a conductor to hold a large amount of charge. Since all the net charges
are stored on the surface of the conductor in equilibrium, what we need is basically
the maximum 2D charge density on a normal conductor.
A very simple model that should give the right order of magnitude is to assume the metal can
afford a single layer of electron or positive ion to stick to a neutral surface.
Based on the density and atomic mass of copper, the density of copper atoms is roughly

\eqar{
  n_{Cu}=&\rho_{Cu} / m_{Cu}\\
  =&8.96\mathrm{g}\cdot \mathrm{cm}^{-3} / 63.546\mathrm{g} \cdot 6.02\cdot10^{23}\\
  =&8.49\cdot10^{22}\mathrm{cm}^{-3}
  \intertext{Which corresponds to a surface charge density of}
  \sigma_{Cu}=&n_{Cu}^{2/3}\\
  =&1.93\cdot10^{15}\mathrm{cm}^{-2}
}

With our assumption of a single charge layer, this corresponds to a surface electric field of
\eqar{
  E=&\frac{e\sigma_{Cu}}{\varepsilon_0}\\
  =&3.49\cdot10^{11}\mathrm{V}\cdot\mathrm{m}^{-1}
}
Any field higher than this will likely lead to field electron or ion emission.
The number agrees in orders of magnitude with the typical field found in field ion
or electron microscope which is ``a few V/\AA''.
(Note that the in field ion microscope the background gas is being ionized
and not the material surface so the material can likely afford a stronger field)

(Note that the argument here does not fully apply to electrolytic capacitor since
the charge is partially stored as ions)

This limits the energy density (since its proportional to field square) of a capacitor
but it's very hard to convert this to a capacitance since it's strongly manufacture process
dependent. However, it is possible to compare this to the other effect that limits the
energy density, dielectric strength.

\subsubsection{Dielectric strength}
This is the max field that can be applied on a material before breakdown occurs.
It depends strongly on the material so I do not have a good way to estimate the value.
However, from a table on Wikipedia, the typical values are between
$10^{6}\ \text{and}\ 10^{9}\mathrm{V}\cdot\mathrm{m}^{-1}$,
much smaller than the ion/electron emission threshold of conductors so this should be
the limiting factor of the energy density in most (all) cases.

\subsection{Actual reason, for real}
As previously mentioned, the reason one typically get a small capacitor value in SI unit is,
well, the SI unit itself. Since the size of a plate capacitor with $\varepsilon_r\approx1$
is given by $\dfrac{\varepsilon_0 S}{d}$, this is really about why $\varepsilon_0$ has a tiny
value in SI unit. As Nick pointed out, in order to understand this, we need to realize that
the charge unit in SI is defined based on Ampere, which is defined based on force between
current. Since $\dfrac{1}{\varepsilon_0}$ qualify the electric static force between a unit
amount of charge, the reason capacitors are small is really because for the current
that generate the ``standard'' (defined later) amount of force between two wires, the force
between the charge delivered by that current in 1 second is large in SI unit.\\

We can directly calculate the ratio of the two forces with a time scale of $t=1\mathrm{s}$
and length scale of $l=1\mathrm{m}$
\eqar{
  F_{mag}=&\frac{\mu_0 I^2}{2\pi L}L\\
  =&\frac{\mu_0 I^2}{2\pi}\\
  F_{ele}=&\frac{Q^2}{4\pi\varepsilon_0 L^2}\\
  =&\frac{I^2t^2}{4\pi\varepsilon_0 L^2}\\
  \dfrac{F_{ele}}{F_{mag}}=&\frac{I^2t^2}{4\pi\varepsilon_0 L^2}\frac{2\pi}{\mu_0 I^2}\\
  =&\frac{c^2t^2}{2L^2}\\
  =&\frac{c^2}{v^2}
}
where $v\equiv\sqrt{2}\dfrac{L}{t}$ is a human velocity scale.
Here we clearly see that the electronic static interaction is much larger than
magnetic interaction \textbf{for unit charge and current in SI unit} because
the SI unit is defined with human length and time scale which gives a velocity scale
much smaller than the speed of light. In fact, the ``standard'' force between
two current carrying wires is already much smaller than $1\mathrm{N}$ but it is not enough
to make the corresponding electric static interaction small enough.

Note that the discussion above was based on the definition of Ampere.
However, given that the standard force generated between wires is much smaller than $1\mathrm{N}$,
it is likely that the rough magnitude of the Ampere unit was historically selected
for other reasons, likely related to how current is generated back then.
I could not find a good note about relation between early generator and
current unit definition and since this strongly depends on the winding numbers and
early permanent magnet I would stop speculating how it is selected.

\end{document}
