\documentclass[10pt,fleqn]{article}
% \usepackage[journal=rsc]{chemstyle}
% \usepackage{mhchem}
\usepackage{amsmath}
\usepackage{amssymb}
\usepackage{amsfonts}
\usepackage{esint}
\usepackage{bbm}
\usepackage{amscd}
\usepackage{picinpar}
\usepackage{graphicx}
\usepackage{tikz}
\usepackage{tikz-3dplot}
\usepackage{indentfirst}
\usepackage{wrapfig}
\usepackage{units}
\usepackage{textcomp}
\usepackage[utf8x]{inputenc}
% \usepackage{feyn}
\usepackage{feynmp}
\usepackage{xkeyval}
\usepackage{xargs}
\usepackage{verbatim}
\usepackage{pgfplots}
\usepackage{hyperref}
\usetikzlibrary{
  arrows,
  calc,
  decorations.pathmorphing,
  decorations.pathreplacing,
  decorations.markings,
  fadings,
  positioning,
  shapes
}

\DeclareGraphicsRule{*}{mps}{*}{}
\newcommand{\ud}{\mathrm{d}}
\newcommand{\ue}{\mathrm{e}}
\newcommand{\ui}{\mathrm{i}}
\newcommand{\res}{\mathrm{Res}}
\newcommand{\Tr}{\mathrm{Tr}}
\newcommand{\dsum}{\displaystyle\sum}
\newcommand{\dprod}{\displaystyle\prod}
\newcommand{\dlim}{\displaystyle\lim}
\newcommand{\dint}{\displaystyle\int}
\newcommand{\fsno}[1]{{\!\not\!{#1}}}
\newcommand{\eqar}[1]
{
  \begin{align*}
    #1
  \end{align*}
}
\newcommand{\texp}[2]{\ensuremath{{#1}\times10^{#2}}}
\newcommand{\dexp}[2]{\ensuremath{{#1}\cdot10^{#2}}}
\newcommand{\eval}[2]{{\left.{#1}\right|_{#2}}}
\newcommand{\paren}[1]{{\left({#1}\right)}}
\newcommand{\lparen}[1]{{\left({#1}\right.}}
\newcommand{\rparen}[1]{{\left.{#1}\right)}}
\newcommand{\abs}[1]{{\left|{#1}\right|}}
\newcommand{\sqr}[1]{{\left[{#1}\right]}}
\newcommand{\crly}[1]{{\left\{{#1}\right\}}}
\newcommand{\angl}[1]{{\left\langle{#1}\right\rangle}}
\newcommand{\tpdiff}[4][{}]{{\paren{\frac{\partial^{#1} {#2}}{\partial {#3}{}^{#1}}}_{#4}}}
\newcommand{\tpsdiff}[4][{}]{{\paren{\frac{\partial^{#1}}{\partial {#3}{}^{#1}}{#2}}_{#4}}}
\newcommand{\pdiff}[3][{}]{{\frac{\partial^{#1} {#2}}{\partial {#3}{}^{#1}}}}
\newcommand{\diff}[3][{}]{{\frac{\ud^{#1} {#2}}{\ud {#3}{}^{#1}}}}
\newcommand{\psdiff}[3][{}]{{\frac{\partial^{#1}}{\partial {#3}{}^{#1}} {#2}}}
\newcommand{\sdiff}[3][{}]{{\frac{\ud^{#1}}{\ud {#3}{}^{#1}} {#2}}}
\newcommand{\tpddiff}[4][{}]{{\left(\dfrac{\partial^{#1} {#2}}{\partial {#3}{}^{#1}}\right)_{#4}}}
\newcommand{\tpsddiff}[4][{}]{{\paren{\dfrac{\partial^{#1}}{\partial {#3}{}^{#1}}{#2}}_{#4}}}
\newcommand{\pddiff}[3][{}]{{\dfrac{\partial^{#1} {#2}}{\partial {#3}{}^{#1}}}}
\newcommand{\ddiff}[3][{}]{{\dfrac{\ud^{#1} {#2}}{\ud {#3}{}^{#1}}}}
\newcommand{\psddiff}[3][{}]{{\frac{\partial^{#1}}{\partial{}^{#1} {#3}} {#2}}}
\newcommand{\sddiff}[3][{}]{{\frac{\ud^{#1}}{\ud {#3}{}^{#1}} {#2}}}
\usepackage{fancyhdr}
\usepackage{multirow}
\usepackage{fontenc}
% \usepackage{tipa}
\usepackage{ulem}
\usepackage{color}
\usepackage{cancel}
\newcommand{\hcancel}[2][black]{\setbox0=\hbox{#2}%
  \rlap{\raisebox{.45\ht0}{\textcolor{#1}{\rule{\wd0}{1pt}}}}#2}
\pagestyle{fancy}
\setlength{\headheight}{67pt}
\fancyhead{}
\fancyfoot{}
\fancyfoot[C]{\thepage}
\fancyhead[R]{}
\renewcommand{\footruleskip}{0pt}
\renewcommand{\headrulewidth}{0.4pt}
\renewcommand{\footrulewidth}{0pt}

\newcommand\pgfmathsinandcos[3]{%
  \pgfmathsetmacro#1{sin(#3)}%
  \pgfmathsetmacro#2{cos(#3)}%
}
\newcommand\LongitudePlane[3][current plane]{%
  \pgfmathsinandcos\sinEl\cosEl{#2} % elevation
  \pgfmathsinandcos\sint\cost{#3} % azimuth
  \tikzset{#1/.estyle={cm={\cost,\sint*\sinEl,0,\cosEl,(0,0)}}}
}
\newcommand\LatitudePlane[3][current plane]{%
  \pgfmathsinandcos\sinEl\cosEl{#2} % elevation
  \pgfmathsinandcos\sint\cost{#3} % latitude
  \pgfmathsetmacro\yshift{\cosEl*\sint}
  \tikzset{#1/.estyle={cm={\cost,0,0,\cost*\sinEl,(0,\yshift)}}} %
}
\newcommand\DrawLongitudeCircle[2][1]{
  \LongitudePlane{\angEl}{#2}
  \tikzset{current plane/.prefix style={scale=#1}}
  % angle of "visibility"
  \pgfmathsetmacro\angVis{atan(sin(#2)*cos(\angEl)/sin(\angEl))} %
  \draw[current plane] (\angVis:1) arc (\angVis:\angVis+180:1);
  \draw[current plane,dashed] (\angVis-180:1) arc (\angVis-180:\angVis:1);
}
\newcommand\DrawLatitudeCircleArrow[2][1]{
  \LatitudePlane{\angEl}{#2}
  \tikzset{current plane/.prefix style={scale=#1}}
  \pgfmathsetmacro\sinVis{sin(#2)/cos(#2)*sin(\angEl)/cos(\angEl)}
  % angle of "visibility"
  \pgfmathsetmacro\angVis{asin(min(1,max(\sinVis,-1)))}
  \draw[current plane,decoration={markings, mark=at position 0.6 with {\arrow{<}}},postaction={decorate},line width=.6mm] (\angVis:1) arc (\angVis:-\angVis-180:1);
  \draw[current plane,dashed,line width=.6mm] (180-\angVis:1) arc (180-\angVis:\angVis:1);
}
\newcommand\DrawLatitudeCircle[2][1]{
  \LatitudePlane{\angEl}{#2}
  \tikzset{current plane/.prefix style={scale=#1}}
  \pgfmathsetmacro\sinVis{sin(#2)/cos(#2)*sin(\angEl)/cos(\angEl)}
  % angle of "visibility"
  \pgfmathsetmacro\angVis{asin(min(1,max(\sinVis,-1)))}
  \draw[current plane] (\angVis:1) arc (\angVis:-\angVis-180:1);
  \draw[current plane,dashed] (180-\angVis:1) arc (180-\angVis:\angVis:1);
}
\newcommand\coil[1]{
  {\rh * cos(\t * pi r)}, {\apart * (2 * #1 + \t) + \rv * sin(\t * pi r)}
}
\makeatletter
\define@key{DrawFromCenter}{style}[{->}]{
  \tikzset{DrawFromCenterPlane/.style={#1}}
}
\define@key{DrawFromCenter}{r}[1]{
  \def\@R{#1}
}
\define@key{DrawFromCenter}{center}[(0, 0)]{
  \def\@Center{#1}
}
\define@key{DrawFromCenter}{theta}[0]{
  \def\@Theta{#1}
}
\define@key{DrawFromCenter}{phi}[0]{
  \def\@Phi{#1}
}
\presetkeys{DrawFromCenter}{style, r, center, theta, phi}{}
\newcommand*\DrawFromCenter[1][]{
  \setkeys{DrawFromCenter}{#1}{
    \pgfmathsinandcos\sint\cost{\@Theta}
    \pgfmathsinandcos\sinp\cosp{\@Phi}
    \pgfmathsinandcos\sinA\cosA{\angEl}
    \pgfmathsetmacro\DX{\@R*\cost*\cosp}
    \pgfmathsetmacro\DY{\@R*(\cost*\sinp*\sinA+\sint*\cosA)}
    \draw[DrawFromCenterPlane] \@Center -- ++(\DX, \DY);
  }
}
\newcommand*\DrawFromCenterText[2][]{
  \setkeys{DrawFromCenter}{#1}{
    \pgfmathsinandcos\sint\cost{\@Theta}
    \pgfmathsinandcos\sinp\cosp{\@Phi}
    \pgfmathsinandcos\sinA\cosA{\angEl}
    \pgfmathsetmacro\DX{\@R*\cost*\cosp}
    \pgfmathsetmacro\DY{\@R*(\cost*\sinp*\sinA+\sint*\cosA)}
    \draw[DrawFromCenterPlane] \@Center -- ++(\DX, \DY) node {#2};
  }
}
\makeatother
\tikzstyle{snakearrow} = [decorate, decoration={pre length=0.2cm,
  post length=0.2cm, snake, amplitude=.4mm,
  segment length=2mm},thick, ->]
%% document-wide tikz options and styles
\tikzset{%
  >=latex, % option for nice arrows
  inner sep=0pt,%
  outer sep=2pt,%
  mark coordinate/.style={inner sep=0pt,outer sep=0pt,minimum size=3pt,
    fill=black,circle}%
}
\addtolength{\hoffset}{-1.3cm}
\addtolength{\voffset}{-2cm}
\addtolength{\textwidth}{3cm}
\addtolength{\textheight}{2.5cm}
\renewcommand{\footskip}{10pt}
\setlength{\headwidth}{\textwidth}
\setlength{\headsep}{20pt}
\setlength{\marginparwidth}{0pt}
\parindent=0pt
\title{Fit model for diabatic lowering}
\begin{document}

\section{Classical model}
\subsection{Energy distribution}

Maxwell-Boltzmann distribution in a 3D harmonic trap.

Let $\varepsilon = \beta E$

\href{https://en.wikipedia.org/wiki/Gas_in_a_harmonic_trap#Massive_Maxwell.E2.80.93Boltzmann_particles}{PDF}: $p(\varepsilon)\ud\varepsilon = \dfrac{\varepsilon^2}{2}\ue^{-\varepsilon}\ud\varepsilon$

CDF (integrating PDF): $\mathbf{cdf}(\varepsilon) = 1 - \left(1 + \varepsilon + \dfrac{\varepsilon^2}{2}\right) \ue^{-\varepsilon}$

\subsection{Model of diabatic lowering}

When the lowering is fast (diabatic) enough, assume the wavefunction (motion) of
the particle doesn't change. Based on the virial theorem, half of the energy
is potiential energy, which is lowered by the lowering factor, and the other half
is kinetic energy, which does not change right after lowering so the final energy
of the particle with initial energy $E$ is $E' = \dfrac{\gamma + 1}{2}E$,
where $\gamma$ is the lowering factor.

\subsection{Model of survival}

Assume that the atom will leave the trap if its final energy is higher than
the final trap depth.

\subsection{The full experimental sequence}

The atom starts in the trap with trap depth $E_{trap}$ and initial energy $E_0$.

The trap depth is first ramped adabatically by a factor of $\alpha$
(to $\alpha E_{trap}$) ($\alpha = 1$ if this step is skipped).
The energy of the atom is $\sqrt{\alpha}E_0$ due to adabatic compression.

The trap depth is then diabatically lowered to $\gamma E_{trap}$, the total energy
of the atom is now $\dfrac{\alpha + \gamma}{2\sqrt{\alpha}}E_0$. The atom will
stay in the trap if

$$\frac{\alpha + \gamma}{2\sqrt{\alpha}}E_0 < \gamma E_{trap}$$

or

$$\varepsilon_0 < \frac{2\gamma\sqrt{\alpha}}{\alpha + \gamma}\varepsilon_{trap}$$

where $\varepsilon_0$ and $\varepsilon_{trap}$ are the initial energy of the atom
and the initial trap depth measured in unit of the initial temperature.

Therefore the total survival probability is,
$$p_0 \mathbf{cdf}\left(\frac{2\gamma\sqrt{\alpha}}{(\alpha + \gamma) \tau}\right)$$

where $p_0$ models the overall survival probability and
$\tau = \varepsilon_{trap}^{-1}$ is the temperature measured in trap depth.

\section{Alternative classical model}

Consider the survival probability in 1D and cube it to approximate the 3d survival
probability.

This is likely less accurate and might overestimate the energy since it
increases the escape condition from energy larger than trap depth to
energy in at least one axis larger than trap depth.

\subsection{Energy distribution}

Maxwell-Boltzmann distribution in a 1D harmonic trap.

Let $\varepsilon = \beta E$, where $E$ in the degrees of freedoms
in this dimension.

PDF: $p(\varepsilon)\ud\varepsilon = \ue^{-\varepsilon}\ud\varepsilon$

CDF (integrating PDF): $\mathbf{cdf}(\varepsilon) = 1 - \ue^{-\varepsilon}$

\subsection{The full experimental sequence}

The atom starts in the trap with 1D trap depth $E_{trap}$
and initial 1D energy $E_0$.

The 1D trap depth is then diabatically lowered to $\gamma E_{trap}$,
the 1D energy of the atom is now $\dfrac{1 + \gamma}{2}E_0$. The atom will
stay in the trap in this dimension if

$$\frac{1 + \gamma}{2}E_0 < \gamma E_{trap}$$

or

$$\varepsilon_0 < \frac{2\gamma}{1 + \gamma}\varepsilon_{trap}$$

where $\varepsilon_0$ and $\varepsilon_{trap}$ are the initial 1D energy
of the atom and the initial 1D trap depth measured in unit of the initial
temperature.

Therefore the total survival probability is,
$$p_0 \mathbf{cdf}\left(\frac{2\gamma}{(1 + \gamma) \tau}\right)^3$$

where $p_0$ models the overall survival probability and
$\tau = \varepsilon_{trap}^{-1}$ is the temperature measured in 1D trap depth.

\section{A more quantum model}

The classical models above ignores the fact that the state after the lowering does not have to be either trapped or not but it can have non-zero overlap with both the trapped and untrapped state. We can get a slightly better model using the coherent states of the initial and the final trap and calculate the survival rate using the overlap between the states.

\subsection{Density matrix}

The density matrix of the initial state is

\eqar{
  \rho=&\paren{1-\ue^{-\varepsilon}}\sum_{n}|n\rangle\ue^{-n\varepsilon}\langle n|
}

where $\varepsilon\equiv\hbar\omega\beta$.

We need to express this in the basis of coherent states. Since the coherent states form an overcomplete basis, there are multiple ways to express this. We can assume that the expression takes the form,

\eqar{
  \rho=&\int|\alpha\rangle f(\abs{\alpha}) \langle\alpha|\ud^2\alpha
}

i.e. there's only diagonal term. Multiply this by the eigen vectors of $H$,

\eqar{
  \langle m|\rho|n\rangle=&\int\langle m|\alpha\rangle f(\abs{\alpha}) \langle\alpha|n\rangle\ud^2\alpha\\
  =&\int \ue^{-\abs{\alpha}^2} \frac{\alpha^m}{\sqrt{m!}} f(\abs{\alpha}) \frac{{\alpha^*}^n}{\sqrt{n!}}\ud^2\alpha\\
  =&\int \ue^{-\abs{\alpha}^2} \frac{\abs{\alpha}^{m+n}\ue^{\ui(m-n)\theta}}{\sqrt{m!n!}} f(\abs{\alpha}) \abs{\alpha}\ud\theta\ud\abs{\alpha}\\
  =&\pi\delta^{mn}\int \ue^{-\abs{\alpha}^2} \frac{\abs{\alpha}^{2n}}{n!} f(\abs{\alpha}) \ud\abs{\alpha}^2
  \intertext{Compare to the expression in energy basis}
  \langle m|\rho|n\rangle=&\delta^{mn}\paren{1-\ue^{-\varepsilon}}\ue^{-n\varepsilon}
}
\eqar{
  \paren{1-\ue^{-\varepsilon}}\ue^{-n\varepsilon}=&\pi\int \ue^{-\abs{\alpha}^2} \frac{\abs{\alpha}^{2n}}{n!} f(\abs{\alpha}) \ud\abs{\alpha}^2
}
Take a guess that $f(\abs{\alpha})=c_1\ue^{c_2\abs{\alpha}^2}$ and do the integral,
we can get
\eqar{
  f\paren{\abs{\alpha}}=&\frac{\ue^{\varepsilon} - 1}{\pi}\exp\paren{-\abs{\alpha}^2\paren{\ue^{\varepsilon} - 1}}
}

\subsection{Squeezed state}
The new coherent states in the final potential are squeezed states in the original trap. The squeeze factor can be obtained from
\eqar{
  \ue^{-2r}=&\frac{\omega_0}{\omega'}\\
  r=&-\frac12\ln\paren{\frac{\omega_0}{\omega'}}
}
where $\omega'$ is the final trapping frequency.
The new ground state is,
\eqar{
  |0\rangle'=&S\paren{r}|0\rangle
}
where $\displaystyle S\equiv\exp\paren{\frac{r}2\paren{a^2-{a^\dagger}^2}}$ is the squeeze operator.
The new coherent state with amplitude $\alpha'$ is
\eqar{
  |\alpha'\rangle'=&S\paren{r}|\alpha'\rangle\\
  =&S\paren{r}D\paren{\alpha'}|0\rangle\\
  =&S\paren{r}\exp\paren{\alpha'a^\dagger-\alpha'^*a}|0\rangle\\
  =&S\paren{r}\exp\paren{\alpha'a^\dagger-\alpha'^*a}S^\dagger\paren{r}S\paren{r}|0\rangle\\
  =&\exp\paren{\alpha'\paren{\cosh r a^\dagger+\sinh r a}-\alpha'^*\paren{\cosh r a+\sinh r a^\dagger}}S\paren{r}|0\rangle\\
  =&\exp\paren{\paren{\alpha'\cosh r-\alpha'^*\sinh r}a^\dagger-\paren{\alpha'^*\cosh r-\alpha'\sinh r}a}S\paren{r}|0\rangle\\
  =&D\paren{\alpha'\cosh r-\alpha'^*\sinh r}S\paren{r}|0\rangle
}
The overlap between this state and a coherent state is
\eqar{
  \langle\alpha|\alpha'\rangle'=&\langle0|D\paren{-\alpha}D\paren{\alpha'\cosh r-\alpha'^*\sinh r}S\paren{r}|0\rangle\\
  =&\langle0|D\paren{-\alpha+\alpha'\cosh r-\alpha'^*\sinh r}S\paren{r}|0\rangle
}
Let $\alpha''\equiv\alpha-\alpha'\cosh r+\alpha'^*\sinh r$
\eqar{
  \langle\alpha|\alpha'\rangle'=&\langle0|D\paren{-\alpha''}S\paren{r}|0\rangle\\
  =&\ue^{-\abs{\alpha''}^2/2}\sum_{n}\langle n|\frac{{\alpha''}^n}{\sqrt{n!}}
  \frac1{\sqrt{\cosh r}}\sum_{m}\frac{\paren{-1}^m\sqrt{(2m)!}}{2^mm!}\tanh^mr|2m\rangle\\
  =&\frac{\ue^{-\abs{\alpha''}^2/2}}{\sqrt{\cosh r}}\sum_{n}\frac{1}{n!}\paren{-\frac{\alpha''^2}{2}\tanh r}^n\\
  =&\frac{1}{\sqrt{\cosh r}}\exp\paren{-\frac{\abs{\alpha''}^2}2-\frac{\alpha''^2}{2}\tanh r}
}
\subsection{Survival}
The survival after lowering in this model for a threshold $\gamma$
\eqar{
  p=&\frac1\pi\Tr\paren{\int_{\abs{\alpha'}\leqslant\gamma}|\alpha'\rangle'\ud^2\alpha'\langle\alpha'|'\rho}\\
  =&\frac1\pi\int_{\abs{\alpha'}\leqslant\gamma}\langle\alpha'|'\rho|\alpha'\rangle'\ud^2\alpha'\\
  =&\frac{\ue^{\varepsilon} - 1}{\pi^2}\int\ud^2\alpha\int_{\abs{\alpha'}\leqslant\gamma}\exp\paren{-\abs{\alpha}^2\paren{\ue^{\varepsilon} - 1}} \abs{\langle\alpha|\alpha'\rangle'}^2\ud^2\alpha'\\
  =&\frac{\ue^{\varepsilon} - 1}{\pi^2\cosh r}\int\ud^2\alpha\int_{\abs{\alpha'}\leqslant\gamma}\exp\paren{-\abs{\alpha}^2\paren{\ue^{\varepsilon} - 1}} \abs{\exp\paren{-\abs{\alpha''}^2-\alpha''^2\tanh r}}\ud^2\alpha'\\
  =&\frac{\ue^{\varepsilon} - 1}{\pi^2\cosh r}\int\ud^2\alpha\int_{\abs{\alpha'}\leqslant\gamma}\exp\paren{-\abs{\alpha}^2\paren{\ue^{\varepsilon} - 1}}\exp\paren{-\abs{\alpha''}^2-\Re\paren{\alpha''^2\tanh r}}\ud^2\alpha'
}
Let the real and imaginary part of $\alpha$, $\alpha'$ and $\alpha''$ be $\alpha=\alpha_r+\ui\alpha_i$, $\alpha'=\alpha'_r+\ui\alpha'_i$, $\alpha''=\alpha''_r+\ui\alpha''_i$.
\eqar{
  \alpha''=&\alpha-\alpha'\cosh r+\alpha'^*\sinh r\\
  =&\alpha_r+\ui\alpha_i-\alpha'_r\cosh r+\alpha'_r\sinh r-\ui\alpha'_i\cosh r-\ui\alpha'_i\sinh r\\
  =&\alpha_r-\alpha'_r\cosh r+\alpha'_r\sinh r+\ui\paren{\alpha_i-\alpha'_i\cosh r-\alpha'_i\sinh r}\\
  \alpha''_r=&\alpha_r-\alpha'_r\cosh r\paren{1-\tanh r}\\
  \alpha''_i=&\alpha_i-\alpha'_i\cosh r\paren{1+\tanh r}
}
\eqar{
  p=&\frac{\ue^{\varepsilon} - 1}{\pi^2\cosh r}
  \int\ud^2\alpha\int_{\abs{\alpha'}\leqslant\gamma}\ud^2\alpha'
  \exp\paren{-\paren{\alpha_r^2+\alpha_i^2}\paren{\ue^{\varepsilon} - 1}}\\
  &\exp\paren{-\paren{{\alpha''_r}^2+{\alpha''_i}^2}-\paren{{\alpha''_r}^2-{\alpha''_i}^2}\tanh r}\\
  =&\frac{\ue^{\varepsilon} - 1}{\pi^2\cosh r}\int\ud^2\alpha\int_{\abs{\alpha'}\leqslant\gamma}\ud^2\alpha'\\
  &\exp\paren{-\alpha_r^2\paren{\ue^{\varepsilon} - 1}-{\alpha''_r}^2\paren{1+\tanh r}}
  \exp\paren{-\alpha_i^2\paren{\ue^{\varepsilon} - 1}-{\alpha''_i}^2\paren{1-\tanh r}}
}
\eqar{
  {\alpha''_r}^2=&\paren{\alpha_r-\alpha'_r\cosh r\paren{1-\tanh r}}^2\\
  =&\alpha_r^2+{\alpha'_r}^2\cosh^2r\paren{1-\tanh r}^2-2\alpha_r\alpha'_r\cosh r\paren{1-\tanh r}\\
  {\alpha''_r}^2\paren{1+\tanh r}=&\alpha_r^2\paren{1+\tanh r}+{\alpha'_r}^2\cosh^2r\paren{1-\tanh r}\paren{1-\tanh^2 r}-2\alpha_r\alpha'_r\cosh r\paren{1-\tanh^2 r}\\
  =&\alpha_r^2\paren{1+\tanh r}+{\alpha'_r}^2\paren{1-\tanh r}-\frac{2\alpha_r\alpha'_r}{\cosh r}
}
\eqar{
  {\alpha''_i}^2=&\paren{\alpha_i-\alpha'_i\cosh r\paren{1+\tanh r}}^2\\
  =&\alpha_i^2+{\alpha'_i}^2\cosh^2r\paren{1+\tanh r}^2-2\alpha_i\alpha'_i\cosh r\paren{1+\tanh r}\\
  {\alpha''_i}^2\paren{1-\tanh r}=&\alpha_i^2\paren{1-\tanh r}+{\alpha'_i}^2\cosh^2r\paren{1+\tanh r}\paren{1-\tanh^2 r}-2\alpha_i\alpha'_i\cosh r\paren{1-\tanh^2 r}\\
  =&\alpha_i^2\paren{1-\tanh r}+{\alpha'_i}^2\paren{1+\tanh r}-\frac{2\alpha_i\alpha'_i}{\cosh r}
}
\eqar{
  p=&\frac{\ue^{\varepsilon} - 1}{\pi^2\cosh r}\int\ud^2\alpha\int_{\abs{\alpha'}\leqslant\gamma}\ud^2\alpha'\\
  &\exp\paren{-\alpha_r^2\paren{\ue^{\varepsilon} + \tanh r}
    +\frac{2\alpha_r\alpha'_r}{\cosh r}
    -{\alpha'_r}^2\paren{1-\tanh r}
  }\\
  &\exp\paren{-\alpha_i^2\paren{\ue^{\varepsilon} - \tanh r}
    +\frac{2\alpha_i\alpha'_i}{\cosh r}
    -{\alpha'_i}^2\paren{1+\tanh r}
  }\\
  =&\frac{\ue^{\varepsilon} - 1}{\pi^2\cosh r}\sqrt{\frac{\pi}{\ue^{\varepsilon} + \tanh r}}\sqrt{\frac{\pi}{\ue^{\varepsilon} - \tanh r}}\int_{\abs{\alpha'}\leqslant\gamma}\ud^2\alpha'\\
  &\exp\paren{\frac{{\alpha'_r}^2}{\paren{\ue^{\varepsilon} + \tanh r}\cosh^2r} - {\alpha'_r}^2\paren{1 - \tanh r}}\\
  &\exp\paren{\frac{{\alpha'_i}^2}{\paren{\ue^{\varepsilon} - \tanh r}\cosh^2r} - {\alpha'_i}^2\paren{1 + \tanh r}}\\
  =&\frac{\ue^{\varepsilon} - 1}{\pi\cosh r\sqrt{\ue^{2\varepsilon} - \tanh^2 r}}\int_{{\alpha'_i}^2+{\alpha'_r}^2\leqslant\gamma^2}\ud\alpha'_r\ud\alpha'_i\\
  &\exp\paren{
    \frac{\paren{1 - \tanh r}\paren{1 - \ue^\varepsilon}}{\ue^{\varepsilon} + \tanh r}{\alpha'_r}^2
  }
  \exp\paren{
    \frac{\paren{1 + \tanh r}\paren{1 - \ue^\varepsilon}}{\ue^{\varepsilon} - \tanh r}{\alpha'_i}^2
  }\\
  =&\frac{1}{\pi\cosh r\sqrt{\ue^{2\varepsilon} - \tanh^2 r}}\int_{x^2+y^2\leqslant\paren{\ue^{\varepsilon} - 1}\gamma^2}\ud\alpha'_r\ud\alpha'_i\\
  &\exp\paren{
    -\frac{\paren{1 - \tanh r}}{\ue^{\varepsilon} + \tanh r}x^2
  }
  \exp\paren{
    -\frac{\paren{1 + \tanh r}}{\ue^{\varepsilon} - \tanh r}y^2
  }
}

\end{document}
