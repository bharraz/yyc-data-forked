\documentclass[10pt,fleqn]{article}
% \usepackage[journal=rsc]{chemstyle}
% \usepackage{mhchem}
\usepackage{amsmath}
\usepackage{amssymb}
\usepackage{amsfonts}
\usepackage{esint}
\usepackage{bbm}
\usepackage{amscd}
\usepackage{picinpar}
\usepackage{graphicx}
\usepackage{tikz}
\usepackage{tikz-3dplot}
\usepackage{indentfirst}
\usepackage{wrapfig}
\usepackage{units}
\usepackage{textcomp}
\usepackage[utf8x]{inputenc}
% \usepackage{feyn}
\usepackage{xkeyval}
\usepackage{xargs}
\usepackage{verbatim}
\usepackage{pgfplots}
\usepackage{hyperref}
\usetikzlibrary{
  arrows,
  calc,
  decorations.pathmorphing,
  decorations.pathreplacing,
  decorations.markings,
  fadings,
  positioning,
  shapes
}

\DeclareGraphicsRule{*}{mps}{*}{}
\newcommand{\ud}{\mathrm{d}}
\newcommand{\ue}{\mathrm{e}}
\newcommand{\ui}{\mathrm{i}}
\newcommand{\res}{\mathrm{Res}}
\newcommand{\Tr}{\mathrm{Tr}}
\newcommand{\dsum}{\displaystyle\sum}
\newcommand{\dprod}{\displaystyle\prod}
\newcommand{\dlim}{\displaystyle\lim}
\newcommand{\dint}{\displaystyle\int}
\newcommand{\fsno}[1]{{\!\not\!{#1}}}
\newcommand{\eqar}[1]
{
  \begin{align*}
    #1
  \end{align*}
}
\newcommand{\texp}[2]{\ensuremath{{#1}\times10^{#2}}}
\newcommand{\dexp}[2]{\ensuremath{{#1}\cdot10^{#2}}}
\newcommand{\eval}[2]{{\left.{#1}\right|_{#2}}}
\newcommand{\paren}[1]{{\left({#1}\right)}}
\newcommand{\lparen}[1]{{\left({#1}\right.}}
\newcommand{\rparen}[1]{{\left.{#1}\right)}}
\newcommand{\abs}[1]{{\left|{#1}\right|}}
\newcommand{\sqr}[1]{{\left[{#1}\right]}}
\newcommand{\crly}[1]{{\left\{{#1}\right\}}}
\newcommand{\angl}[1]{{\left\langle{#1}\right\rangle}}
\newcommand{\tpdiff}[4][{}]{{\paren{\frac{\partial^{#1} {#2}}{\partial {#3}{}^{#1}}}_{#4}}}
\newcommand{\tpsdiff}[4][{}]{{\paren{\frac{\partial^{#1}}{\partial {#3}{}^{#1}}{#2}}_{#4}}}
\newcommand{\pdiff}[3][{}]{{\frac{\partial^{#1} {#2}}{\partial {#3}{}^{#1}}}}
\newcommand{\diff}[3][{}]{{\frac{\ud^{#1} {#2}}{\ud {#3}{}^{#1}}}}
\newcommand{\psdiff}[3][{}]{{\frac{\partial^{#1}}{\partial {#3}{}^{#1}} {#2}}}
\newcommand{\sdiff}[3][{}]{{\frac{\ud^{#1}}{\ud {#3}{}^{#1}} {#2}}}
\newcommand{\tpddiff}[4][{}]{{\left(\dfrac{\partial^{#1} {#2}}{\partial {#3}{}^{#1}}\right)_{#4}}}
\newcommand{\tpsddiff}[4][{}]{{\paren{\dfrac{\partial^{#1}}{\partial {#3}{}^{#1}}{#2}}_{#4}}}
\newcommand{\pddiff}[3][{}]{{\dfrac{\partial^{#1} {#2}}{\partial {#3}{}^{#1}}}}
\newcommand{\ddiff}[3][{}]{{\dfrac{\ud^{#1} {#2}}{\ud {#3}{}^{#1}}}}
\newcommand{\psddiff}[3][{}]{{\frac{\partial^{#1}}{\partial{}^{#1} {#3}} {#2}}}
\newcommand{\sddiff}[3][{}]{{\frac{\ud^{#1}}{\ud {#3}{}^{#1}} {#2}}}
\usepackage{fancyhdr}
\usepackage{multirow}
\usepackage{fontenc}
% \usepackage{tipa}
\usepackage{ulem}
\usepackage{color}
\usepackage{cancel}
\newcommand{\hcancel}[2][black]{\setbox0=\hbox{#2}%
  \rlap{\raisebox{.45\ht0}{\textcolor{#1}{\rule{\wd0}{1pt}}}}#2}
\pagestyle{fancy}
\setlength{\headheight}{67pt}
\fancyhead{}
\fancyfoot{}
\fancyfoot[C]{\thepage}
\fancyhead[R]{}
\renewcommand{\footruleskip}{0pt}
\renewcommand{\headrulewidth}{0.4pt}
\renewcommand{\footrulewidth}{0pt}

\newcommand\pgfmathsinandcos[3]{%
  \pgfmathsetmacro#1{sin(#3)}%
  \pgfmathsetmacro#2{cos(#3)}%
}
\newcommand\LongitudePlane[3][current plane]{%
  \pgfmathsinandcos\sinEl\cosEl{#2} % elevation
  \pgfmathsinandcos\sint\cost{#3} % azimuth
  \tikzset{#1/.estyle={cm={\cost,\sint*\sinEl,0,\cosEl,(0,0)}}}
}
\newcommand\LatitudePlane[3][current plane]{%
  \pgfmathsinandcos\sinEl\cosEl{#2} % elevation
  \pgfmathsinandcos\sint\cost{#3} % latitude
  \pgfmathsetmacro\yshift{\cosEl*\sint}
  \tikzset{#1/.estyle={cm={\cost,0,0,\cost*\sinEl,(0,\yshift)}}} %
}
\newcommand\DrawLongitudeCircle[2][1]{
  \LongitudePlane{\angEl}{#2}
  \tikzset{current plane/.prefix style={scale=#1}}
  % angle of "visibility"
  \pgfmathsetmacro\angVis{atan(sin(#2)*cos(\angEl)/sin(\angEl))} %
  \draw[current plane] (\angVis:1) arc (\angVis:\angVis+180:1);
  \draw[current plane,dashed] (\angVis-180:1) arc (\angVis-180:\angVis:1);
}
\newcommand\DrawLatitudeCircleArrow[2][1]{
  \LatitudePlane{\angEl}{#2}
  \tikzset{current plane/.prefix style={scale=#1}}
  \pgfmathsetmacro\sinVis{sin(#2)/cos(#2)*sin(\angEl)/cos(\angEl)}
  % angle of "visibility"
  \pgfmathsetmacro\angVis{asin(min(1,max(\sinVis,-1)))}
  \draw[current plane,decoration={markings, mark=at position 0.6 with {\arrow{<}}},postaction={decorate},line width=.6mm] (\angVis:1) arc (\angVis:-\angVis-180:1);
  \draw[current plane,dashed,line width=.6mm] (180-\angVis:1) arc (180-\angVis:\angVis:1);
}
\newcommand\DrawLatitudeCircle[2][1]{
  \LatitudePlane{\angEl}{#2}
  \tikzset{current plane/.prefix style={scale=#1}}
  \pgfmathsetmacro\sinVis{sin(#2)/cos(#2)*sin(\angEl)/cos(\angEl)}
  % angle of "visibility"
  \pgfmathsetmacro\angVis{asin(min(1,max(\sinVis,-1)))}
  \draw[current plane] (\angVis:1) arc (\angVis:-\angVis-180:1);
  \draw[current plane,dashed] (180-\angVis:1) arc (180-\angVis:\angVis:1);
}
\newcommand\coil[1]{
  {\rh * cos(\t * pi r)}, {\apart * (2 * #1 + \t) + \rv * sin(\t * pi r)}
}
\makeatletter
\define@key{DrawFromCenter}{style}[{->}]{
  \tikzset{DrawFromCenterPlane/.style={#1}}
}
\define@key{DrawFromCenter}{r}[1]{
  \def\@R{#1}
}
\define@key{DrawFromCenter}{center}[(0, 0)]{
  \def\@Center{#1}
}
\define@key{DrawFromCenter}{theta}[0]{
  \def\@Theta{#1}
}
\define@key{DrawFromCenter}{phi}[0]{
  \def\@Phi{#1}
}
\presetkeys{DrawFromCenter}{style, r, center, theta, phi}{}
\newcommand*\DrawFromCenter[1][]{
  \setkeys{DrawFromCenter}{#1}{
    \pgfmathsinandcos\sint\cost{\@Theta}
    \pgfmathsinandcos\sinp\cosp{\@Phi}
    \pgfmathsinandcos\sinA\cosA{\angEl}
    \pgfmathsetmacro\DX{\@R*\cost*\cosp}
    \pgfmathsetmacro\DY{\@R*(\cost*\sinp*\sinA+\sint*\cosA)}
    \draw[DrawFromCenterPlane] \@Center -- ++(\DX, \DY);
  }
}
\newcommand*\DrawFromCenterText[2][]{
  \setkeys{DrawFromCenter}{#1}{
    \pgfmathsinandcos\sint\cost{\@Theta}
    \pgfmathsinandcos\sinp\cosp{\@Phi}
    \pgfmathsinandcos\sinA\cosA{\angEl}
    \pgfmathsetmacro\DX{\@R*\cost*\cosp}
    \pgfmathsetmacro\DY{\@R*(\cost*\sinp*\sinA+\sint*\cosA)}
    \draw[DrawFromCenterPlane] \@Center -- ++(\DX, \DY) node {#2};
  }
}
\makeatother
\tikzstyle{snakearrow} = [decorate, decoration={pre length=0.2cm,
  post length=0.2cm, snake, amplitude=.4mm,
  segment length=2mm},thick, ->]
%% document-wide tikz options and styles
\tikzset{%
  >=latex, % option for nice arrows
  inner sep=0pt,%
  outer sep=2pt,%
  mark coordinate/.style={inner sep=0pt,outer sep=0pt,minimum size=3pt,
    fill=black,circle}%
}
\addtolength{\hoffset}{-1.3cm}
\addtolength{\voffset}{-2cm}
\addtolength{\textwidth}{3cm}
\addtolength{\textheight}{2.5cm}
\renewcommand{\footskip}{10pt}
\setlength{\headwidth}{\textwidth}
\setlength{\headsep}{20pt}
\setlength{\marginparwidth}{0pt}
\parindent=0pt
\title{Response of dressing rotation to amplitude noise}

\ifpdf
  % Ensure reproducible output
  \pdfinfoomitdate=1
  \pdfsuppressptexinfo=-1
  \pdftrailerid{}
  \hypersetup{
    pdfcreator={},
    pdfproducer={}
  }
\fi

\begin{document}

\maketitle

Dressing Hamiltonian
\eqar{
  H=&\Omega\sigma_x+\delta\sigma_z
}
Large $\Omega$ for the target ion and small $\Omega$ for the crosstalk ion.\\

The rotation should robustly turn the $|0\rangle$ and $|1\rangle$ states into
the eigen state of the dressing Hamiltonian for both small and large $\Omega$.\\

The two eigen states are
\eqar{
  |\psi_\pm\rangle=&\frac1{\sqrt2}\begin{pmatrix}
    \pm\sqrt{1\pm1/\sqrt{1 + (\Omega/\delta)^2}}\\
    \sqrt{1\mp1/\sqrt{1 + (\Omega/\delta)^2}}\\
  \end{pmatrix}
}
The transformation needed is
\eqar{
  U=&\frac1{\sqrt2}\begin{pmatrix}
    \sqrt{1+1/\sqrt{1 + (\Omega/\delta)^2}}&-\sqrt{1-1/\sqrt{1 + (\Omega/\delta)^2}}\\
    \sqrt{1-1/\sqrt{1 + (\Omega/\delta)^2}}&\sqrt{1+1/\sqrt{1 + (\Omega/\delta)^2}}\\
  \end{pmatrix}
}
For small $\Omega$,
\eqar{
  U\approx&\begin{pmatrix}
    1&-\Omega/\delta / 2\\
    \Omega/\delta / 2&1\\
  \end{pmatrix}\\
  =&1-\frac{\ui\Omega}{2\delta}\sigma_y
}
For large $\Omega$, if $\Omega=\Omega_0(1+\varepsilon)$

\eqar{
  &\sqrt{1\pm1/\sqrt{1 + (\Omega/\delta)^2}}\\
  \approx&\sqrt{1\pm1/\sqrt{1 + (\Omega_0/\delta)^2}}-\frac{(\Omega_0/\delta)^2}{2\paren{1 + (\Omega_0/\delta)^2}^{3/2}\sqrt{1\pm1/\sqrt{1 + (\Omega_0/\delta)^2}}}\varepsilon\\
  \approx&\sqrt{1\pm1/\sqrt{1 + (\Omega_0/\delta)^2}}-\frac{(\Omega_0/\delta)\sqrt{1\mp1/\sqrt{1 + (\Omega_0/\delta)^2}}}{2\paren{1 + (\Omega_0/\delta)^2}}\varepsilon
}
\eqar{
  U=&\frac1{\sqrt2}\begin{pmatrix}
    \sqrt{1+1/\sqrt{1 + (\Omega/\delta)^2}}&-\sqrt{1-1/\sqrt{1 + (\Omega/\delta)^2}}\\
    \sqrt{1-1/\sqrt{1 + (\Omega/\delta)^2}}&\sqrt{1+1/\sqrt{1 + (\Omega/\delta)^2}}\\
  \end{pmatrix}\\
  \approx&\frac1{\sqrt2}\begin{pmatrix}
    \sqrt{1+1/\sqrt{1 + (\Omega_0/\delta)^2}}&-\sqrt{1-1/\sqrt{1 + (\Omega_0/\delta)^2}}\\
    \sqrt{1-1/\sqrt{1 + (\Omega_0/\delta)^2}}&\sqrt{1+1/\sqrt{1 + (\Omega_0/\delta)^2}}\\
  \end{pmatrix}\\
  &+\frac{(\Omega_0/\delta)\varepsilon}{2\sqrt2\paren{1 + (\Omega_0/\delta)^2}}\begin{pmatrix}
    \sqrt{1-1/\sqrt{1 + (\Omega_0/\delta)^2}}&-\sqrt{1+1/\sqrt{1 + (\Omega_0/\delta)^2}}\\
    \sqrt{1+1/\sqrt{1 + (\Omega_0/\delta)^2}}&\sqrt{1-1/\sqrt{1 + (\Omega_0/\delta)^2}}\\
  \end{pmatrix}\\
  =&\sqrt{\frac{1+1/\sqrt{1 + (\Omega_0/\delta)^2}}{2}}-
  \sqrt{\frac{1-1/\sqrt{1 + (\Omega_0/\delta)^2}}{2}}\ui\sigma_y\\
  &+\frac{(\Omega_0/\delta)\sqrt{1-1/\sqrt{1 + (\Omega_0/\delta)^2}}}{2\sqrt2\paren{1 + (\Omega_0/\delta)^2}}\varepsilon
  +\frac{(\Omega_0/\delta)\sqrt{1+1/\sqrt{1 + (\Omega_0/\delta)^2}}}{2\sqrt2\paren{1 + (\Omega_0/\delta)^2}}\ui\sigma_y\varepsilon
}

The pulses should consists of rotations with the same Rabi frequency
but may have variable detuning, phase and time. Note that only the $xy$ component
of the unitary matters. The $z$ component and the global phase can be perfectly
cancelled out by using the time reversal as the undress pulse sequence.

\section{Single rotation}
\eqar{
  &\exp\paren{\ui\paren{X\paren{\cos\varphi\sigma_x+\sin\varphi\sigma_y}+Z\sigma_z}}\\
  =&\cos\paren{\sqrt{X^2+Z^2}}+\ui\sin\paren{\sqrt{X^2+Z^2}}\frac{X\cos\varphi}{\sqrt{X^2+Z^2}}\sigma_x\\
  &+\ui\sin\paren{\sqrt{X^2+Z^2}}\frac{X\sin\varphi}{\sqrt{X^2+Z^2}}\sigma_y+\ui\sin\paren{\sqrt{X^2+Z^2}}\frac{Z}{\sqrt{X^2+Z^2}}\sigma_z
}
$xy$ component
\eqar{
  a_1=&\sqrt{\frac{X^2\cos^2\paren{\sqrt{X^2+Z^2}}+Z^2}{X^2+Z^2}}\\
  x_1=&-\ui\frac{\sin\paren{\sqrt{X^2+Z^2}}X}{\sqrt{X^2\cos^2\paren{\sqrt{X^2+Z^2}}+Z^2}}\paren{\cos\paren{\sqrt{X^2+Z^2}}\cos\varphi+\frac{Z\sin\paren{\sqrt{X^2+Z^2}}\sin\varphi}{\sqrt{X^2+Z^2}}}\\
  y_1=&\ui\frac{\sin\paren{\sqrt{X^2+Z^2}}X}{\sqrt{X^2\cos^2\paren{\sqrt{X^2+Z^2}}+Z^2}}\paren{
    \cos\paren{\sqrt{X^2+Z^2}}\sin\varphi
    +\frac{Z\sin\paren{\sqrt{X^2+Z^2}}\cos\varphi}{\sqrt{X^2+Z^2}}
  }
}
\eqar{
  &\cos\paren{\sqrt{X^2+Z^2}}\cos\varphi+\frac{Z\sin\paren{\sqrt{X^2+Z^2}}\sin\varphi}{\sqrt{X^2+Z^2}}=0\\
  &-\sin\paren{\sqrt{X^2+Z^2}}\cos\varphi+\frac{Z\cos\paren{\sqrt{X^2+Z^2}}\sin\varphi}{\sqrt{X^2+Z^2}}-\frac{Z\sin\paren{\sqrt{X^2+Z^2}}\sin\varphi}{X^2+Z^2}=0\\
  &\cos\paren{\sqrt{X^2+Z^2}}\cos\varphi=-\frac{Z\sin\paren{\sqrt{X^2+Z^2}}}{\sqrt{X^2+Z^2}}\sin\varphi\\
  &\sin\paren{\sqrt{X^2+Z^2}}\cos\varphi=\frac{Z\cos\paren{\sqrt{X^2+Z^2}}\sin\varphi}{\sqrt{X^2+Z^2}}-\frac{Z\sin\paren{\sqrt{X^2+Z^2}}\sin\varphi}{X^2+Z^2}\\
  &-\frac{Z\sin^2\paren{\sqrt{X^2+Z^2}}}{\sqrt{X^2+Z^2}}=\frac{Z\cos^2\paren{\sqrt{X^2+Z^2}}}{\sqrt{X^2+Z^2}}-\frac{Z\sin\paren{\sqrt{X^2+Z^2}}\cos\paren{\sqrt{X^2+Z^2}}}{X^2+Z^2}\\
  &\sin\paren{\sqrt{X^2+Z^2}}\cos\paren{\sqrt{X^2+Z^2}}=\sqrt{X^2+Z^2}
}
Can't satisfy.

\subsection{Small power limit}
\eqar{
  &\exp\paren{\ui\paren{X\paren{\cos\varphi\sigma_x+\sin\varphi\sigma_y}+Z\sigma_z}}\\
  \approx&\cos Z+\ui\sin Z\sigma_z+\ui X\frac{\sin Z}{Z}\paren{\cos\varphi\sigma_x+\sin\varphi\sigma_y}
}

\section{Two rotations}
\eqar{
  &\exp\paren{\ui\paren{X_1\paren{\cos\varphi_1\sigma_x+\sin\varphi_1\sigma_y}+Z_1\sigma_z}}\exp\paren{\ui\paren{X_2\paren{\cos\varphi_2\sigma_x+\sin\varphi_2\sigma_y}+Z_2\sigma_z}}\\
  =&\lparen{\cos\paren{\sqrt{X_1^2+Z_1^2}}+\ui\sin\paren{\sqrt{X_1^2+Z_1^2}}\frac{X_1\cos\varphi_1}{\sqrt{X_1^2+Z_1^2}}\sigma_x}\\
  &+\rparen{\ui\sin\paren{\sqrt{X_1^2+Z_1^2}}\frac{X_1\sin\varphi_1}{\sqrt{X_1^2+Z_1^2}}\sigma_y+\ui\sin\paren{\sqrt{X_1^2+Z_1^2}}\frac{Z_1}{\sqrt{X_1^2+Z_1^2}}\sigma_z}\\
  &\lparen{\cos\paren{\sqrt{X_2^2+Z_2^2}}+\ui\sin\paren{\sqrt{X_2^2+Z_2^2}}\frac{X_2\cos\varphi_2}{\sqrt{X_2^2+Z_2^2}}\sigma_x}\\
  &+\rparen{\ui\sin\paren{\sqrt{X_2^2+Z_2^2}}\frac{X_2\sin\varphi_2}{\sqrt{X_2^2+Z_2^2}}\sigma_y+\ui\sin\paren{\sqrt{X_2^2+Z_2^2}}\frac{Z_2}{\sqrt{X_2^2+Z_2^2}}\sigma_z}\\
  =&\cos\paren{\sqrt{X_1^2+Z_1^2}}\cos\paren{\sqrt{X_2^2+Z_2^2}}
  -\frac{\sin\paren{\sqrt{X_1^2+Z_1^2}}\sin\paren{\sqrt{X_2^2+Z_2^2}}Z_1Z_2}{\sqrt{X_1^2+Z_1^2}\sqrt{X_2^2+Z_2^2}}\\
  &
  -\sin\paren{\sqrt{X_1^2+Z_1^2}}\sin\paren{\sqrt{X_2^2+Z_2^2}}\frac{X_1X_2\cos\paren{\varphi_1-\varphi_2}}{\sqrt{X_1^2+Z_1^2}\sqrt{X_2^2+Z_2^2}}
  \\
  &+\ui\sin\paren{\sqrt{X_1^2+Z_1^2}}\frac{X_1\cos\varphi_1}{\sqrt{X_1^2+Z_1^2}}\cos\paren{\sqrt{X_2^2+Z_2^2}}\sigma_x+\ui\cos\paren{\sqrt{X_1^2+Z_1^2}}\sin\paren{\sqrt{X_2^2+Z_2^2}}\frac{X_2\cos\varphi_2}{\sqrt{X_2^2+Z_2^2}}\sigma_x\\
  &+\ui\frac{\sin\paren{\sqrt{X_1^2+Z_1^2}}\sin\paren{\sqrt{X_2^2+Z_2^2}}\paren{Z_1X_2\sin\varphi_2-X_1Z_2\sin\varphi_1}}{\sqrt{X_1^2+Z_1^2}\sqrt{X_2^2+Z_2^2}}\sigma_x\\
  &+\ui\sin\paren{\sqrt{X_1^2+Z_1^2}}\frac{X_1\sin\varphi_1}{\sqrt{X_1^2+Z_1^2}}\cos\paren{\sqrt{X_2^2+Z_2^2}}\sigma_y-\ui\sin\paren{\sqrt{X_1^2+Z_1^2}}\frac{Z_1}{\sqrt{X_1^2+Z_1^2}}\sin\paren{\sqrt{X_2^2+Z_2^2}}\frac{X_2\cos\varphi_2}{\sqrt{X_2^2+Z_2^2}}\sigma_y\\
  &+\ui\cos\paren{\sqrt{X_1^2+Z_1^2}}\sin\paren{\sqrt{X_2^2+Z_2^2}}\frac{X_2\sin\varphi_2}{\sqrt{X_2^2+Z_2^2}}\sigma_y+\ui\sin\paren{\sqrt{X_1^2+Z_1^2}}\frac{X_1\cos\varphi_1}{\sqrt{X_1^2+Z_1^2}}\sin\paren{\sqrt{X_2^2+Z_2^2}}\frac{Z_2}{\sqrt{X_2^2+Z_2^2}}\sigma_y\\
  &+\ui\sin\paren{\sqrt{X_1^2+Z_1^2}}\frac{Z_1}{\sqrt{X_1^2+Z_1^2}}\cos\paren{\sqrt{X_2^2+Z_2^2}}\sigma_z+\ui\sin\paren{\sqrt{X_1^2+Z_1^2}}\frac{X_1\sin\varphi_1}{\sqrt{X_1^2+Z_1^2}}\sin\paren{\sqrt{X_2^2+Z_2^2}}\frac{X_2\cos\varphi_2}{\sqrt{X_2^2+Z_2^2}}\sigma_z\\
  &-\ui\sin\paren{\sqrt{X_1^2+Z_1^2}}\frac{X_1\cos\varphi_1}{\sqrt{X_1^2+Z_1^2}}\sin\paren{\sqrt{X_2^2+Z_2^2}}\frac{X_2\sin\varphi_2}{\sqrt{X_2^2+Z_2^2}}\sigma_z+\ui\cos\paren{\sqrt{X_1^2+Z_1^2}}\sin\paren{\sqrt{X_2^2+Z_2^2}}\frac{Z_2}{\sqrt{X_2^2+Z_2^2}}\sigma_z
  \\
}

\subsection{Small power limit}
\eqar{
  &\exp\paren{\ui\paren{X_1\paren{\cos\varphi_1\sigma_x+\sin\varphi_1\sigma_y}+Z_1\sigma_z}}\exp\paren{\ui\paren{X_2\paren{\cos\varphi_2\sigma_x+\sin\varphi_2\sigma_y}+Z_2\sigma_z}}\\
  \approx&\paren{\cos Z_1+\ui\sin Z_1\sigma_z+\ui X_1\frac{\sin Z_1}{Z_1}\paren{\cos\varphi_1\sigma_x+\sin\varphi_1\sigma_y}}\\
  &\paren{\cos Z_2+\ui\sin Z_2\sigma_z+\ui X_2\frac{\sin Z_2}{Z_2}\paren{\cos\varphi_2\sigma_x+\sin\varphi_2\sigma_y}}\\
  \approx&
  \cos\paren{Z_1+Z_2}+\ui\sin\paren{Z_1+Z_2}\sigma_z
  \\
  &
  +\ui X_2\frac{\sin Z_2}{Z_2}\paren{
    \cos\paren{\varphi_2-Z_1}\sigma_x
    +\sin\paren{\varphi_2-Z_1}\sigma_y
  }
  \\
  &+\ui X_1\frac{\sin Z_1}{Z_1}
  \paren{\cos\paren{\varphi_1+Z_2}\sigma_x+\sin\paren{\varphi_1+Z_2}\sigma_y}
  \\
}
\eqar{
  &\exp\paren{\ui\paren{X_1\paren{\cos\varphi_1\sigma_x+\sin\varphi_1\sigma_y}+Z_1\sigma_z}}\exp\paren{\ui\paren{X_2\paren{\cos\varphi_2\sigma_x+\sin\varphi_2\sigma_y}+Z_2\sigma_z}}\\
  &\exp\paren{\ui\paren{X_3\paren{\cos\varphi_3\sigma_x+\sin\varphi_3\sigma_y}+Z_3\sigma_z}}\\
  \approx&\paren{\cos Z_1+\ui\sin Z_1\sigma_z+\ui X_1\frac{\sin Z_1}{Z_1}\paren{\cos\varphi_1\sigma_x+\sin\varphi_1\sigma_y}}\\
  &\paren{\cos Z_2+\ui\sin Z_2\sigma_z+\ui X_2\frac{\sin Z_2}{Z_2}\paren{\cos\varphi_2\sigma_x+\sin\varphi_2\sigma_y}}\\
  &\paren{\cos Z_3+\ui\sin Z_3\sigma_z+\ui X_3\frac{\sin Z_3}{Z_3}\paren{\cos\varphi_3\sigma_x+\sin\varphi_3\sigma_y}}\\
  \approx&
  \paren{\cos\paren{Z_1+Z_2}+\ui\sin\paren{Z_1+Z_2}\sigma_z}\paren{\cos Z_3+\ui\sin Z_3\sigma_z}\\
  &+\ui X_3\frac{\sin Z_3}{Z_3}\paren{\cos\paren{Z_1+Z_2}+\ui\sin\paren{Z_1+Z_2}\sigma_z}\paren{\cos\varphi_3\sigma_x+\sin\varphi_3\sigma_y}\\
  &
  +\ui X_2\frac{\sin Z_2}{Z_2}\paren{
    \cos\paren{\varphi_2-Z_1}\sigma_x
    +\sin\paren{\varphi_2-Z_1}\sigma_y
  }\paren{\cos Z_3+\ui\sin Z_3\sigma_z}
  \\
  &+\ui X_1\frac{\sin Z_1}{Z_1}
  \paren{\cos\paren{\varphi_1+Z_2}\sigma_x+\sin\paren{\varphi_1+Z_2}\sigma_y}\paren{\cos Z_3+\ui\sin Z_3\sigma_z}
  \\
  \approx&\cos\paren{Z_1+Z_2+Z_3}+\ui\sin\paren{Z_1+Z_2+Z_3}\sigma_z\\
  &+\ui X_3\frac{\sin Z_3}{Z_3}\paren{\cos\paren{Z_1+Z_2}+\ui\sin\paren{Z_1+Z_2}\sigma_z}\paren{\cos\varphi_3\sigma_x+\sin\varphi_3\sigma_y}\\
  &
  +\ui X_2\frac{\sin Z_2}{Z_2}\paren{
    \cos\paren{\varphi_2-Z_1}\sigma_x
    +\sin\paren{\varphi_2-Z_1}\sigma_y
  }\paren{\cos Z_3+\ui\sin Z_3\sigma_z}
  \\
  &+\ui X_1\frac{\sin Z_1}{Z_1}
  \paren{\cos\paren{\varphi_1+Z_2}\sigma_x+\sin\paren{\varphi_1+Z_2}\sigma_y}\paren{\cos Z_3+\ui\sin Z_3\sigma_z}
  \\
}

\end{document}
