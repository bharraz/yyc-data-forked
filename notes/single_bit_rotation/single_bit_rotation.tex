\documentclass[10pt,fleqn]{article}
% \usepackage[journal=rsc]{chemstyle}
% \usepackage{mhchem}
\usepackage{amsmath}
\usepackage{amssymb}
\usepackage{amsfonts}
\usepackage{esint}
\usepackage{bbm}
\usepackage{amscd}
\usepackage{picinpar}
\usepackage{graphicx}
\usepackage{tikz}
\usepackage{tikz-3dplot}
\usepackage{indentfirst}
\usepackage{wrapfig}
\usepackage{units}
\usepackage{textcomp}
\usepackage[utf8x]{inputenc}
% \usepackage{feyn}
\usepackage{feynmp}
\usepackage{xkeyval}
\usepackage{xargs}
\usepackage{verbatim}
\usepackage{pgfplots}
\usepackage{hyperref}
\usetikzlibrary{
  arrows,
  calc,
  decorations.pathmorphing,
  decorations.pathreplacing,
  decorations.markings,
  fadings,
  positioning,
  shapes
}

\DeclareGraphicsRule{*}{mps}{*}{}
\newcommand{\ud}{\mathrm{d}}
\newcommand{\ue}{\mathrm{e}}
\newcommand{\ui}{\mathrm{i}}
\newcommand{\res}{\mathrm{Res}}
\newcommand{\Tr}{\mathrm{Tr}}
\newcommand{\dsum}{\displaystyle\sum}
\newcommand{\dprod}{\displaystyle\prod}
\newcommand{\dlim}{\displaystyle\lim}
\newcommand{\dint}{\displaystyle\int}
\newcommand{\fsno}[1]{{\!\not\!{#1}}}
\newcommand{\eqar}[1]
{
  \begin{align*}
    #1
  \end{align*}
}
\newcommand{\texp}[2]{\ensuremath{{#1}\times10^{#2}}}
\newcommand{\dexp}[2]{\ensuremath{{#1}\cdot10^{#2}}}
\newcommand{\eval}[2]{{\left.{#1}\right|_{#2}}}
\newcommand{\paren}[1]{{\left({#1}\right)}}
\newcommand{\lparen}[1]{{\left({#1}\right.}}
\newcommand{\rparen}[1]{{\left.{#1}\right)}}
\newcommand{\abs}[1]{{\left|{#1}\right|}}
\newcommand{\sqr}[1]{{\left[{#1}\right]}}
\newcommand{\crly}[1]{{\left\{{#1}\right\}}}
\newcommand{\angl}[1]{{\left\langle{#1}\right\rangle}}
\newcommand{\tpdiff}[4][{}]{{\paren{\frac{\partial^{#1} {#2}}{\partial {#3}{}^{#1}}}_{#4}}}
\newcommand{\tpsdiff}[4][{}]{{\paren{\frac{\partial^{#1}}{\partial {#3}{}^{#1}}{#2}}_{#4}}}
\newcommand{\pdiff}[3][{}]{{\frac{\partial^{#1} {#2}}{\partial {#3}{}^{#1}}}}
\newcommand{\diff}[3][{}]{{\frac{\ud^{#1} {#2}}{\ud {#3}{}^{#1}}}}
\newcommand{\psdiff}[3][{}]{{\frac{\partial^{#1}}{\partial {#3}{}^{#1}} {#2}}}
\newcommand{\sdiff}[3][{}]{{\frac{\ud^{#1}}{\ud {#3}{}^{#1}} {#2}}}
\newcommand{\tpddiff}[4][{}]{{\left(\dfrac{\partial^{#1} {#2}}{\partial {#3}{}^{#1}}\right)_{#4}}}
\newcommand{\tpsddiff}[4][{}]{{\paren{\dfrac{\partial^{#1}}{\partial {#3}{}^{#1}}{#2}}_{#4}}}
\newcommand{\pddiff}[3][{}]{{\dfrac{\partial^{#1} {#2}}{\partial {#3}{}^{#1}}}}
\newcommand{\ddiff}[3][{}]{{\dfrac{\ud^{#1} {#2}}{\ud {#3}{}^{#1}}}}
\newcommand{\psddiff}[3][{}]{{\frac{\partial^{#1}}{\partial{}^{#1} {#3}} {#2}}}
\newcommand{\sddiff}[3][{}]{{\frac{\ud^{#1}}{\ud {#3}{}^{#1}} {#2}}}
\usepackage{fancyhdr}
\usepackage{multirow}
\usepackage{fontenc}
% \usepackage{tipa}
\usepackage{ulem}
\usepackage{color}
\usepackage{cancel}
\newcommand{\hcancel}[2][black]{\setbox0=\hbox{#2}%
  \rlap{\raisebox{.45\ht0}{\textcolor{#1}{\rule{\wd0}{1pt}}}}#2}
\pagestyle{fancy}
\setlength{\headheight}{67pt}
\fancyhead{}
\fancyfoot{}
\fancyfoot[C]{\thepage}
\fancyhead[R]{}
\renewcommand{\footruleskip}{0pt}
\renewcommand{\headrulewidth}{0.4pt}
\renewcommand{\footrulewidth}{0pt}

\newcommand\pgfmathsinandcos[3]{%
  \pgfmathsetmacro#1{sin(#3)}%
  \pgfmathsetmacro#2{cos(#3)}%
}
\newcommand\LongitudePlane[3][current plane]{%
  \pgfmathsinandcos\sinEl\cosEl{#2} % elevation
  \pgfmathsinandcos\sint\cost{#3} % azimuth
  \tikzset{#1/.estyle={cm={\cost,\sint*\sinEl,0,\cosEl,(0,0)}}}
}
\newcommand\LatitudePlane[3][current plane]{%
  \pgfmathsinandcos\sinEl\cosEl{#2} % elevation
  \pgfmathsinandcos\sint\cost{#3} % latitude
  \pgfmathsetmacro\yshift{\cosEl*\sint}
  \tikzset{#1/.estyle={cm={\cost,0,0,\cost*\sinEl,(0,\yshift)}}} %
}
\newcommand\DrawLongitudeCircle[2][1]{
  \LongitudePlane{\angEl}{#2}
  \tikzset{current plane/.prefix style={scale=#1}}
  % angle of "visibility"
  \pgfmathsetmacro\angVis{atan(sin(#2)*cos(\angEl)/sin(\angEl))} %
  \draw[current plane] (\angVis:1) arc (\angVis:\angVis+180:1);
  \draw[current plane,dashed] (\angVis-180:1) arc (\angVis-180:\angVis:1);
}
\newcommand\DrawLatitudeCircleArrow[2][1]{
  \LatitudePlane{\angEl}{#2}
  \tikzset{current plane/.prefix style={scale=#1}}
  \pgfmathsetmacro\sinVis{sin(#2)/cos(#2)*sin(\angEl)/cos(\angEl)}
  % angle of "visibility"
  \pgfmathsetmacro\angVis{asin(min(1,max(\sinVis,-1)))}
  \draw[current plane,decoration={markings, mark=at position 0.6 with {\arrow{<}}},postaction={decorate},line width=.6mm] (\angVis:1) arc (\angVis:-\angVis-180:1);
  \draw[current plane,dashed,line width=.6mm] (180-\angVis:1) arc (180-\angVis:\angVis:1);
}
\newcommand\DrawLatitudeCircle[2][1]{
  \LatitudePlane{\angEl}{#2}
  \tikzset{current plane/.prefix style={scale=#1}}
  \pgfmathsetmacro\sinVis{sin(#2)/cos(#2)*sin(\angEl)/cos(\angEl)}
  % angle of "visibility"
  \pgfmathsetmacro\angVis{asin(min(1,max(\sinVis,-1)))}
  \draw[current plane] (\angVis:1) arc (\angVis:-\angVis-180:1);
  \draw[current plane,dashed] (180-\angVis:1) arc (180-\angVis:\angVis:1);
}
\newcommand\coil[1]{
  {\rh * cos(\t * pi r)}, {\apart * (2 * #1 + \t) + \rv * sin(\t * pi r)}
}
\makeatletter
\define@key{DrawFromCenter}{style}[{->}]{
  \tikzset{DrawFromCenterPlane/.style={#1}}
}
\define@key{DrawFromCenter}{r}[1]{
  \def\@R{#1}
}
\define@key{DrawFromCenter}{center}[(0, 0)]{
  \def\@Center{#1}
}
\define@key{DrawFromCenter}{theta}[0]{
  \def\@Theta{#1}
}
\define@key{DrawFromCenter}{phi}[0]{
  \def\@Phi{#1}
}
\presetkeys{DrawFromCenter}{style, r, center, theta, phi}{}
\newcommand*\DrawFromCenter[1][]{
  \setkeys{DrawFromCenter}{#1}{
    \pgfmathsinandcos\sint\cost{\@Theta}
    \pgfmathsinandcos\sinp\cosp{\@Phi}
    \pgfmathsinandcos\sinA\cosA{\angEl}
    \pgfmathsetmacro\DX{\@R*\cost*\cosp}
    \pgfmathsetmacro\DY{\@R*(\cost*\sinp*\sinA+\sint*\cosA)}
    \draw[DrawFromCenterPlane] \@Center -- ++(\DX, \DY);
  }
}
\newcommand*\DrawFromCenterText[2][]{
  \setkeys{DrawFromCenter}{#1}{
    \pgfmathsinandcos\sint\cost{\@Theta}
    \pgfmathsinandcos\sinp\cosp{\@Phi}
    \pgfmathsinandcos\sinA\cosA{\angEl}
    \pgfmathsetmacro\DX{\@R*\cost*\cosp}
    \pgfmathsetmacro\DY{\@R*(\cost*\sinp*\sinA+\sint*\cosA)}
    \draw[DrawFromCenterPlane] \@Center -- ++(\DX, \DY) node {#2};
  }
}
\makeatother
\tikzstyle{snakearrow} = [decorate, decoration={pre length=0.2cm,
  post length=0.2cm, snake, amplitude=.4mm,
  segment length=2mm},thick, ->]
%% document-wide tikz options and styles
\tikzset{%
  >=latex, % option for nice arrows
  inner sep=0pt,%
  outer sep=2pt,%
  mark coordinate/.style={inner sep=0pt,outer sep=0pt,minimum size=3pt,
    fill=black,circle}%
}
\addtolength{\hoffset}{-1.3cm}
\addtolength{\voffset}{-2cm}
\addtolength{\textwidth}{3cm}
\addtolength{\textheight}{2.5cm}
\renewcommand{\footskip}{10pt}
\setlength{\headwidth}{\textwidth}
\setlength{\headsep}{20pt}
\setlength{\marginparwidth}{0pt}
\parindent=0pt
\title{Response of single bit rotation to amplitude noise}

\ifpdf
  % Ensure reproducible output
  \pdfinfoomitdate=1
  \pdfsuppressptexinfo=-1
  \pdftrailerid{}
  \hypersetup{
    pdfcreator={},
    pdfproducer={}
  }
\fi

\begin{document}

\maketitle

\section{Goal}
Try to derive the pulse sequence to achieve robustness against (DC) amplitude noise
in a brute force/generic way.

\section{Pulse sequence with three rotations}
\begin{enumerate}
\item $\sigma_1\equiv\sigma_x\cos\theta_1+\sigma_y\sin\theta_1$ by angle $\psi_1$
\item $\sigma_2\equiv\sigma_x$ by angle $\psi_2$
\item $\sigma_3\equiv\sigma_x\cos\theta_3+\sigma_y\sin\theta_3$ by angle $\psi_3$
\end{enumerate}

Full rotation.
\eqar{
  U=&\exp\paren{\ui\psi_1\sigma_1/2}\exp\paren{\ui\psi_2\sigma_2/2}\exp\paren{\ui\psi_3\sigma_3/2}
}

When $\psi_1$, $\psi_2$, $\psi_3$ are changed by a small fraction $2\varepsilon$ to
$\psi_1\paren{1+2\varepsilon}$, $\psi_2\paren{1+2\varepsilon}$,
$\psi_3\paren{1+2\varepsilon}$ respectively.\\

The derivative of $U$ w.r.t. $\varepsilon$ is.
\eqar{
  \pdiff{U}{\varepsilon}=&\ui\psi_1\exp\paren{\ui\psi_1\sigma_1/2}\sigma_1\exp\paren{\ui\psi_2\sigma_2/2}\exp\paren{\ui\psi_3\sigma_3/2}+\\
  &\ui\psi_2\exp\paren{\ui\psi_1\sigma_1/2}\sigma_2\exp\paren{\ui\psi_2\sigma_2/2}\exp\paren{\ui\psi_3\sigma_3/2}+\\
  &\ui\psi_3\exp\paren{\ui\psi_1\sigma_1/2}\exp\paren{\ui\psi_2\sigma_2/2}\sigma_3\exp\paren{\ui\psi_3\sigma_3/2}
}
Simplifying it by changing coordinates
\eqar{
  &\exp\paren{-\ui\psi_1\sigma_1/2}\pdiff{U}{\varepsilon}\exp\paren{-\ui\psi_2\sigma_2/2}\exp\paren{-\ui\psi_3\sigma_3/2}\\
  =&\ui\psi_1\sigma_1+\ui\psi_2\sigma_2+
  \ui\psi_3\exp\paren{\ui\psi_2\sigma_2/2}\sigma_3\exp\paren{-\ui\psi_2\sigma_1/2}\\
  =&\ui\psi_1\sigma_1+\ui\psi_2\sigma_2+
  \ui\psi_3\paren{\sigma_3\cos\psi_2-\sigma_z\sin\theta_3\sin\psi_2+\sigma_2\cos\theta_3(1-\cos\psi_2)}
}
For this to be $0$, the coefficient for $\sigma_z$ has to be $0$.
This means $\psi_3=0$ (two rotation only), $\sin\theta_3=0$
(third rotation on the same axis as the second one, so effectivel no third rotation),
or $\sin\psi_2=0$. For a true three rotation sequence,
the only option is $\sin\psi_2=0$,
which leave us with two possibilities $\cos\psi_2=\pm1$.\\

For $\cos\psi_2=1$,
\eqar{
  &\exp\paren{-\ui\psi_1\sigma_1/2}\pdiff{U}{\varepsilon}\exp\paren{-\ui\psi_2\sigma_2/2}\exp\paren{-\ui\psi_3\sigma_3/2}\\
  =&\ui\psi_1\sigma_1+\ui\psi_2\sigma_2+\ui\psi_3\sigma_3
}

For $\cos\psi_2=-1$,
\eqar{
  &\exp\paren{-\ui\psi_1\sigma_1/2}\pdiff{U}{\varepsilon}\exp\paren{-\ui\psi_2\sigma_2/2}\exp\paren{-\ui\psi_3\sigma_3/2}\\
  =&\ui\psi_1\sigma_1+\ui\psi_2\sigma_2+
  \ui\psi_3\paren{-\sigma_3+2\sigma_2\cos\theta_3}\\
  =&\ui\psi_1\sigma_1+\ui\psi_2\sigma_2+
  \ui\psi_3\paren{\sigma_x\cos\theta_3-\sigma_y\sin\theta_3}
}

For both of these cases, the robust condition is for the sum of
the three Pauli vectors to be $0$
(in the first case these are the Pauli vector that corresponds to the rotation
and in the second case the last one is replaced
with its reflection about the $x$ axis).
The robust condition for the first one
is also the same as the cross talk calculation condition which allows
the pulse sequence to cancel cross talk and overrotation error to the first order
at the same time.\\

As for the actual rotation, for the first case, the second pulse is a full rotation
so the equivalent final operation is simply the combination
of the first and the third one. Unless one of them is a $\pi$ or $2\pi$ rotation,
the combined result will have a component that is a $z$ rotation.
This operation cannot be removed by changing the coordinate system
(i.e. by making the second rotation along another axis rather than the $x$ axis).
This could be cancelled out for single-qubit gate by adjusting the phase
but may not be as easy for two-qubit gates.\\

For the second case,
\eqar{
  U=&\exp\paren{\ui\psi_1\sigma_1/2}\sigma_x\exp\paren{\ui\psi_3\sigma_3/2}\\
  =&\paren{\cos\frac{\psi_1}{2}+\ui\sigma_x\cos\theta_1\sin\frac{\psi_1}{2}+\ui\sigma_y\sin\theta_1\sin\frac{\psi_1}{2}}\sigma_x\paren{\cos\frac{\psi_3}{2}+\ui\sigma_x\cos\theta_3\sin\frac{\psi_3}{2}+\ui\sigma_y\sin\theta_3\sin\frac{\psi_3}{2}}\\
  =&\sigma_x\cos\frac{\psi_1}{2}\cos\frac{\psi_3}{2}+\ui\cos\frac{\psi_1}{2}\cos\theta_3\sin\frac{\psi_3}{2}-\sigma_z\cos\frac{\psi_1}{2}\sin\theta_3\sin\frac{\psi_3}{2}\\
  &+\ui\cos\theta_1\sin\frac{\psi_1}{2}\cos\frac{\psi_3}{2}-\sigma_x\cos\theta_1\cos\theta_3\sin\frac{\psi_1}{2}\sin\frac{\psi_3}{2}-\sigma_y\cos\theta_1\sin\theta_3\sin\frac{\psi_1}{2}\sin\frac{\psi_3}{2}\\
  &+\sigma_z\sin\theta_1\sin\frac{\psi_1}{2}\cos\frac{\psi_3}{2}-\sigma_y\sin\theta_1\cos\theta_3\sin\frac{\psi_1}{2}\sin\frac{\psi_3}{2}+\sigma_x\sin\theta_1\sin\theta_3\sin\frac{\psi_1}{2}\sin\frac{\psi_3}{2}\\
  =&\ui\cos\theta_3\cos\frac{\psi_1}{2}\sin\frac{\psi_3}{2}+\ui\cos\theta_1\sin\frac{\psi_1}{2}\cos\frac{\psi_3}{2}\\
  &+\sigma_x\cos\frac{\psi_1}{2}\cos\frac{\psi_3}{2}-\sigma_x\cos\theta_1\cos\theta_3\sin\frac{\psi_1}{2}\sin\frac{\psi_3}{2}+\sigma_x\sin\theta_1\sin\theta_3\sin\frac{\psi_1}{2}\sin\frac{\psi_3}{2}\\
  &-\sigma_y\cos\theta_1\sin\theta_3\sin\frac{\psi_1}{2}\sin\frac{\psi_3}{2}-\sigma_y\sin\theta_1\cos\theta_3\sin\frac{\psi_1}{2}\sin\frac{\psi_3}{2}\\
  &-\sigma_z\sin\theta_3\cos\frac{\psi_1}{2}\sin\frac{\psi_3}{2}+\sigma_z\sin\theta_1\sin\frac{\psi_1}{2}\cos\frac{\psi_3}{2}
}

\clearpage
\appendix
\section{Decompose an arbitray single qubit rotation into an $xy$ rotation
  and a $z$ rotation}

Arbitrary single qubit rotation,
\eqar{
  U=&a_0 + x_0 \sigma_x + y_0 \sigma_y + z_0 \sigma_z
}
$xy$ rotation followed by $z$ rotation
\eqar{
  U=&U_{xy} U_z\\
  =&\paren{a_1 + x_1 \sigma_x + y_1 \sigma_y}\paren{\cos\theta + \ui\sin\theta\sigma_z}\\
  =&a_1\cos\theta + \ui a_1\sin\theta \sigma_z + \cos\theta x_1 \sigma_x + \ui\sin\theta x_1 \sigma_x\sigma_z + \cos\theta y_1 \sigma_y + \ui\sin\theta y_1 \sigma_y\sigma_z\\
  =&a_1\cos\theta + \paren{\cos\theta x_1 - \sin\theta y_1} \sigma_x + \paren{\cos\theta y_1 + \sin\theta x_1} \sigma_y + \ui a_1\sin\theta \sigma_z
}
\eqar{
  a_0=&a_1\cos\theta\\
  x_0=&x_1\cos\theta - y_1\sin\theta\\
  y_0=&y_1\cos\theta + x_1\sin\theta\\
  z_0=&\ui a_1\sin\theta
}
so we have
\eqar{
  a_1=&\sqrt{a_0^2-z_0^2}\\
  \cos\theta=&\frac{a_0}{\sqrt{a_0^2-z_0^2}}\\
  \sin\theta=&-\frac{\ui z_0}{\sqrt{a_0^2-z_0^2}}\\
  x_1=&\frac{a_0x_0-\ui z_0y_0}{\sqrt{a_0^2-z_0^2}}\\
  y_1=&\frac{a_0y_0+\ui z_0x_0}{\sqrt{a_0^2-z_0^2}}
}

For the special case of $a_1=0$ (i.e. $a_0^2=z_0^2$),
we can prove that in this case we must have $a_0=z_0=0$ for $U$ to remain
unitary~(Appendix~\ref{ap:prove-a0z0})
\footnote{A non-unitary $U$ may have a similar decomposition
  though the $z$ part may also not be unitary anymore.}.
In this case, $U$ already contains no $\sigma_z$ term so we can simply
return $U$ and $I$ as the decomposition.

\section{Direct prove that $a_0=z_0=0$ if $a_0^2=z_0^2$}
\label{ap:prove-a0z0}
For $U$ to be unitary, we must have $UU^\dagger=1$
\eqar{
  UU^\dagger=&\paren{a_0 + x_0 \sigma_x + y_0 \sigma_y + z_0 \sigma_z}\paren{a_0^* + x_0^* \sigma_x + y_0^* \sigma_y + z_0^* \sigma_z}\\
  =&a_0a_0^* + a_0x_0^* \sigma_x + a_0y_0^* \sigma_y + a_0z_0^* \sigma_z + a_0^*x_0 \sigma_x + x_0^*x_0 \sigma_x \sigma_x + y_0^*x_0 \sigma_x \sigma_y + z_0^*x_0 \sigma_x \sigma_z +\\
  &a_0^*y_0 \sigma_y + x_0^* y_0 \sigma_y\sigma_x + y_0^* y_0 \sigma_y\sigma_y + z_0^* y_0 \sigma_y\sigma_z + a_0^*z_0 \sigma_z + x_0^* z_0 \sigma_z\sigma_x + y_0^* z_0 \sigma_z\sigma_y + z_0^* z_0 \sigma_z\sigma_z\\
  =&\abs{a_0}^2 + \abs{x_0}^2 + \abs{y_0}^2 + \abs{z_0}^2\\
  &+\paren{a_0 x_0^* + a_0^* x_0 + \ui z_0^* y_0 - \ui y_0^* z_0}\sigma_x\\
  &+\paren{a_0 y_0^* + a_0^* y_0 + \ui x_0^* z_0 - \ui z_0^* x_0}\sigma_y\\
  &+\paren{a_0 z_0^* + a_0^* z_0 + \ui y_0^* x_0 - \ui x_0^* y_0}\sigma_z\\
  =&\abs{a_0}^2 + \abs{x_0}^2 + \abs{y_0}^2 + \abs{z_0}^2\\
  &+2\paren{\mathrm{Re}\paren{a_0 x_0^* + \ui y_0 z_0^*}}\sigma_x\\
  &+2\paren{\mathrm{Re}\paren{a_0 y_0^* + \ui z_0 x_0^*}}\sigma_y\\
  &+2\paren{\mathrm{Re}\paren{a_0 z_0^* + \ui x_0 y_0^*}}\sigma_z
}
So we have
\eqar{
  \abs{a_0}^2 + \abs{x_0}^2 + \abs{y_0}^2 + \abs{z_0}^2=&1\\
  \mathrm{Re}\paren{a_0 x_0^* + \ui y_0 z_0^*}=&0\\
  \mathrm{Re}\paren{a_0 y_0^* + \ui z_0 x_0^*}=&0\\
  \mathrm{Re}\paren{a_0 z_0^* + \ui x_0 y_0^*}=&0
}
For $a_0^2=z_0^2$, we have either $a_0=z_0$ or $a_0=-z_0$.
If $a_0=z_0$, the second and the third constaints turns into,
\eqar{
  \mathrm{Re}\paren{a_0 x_0^* - \ui a_0 y_0^*}=&0\\
  \mathrm{Im}\paren{a_0 x_0^* - \ui a_0 y_0^*}=&0
}
or
\eqar{
  a_0 \paren{x_0^* - \ui y_0^*}=&0
}
the last constraint turns into,
\eqar{
  \mathrm{Re}\paren{\abs{a_0}^2 + \ui x_0 y_0^*}=&0
}
If $a_0\neq0$, we have $x_0^* = \ui y_0^*$,
\eqar{
  \mathrm{Re}\paren{\abs{a_0}^2 + \abs{x_0}^2}=&0
}
which requires $a_0=0$ and $x_0=0$ contradicting with $a_0\neq0$.

If $a_0=z_0$, the second and the third constaints turns into,
\eqar{
  \mathrm{Re}\paren{a_0 x_0^* + \ui a_0 y_0^*}=&0\\
  \mathrm{Im}\paren{a_0 x_0^* + \ui a_0 y_0^*}=&0
}
or
\eqar{
  a_0 \paren{x_0^* + \ui y_0^*}=&0
}
the last constraint turns into,
\eqar{
  \mathrm{Re}\paren{-\abs{a_0}^2 + \ui x_0 y_0^*}=&0
}
If $a_0\neq0$, we have $x_0^* = -\ui y_0^*$,
\eqar{
  \mathrm{Re}\paren{-\abs{a_0}^2 - \abs{x_0}^2}=&0
}
which requires $a_0=0$ and $x_0=0$ contradicting with $a_0\neq0$.
So in both cases we must have $a_0=z_0=0$.

\end{document}
