\documentclass[10pt,fleqn]{article}
% \usepackage[journal=rsc]{chemstyle}
% \usepackage{mhchem}
\usepackage{amsmath}
\usepackage{amssymb}
\usepackage{amsfonts}
\usepackage{esint}
\usepackage{bbm}
\usepackage{amscd}
\usepackage{caption}
\usepackage{picinpar}
\usepackage{graphicx}
\usepackage{tikz}
\usepackage{tikz-3dplot}
\usepackage{indentfirst}
\usepackage{wrapfig}
\usepackage{units}
\usepackage{textcomp}
\usepackage[utf8x]{inputenc}
% \usepackage{feyn}
\usepackage{feynmp}
\usepackage{xkeyval}
\usepackage{xargs}
\usepackage{verbatim}
\usepackage{pgfplots}
\usepackage{hyperref}
\usetikzlibrary{
  arrows,
  calc,
  decorations.pathmorphing,
  decorations.pathreplacing,
  decorations.markings,
  fadings,
  positioning,
  shapes
}

\DeclareGraphicsRule{*}{mps}{*}{}
\newcommand{\ud}{\mathrm{d}}
\newcommand{\ue}{\mathrm{e}}
\newcommand{\ui}{\mathrm{i}}
\newcommand{\res}{\mathrm{Res}}
\newcommand{\Tr}{\mathrm{Tr}}
\newcommand{\dsum}{\displaystyle\sum}
\newcommand{\dprod}{\displaystyle\prod}
\newcommand{\dlim}{\displaystyle\lim}
\newcommand{\dint}{\displaystyle\int}
\newcommand{\fsno}[1]{{\!\not\!{#1}}}
\newcommand{\eqar}[1]
{
  \begin{align*}
    #1
  \end{align*}
}
\newcommand{\texp}[2]{\ensuremath{{#1}\times10^{#2}}}
\newcommand{\dexp}[2]{\ensuremath{{#1}\cdot10^{#2}}}
\newcommand{\eval}[2]{{\left.{#1}\right|_{#2}}}
\newcommand{\paren}[1]{{\left({#1}\right)}}
\newcommand{\lparen}[1]{{\left({#1}\right.}}
\newcommand{\rparen}[1]{{\left.{#1}\right)}}
\newcommand{\abs}[1]{{\left|{#1}\right|}}
\newcommand{\sqr}[1]{{\left[{#1}\right]}}
\newcommand{\crly}[1]{{\left\{{#1}\right\}}}
\newcommand{\angl}[1]{{\left\langle{#1}\right\rangle}}
\newcommand{\tpdiff}[4][{}]{{\paren{\frac{\partial^{#1} {#2}}{\partial {#3}{}^{#1}}}_{#4}}}
\newcommand{\tpsdiff}[4][{}]{{\paren{\frac{\partial^{#1}}{\partial {#3}{}^{#1}}{#2}}_{#4}}}
\newcommand{\pdiff}[3][{}]{{\frac{\partial^{#1} {#2}}{\partial {#3}{}^{#1}}}}
\newcommand{\diff}[3][{}]{{\frac{\ud^{#1} {#2}}{\ud {#3}{}^{#1}}}}
\newcommand{\psdiff}[3][{}]{{\frac{\partial^{#1}}{\partial {#3}{}^{#1}} {#2}}}
\newcommand{\sdiff}[3][{}]{{\frac{\ud^{#1}}{\ud {#3}{}^{#1}} {#2}}}
\newcommand{\tpddiff}[4][{}]{{\left(\dfrac{\partial^{#1} {#2}}{\partial {#3}{}^{#1}}\right)_{#4}}}
\newcommand{\tpsddiff}[4][{}]{{\paren{\dfrac{\partial^{#1}}{\partial {#3}{}^{#1}}{#2}}_{#4}}}
\newcommand{\pddiff}[3][{}]{{\dfrac{\partial^{#1} {#2}}{\partial {#3}{}^{#1}}}}
\newcommand{\ddiff}[3][{}]{{\dfrac{\ud^{#1} {#2}}{\ud {#3}{}^{#1}}}}
\newcommand{\psddiff}[3][{}]{{\frac{\partial^{#1}}{\partial{}^{#1} {#3}} {#2}}}
\newcommand{\sddiff}[3][{}]{{\frac{\ud^{#1}}{\ud {#3}{}^{#1}} {#2}}}
\usepackage{fancyhdr}
\usepackage{multirow}
\usepackage{fontenc}
% \usepackage{tipa}
\usepackage{ulem}
\usepackage{color}
\usepackage{cancel}
\newcommand{\hcancel}[2][black]{\setbox0=\hbox{#2}%
  \rlap{\raisebox{.45\ht0}{\textcolor{#1}{\rule{\wd0}{1pt}}}}#2}
\pagestyle{fancy}
\setlength{\headheight}{67pt}
\fancyhead{}
\fancyfoot{}
\fancyfoot[C]{\thepage}
\fancyhead[R]{}
\renewcommand{\footruleskip}{0pt}
\renewcommand{\headrulewidth}{0.4pt}
\renewcommand{\footrulewidth}{0pt}

\newcommand\pgfmathsinandcos[3]{%
  \pgfmathsetmacro#1{sin(#3)}%
  \pgfmathsetmacro#2{cos(#3)}%
}
\newcommand\LongitudePlane[3][current plane]{%
  \pgfmathsinandcos\sinEl\cosEl{#2} % elevation
  \pgfmathsinandcos\sint\cost{#3} % azimuth
  \tikzset{#1/.estyle={cm={\cost,\sint*\sinEl,0,\cosEl,(0,0)}}}
}
\newcommand\LatitudePlane[3][current plane]{%
  \pgfmathsinandcos\sinEl\cosEl{#2} % elevation
  \pgfmathsinandcos\sint\cost{#3} % latitude
  \pgfmathsetmacro\yshift{\cosEl*\sint}
  \tikzset{#1/.estyle={cm={\cost,0,0,\cost*\sinEl,(0,\yshift)}}} %
}
\newcommand\DrawLongitudeCircle[2][1]{
  \LongitudePlane{\angEl}{#2}
  \tikzset{current plane/.prefix style={scale=#1}}
  % angle of "visibility"
  \pgfmathsetmacro\angVis{atan(sin(#2)*cos(\angEl)/sin(\angEl))} %
  \draw[current plane] (\angVis:1) arc (\angVis:\angVis+180:1);
  \draw[current plane,dashed] (\angVis-180:1) arc (\angVis-180:\angVis:1);
}
\newcommand\DrawLatitudeCircleArrow[2][1]{
  \LatitudePlane{\angEl}{#2}
  \tikzset{current plane/.prefix style={scale=#1}}
  \pgfmathsetmacro\sinVis{sin(#2)/cos(#2)*sin(\angEl)/cos(\angEl)}
  % angle of "visibility"
  \pgfmathsetmacro\angVis{asin(min(1,max(\sinVis,-1)))}
  \draw[current plane,decoration={markings, mark=at position 0.6 with {\arrow{<}}},postaction={decorate},line width=.6mm] (\angVis:1) arc (\angVis:-\angVis-180:1);
  \draw[current plane,dashed,line width=.6mm] (180-\angVis:1) arc (180-\angVis:\angVis:1);
}
\newcommand\DrawLatitudeCircle[2][1]{
  \LatitudePlane{\angEl}{#2}
  \tikzset{current plane/.prefix style={scale=#1}}
  \pgfmathsetmacro\sinVis{sin(#2)/cos(#2)*sin(\angEl)/cos(\angEl)}
  % angle of "visibility"
  \pgfmathsetmacro\angVis{asin(min(1,max(\sinVis,-1)))}
  \draw[current plane] (\angVis:1) arc (\angVis:-\angVis-180:1);
  \draw[current plane,dashed] (180-\angVis:1) arc (180-\angVis:\angVis:1);
}
\newcommand\coil[1]{
  {\rh * cos(\t * pi r)}, {\apart * (2 * #1 + \t) + \rv * sin(\t * pi r)}
}
\makeatletter
\define@key{DrawFromCenter}{style}[{->}]{
  \tikzset{DrawFromCenterPlane/.style={#1}}
}
\define@key{DrawFromCenter}{r}[1]{
  \def\@R{#1}
}
\define@key{DrawFromCenter}{center}[(0, 0)]{
  \def\@Center{#1}
}
\define@key{DrawFromCenter}{theta}[0]{
  \def\@Theta{#1}
}
\define@key{DrawFromCenter}{phi}[0]{
  \def\@Phi{#1}
}
\presetkeys{DrawFromCenter}{style, r, center, theta, phi}{}
\newcommand*\DrawFromCenter[1][]{
  \setkeys{DrawFromCenter}{#1}{
    \pgfmathsinandcos\sint\cost{\@Theta}
    \pgfmathsinandcos\sinp\cosp{\@Phi}
    \pgfmathsinandcos\sinA\cosA{\angEl}
    \pgfmathsetmacro\DX{\@R*\cost*\cosp}
    \pgfmathsetmacro\DY{\@R*(\cost*\sinp*\sinA+\sint*\cosA)}
    \draw[DrawFromCenterPlane] \@Center -- ++(\DX, \DY);
  }
}
\newcommand*\DrawFromCenterText[2][]{
  \setkeys{DrawFromCenter}{#1}{
    \pgfmathsinandcos\sint\cost{\@Theta}
    \pgfmathsinandcos\sinp\cosp{\@Phi}
    \pgfmathsinandcos\sinA\cosA{\angEl}
    \pgfmathsetmacro\DX{\@R*\cost*\cosp}
    \pgfmathsetmacro\DY{\@R*(\cost*\sinp*\sinA+\sint*\cosA)}
    \draw[DrawFromCenterPlane] \@Center -- ++(\DX, \DY) node {#2};
  }
}
\makeatother
\tikzstyle{snakearrow} = [decorate, decoration={pre length=0.2cm,
  post length=0.2cm, snake, amplitude=.4mm,
  segment length=2mm},thick, ->]
%% document-wide tikz options and styles
\tikzset{%
  >=latex, % option for nice arrows
  inner sep=0pt,%
  outer sep=2pt,%
  mark coordinate/.style={inner sep=0pt,outer sep=0pt,minimum size=3pt,
    fill=black,circle}%
}
\addtolength{\hoffset}{-1.3cm}
\addtolength{\voffset}{-2cm}
\addtolength{\textwidth}{3cm}
\addtolength{\textheight}{2.5cm}
\renewcommand{\footskip}{10pt}
\setlength{\headwidth}{\textwidth}
\setlength{\headsep}{20pt}
\setlength{\marginparwidth}{0pt}
\parindent=0pt
\title{Notes on two photon scattering of the molecular state}

% \DeclareCaptionLabelFormat{sectionnumber}{#1 \thesection.#2}
% \captionsetup{labelformat=sectionnumber}

\makeatletter
\long\def\my@drawfill#1#2;{%
  \@skipfalse
  \fill[#1,draw=none] #2;
  \@skiptrue
  \draw[#1,fill=none] #2;
}
\newif\if@skip
\newcommand{\skipit}[1]{%
  \if@skip
  \else
  #1
  \fi
}
\newcommand{\drawfill}[1][]{%
  \my@drawfill{#1}}
\makeatother

\begin{document}

\maketitle

\section{Experiment observations}
We have seen experimental evidences of an anomalous scattering / loss in the
lifetime of the ground atomic / molecular states as well as broadening in excited state
line width. We have measured power dependencies for some of the processes that is
consistent with a two photon process from the ground state (or equivalently a one
photon process from the excited state). The measurement we have done includes

\begin{enumerate}
\item Line width and decay time measurement of Raman process near $v=0$ ($1038 \mathrm{nm}$)\\
  The result is consistent with a $300 \mathrm{MHz}$ to $600 \mathrm{MHz}$
  excited state line width.\\
  We did not make a power dependency measurement.\\
  We measured carefully at multiple red detunings, which all shows similar results.\\
  We measured one point not as carefully for blue detuning
  ($18 \mathrm{GHz}$ or $10 \mathrm{GHz}$ detuned depending on the excited state $J$)
  which shows similar result as the corresponding red detuning.
\item Off resonance PA rate (lifetime) in $|F^{\mathrm{Cs}}\!=\!4, m_{F}^{\mathrm{Cs}}\!=\!4; F^{\mathrm{Na}}\!=\!2, m_{F}^{\mathrm{Na}}\!=\!2\rangle$
  near $v=0$ PA line.\\
  The result is consistent with the Raman line width and decay time measurements above
  which gives a wide excited state line width.
\item Excited state line width for $v=0$.\\
  The line width is narrow ($\leqslant 50 \mathrm{MHz}$)
  when the tweezer is near $976 \mathrm{nm}$ but is wide ($\geqslant 500 \textrm{MHz}$)
  when the tweezer is near $1038 \mathrm{nm}$ red detuned from the $v=0$ resonance
  by a few hundred GHz.\\
  We did not make a power dependency measurement.
\item Line width of a ``mysterious'' line\\
  Tweezer $100 \mathrm{GHz}$ red detuned.\\
  Line width $\approx 100 \mathrm{MHz}$
  (determined to be narrow at the time but could have some broadening)\\
  We did not make a power dependency measurement.
\item $|F^{\mathrm{Cs}}\!=\!3, m_{F}^{\mathrm{Cs}}\!=\!3; F^{\mathrm{Na}}\!=\!2, m_{F}^{\mathrm{Na}}\!=\!2\rangle$ lifetime
  in $307366 \mathrm{GHz}$ tweezer.\\
  We measured a shorter than expected lifetime.\\
  The dependency of the loss rate to tweezer power is consistent with the prediction of
  a two photo process from the tweezer. In another word, it is consistent with
  a excited state broadened proportional to the tweezer power.
\item $|F^{\mathrm{Cs}}\!=\!3, m_{F}^{\mathrm{Cs}}\!=\!3; F^{\mathrm{Na}}\!=\!2, m_{F}^{\mathrm{Na}}\!=\!2\rangle$ lifetime
  in $306638 \mathrm{GHz}$ tweezer.\\
  We measured a shorter than expected lifetime.\\
  The power dependency may be consistent with a single photon process.
\item $v=12$ line width.\\
  Narrow ($25 \mathrm{MHz}$) for tweezer at around $307366 \mathrm{GHz}$
  (blue detuned $\approx 900\mathrm{GHz}$). (No tweezer power dependency taken.)\\
  Narrow ($20 \mathrm{MHz}$) for tweezer at $306612 \mathrm{GHz}$
  (blue detuned $\approx 130\mathrm{GHz}$). (No tweezer power dependency taken.)
  We have also indirectly observed a Raman coupling to a deeper molecular bound state
  between the tweezer and the PA probe beam.\\
  Broad ($250 \mathrm{MHz}$) for tweezer at $306344.5 \mathrm{GHz}$
  (red detuned $\approx 150\mathrm{GHz}$). Linearly depending on tweezer power.\\
\item $v=14$ line width.\\
  Broad ($200 \mathrm{MHz}$) for tweezer at around $307366 \mathrm{GHz}$
  (red detuned $\approx 1900\mathrm{GHz}$). Linearly depending on tweezer power.\\
\item Raman blue detuned from $v=12$.\\
  Tweezer at around $306614 \mathrm{GHz}$.\\
  No Rabi flopping observed.
\end{enumerate}

\section{Preliminary explanations for some of the results}
The power dependencies we measured makes it clear that we very likely
have some two photon (from ground state, one photon from excited state) process involved.
We originally suspected a two photon process to a higher excited state but Olivier also
suggested potentially a coupling to the ground state (unbound) motional continuum.
This initially seems improbable due to the small matrix element with the ground state.
However, I later learned that due to the mismatch in the excited state and ground state
electronic potential, the excited state only have about $20\%$ projection onto the ground
bound state. In another word, even though the coupling to a particular ground trapped
state is small, the total coupling to the ground state continuum is actually significant.\\

In terms of a two photon coupling (from ground state) to higher excited state,
it is harder to calculate but it is easy to believe the existence of highly lossy
state either from radiative or non-radiative (pre-dissociation) process.\\

In terms of a two photon coupling (from ground state) back to the ground electronic state,
it should be easier to calculate (see below) and can more easily explain the apparent
sensitivity to the detuning (red vs blue) of the excited state line width measurements.
We have also directly measured a coupling to the ground bound state when we switched
to blue detuning.

\section{Caveat in the interpretation}

There are two kinds of measurements above
\begin{enumerate}
\item Ground state scattering/loss rate\\
  This includes Raman decay time / line width and atomic state 2-body lifetime.
\item Excited state line width.
\end{enumerate}

The first kind of measurement mainly involve a single laser frequency
or two that are not very far apart ($\approx 300 \mathrm{MHz}$).
This single frequency is also detuned from any known excited state frequency
by tens to hundreds of GHz (up to around $1 \mathrm{THz}$).
This is more directly related to the loss we care about.
(FWIW, the loss during Raman process is one of this type of measurements).\\

The second kind of measurement usually involve two laser frequencies.
\footnote{We've done measurements using the tweezer as PA without additional frequencies but
  those data are generally assumed to be power broadened since we cannot turn down the
  tweezer power too much during the measurement. Therefore, the tweezer PA measurement
  don't usually offer us much information on the excited state line width.}
One of the two frequencies is detuned by few GHz to a THz and the other one is scanned
around the resonance by up to a few GHz.\\

Since we usually express the difference in the ground state scattering rate in terms of
a difference in the excited state line width (lifetime), it is tempting to compare this
line width to the excited state line width direction measured at the same tweezer
frequency. This is the main reason why we were measuring Raman transition using
tweezer around $306614 \mathrm{GHz}$ after observing narrow line width for the nearby
excited state.\\

However, the two are not directly comparable. When we use the directly measured
excited state line width ($\Gamma_e$) to calculate the ground state scattering rate ($\gamma$)
for a certain large detuning ($\Delta$) as,

\[ \gamma=\Gamma\frac{\Omega_1\Omega_2}{\Delta^2} \]

(where $\Omega_1$ and $\Omega_2$ are the single photon Rabi frequencies), we have
assumed that the loss of the excited state
(i.e. the continuum the excited state is coupled to in order to decay)
does not depend strongly on the final energy of the two photon process.
Depending on the structure of the continuum we couple to via the two photon process,
this may be a bad assumption.\\

This has subtly different implications for two photon couple up or two photon couple down.

\subsection{Two photon coupling up}
\begin{wrapfigure}{R}{9cm}
  \begin{center}
    \begin{tikzpicture}
      \draw[->, line width=2] (0, -0.5) -- (0, 5.5);
      \draw[line width=1] (0.5, 0) -- (2, 0) node[right] {$|g\rangle$};
      \draw[line width=1] (0.5, 2.3) -- (2, 2.3) node[right] {$|e\rangle$};
      \fill[opacity=0.2] (0.5, 3.8) rectangle (2, 5.4);
      \drawfill[domain=-0.8:0.8,smooth,variable=\y,red,line width=1,
      fill opacity=0.2]
      plot ({2.8 + 0.8 * exp(-100*pow(\y-0.5, 2)) +
        0.5 * exp(-40*pow(\y-0.2, 2)) +
        0.7 * exp(-70*pow(\y+0.1, 2)) +
        0.9 * exp(-50*pow(\y+0.6, 2))},{\y + 4.6})
      \skipit{--} (2.7, 4.6 + 0.8) \skipit{--} (2.7, 4.6 - 0.8) \skipit{-- cycle};
      \node[below] at (3.4, 4.6 - 0.9) {\small{Loss rate}};

      \draw[->, line width=1,blue] (1.25, 0) -- node[right] {$\Omega_1$} (1.25, 2);
      \draw[->, line width=1,green!80!black] (1.25, 2) -- node[right] {$\Omega_2$} (1.25, 4.0);
      \draw[dashed] (0.5, 4.0) -- (3.72, 4.0);
      \node at (3, 0.2) {(A)};
      \begin{scope}[shift={(4, 0)}]
        \draw[line width=1] (0.5, 0) -- (2, 0) node[right] {$|g\rangle$};
        \draw[line width=1] (0.5, 2.3) -- (2, 2.3) node[right] {$|e\rangle$};
        \fill[opacity=0.2] (0.5, 3.8) rectangle (2, 5.4);
        \drawfill[domain=-0.8:0.8,smooth,variable=\y,red,line width=1,
        fill opacity=0.2]
        plot ({2.8 + 0.8 * exp(-100*pow(\y-0.5, 2)) +
          0.5 * exp(-40*pow(\y-0.2, 2)) +
          0.7 * exp(-70*pow(\y+0.1, 2)) +
          0.9 * exp(-50*pow(\y+0.6, 2))},{\y + 4.6})
        \skipit{--} (2.7, 4.6 + 0.8) \skipit{--} (2.7, 4.6 - 0.8) \skipit{-- cycle};
        \node[below] at (3.4, 4.6 - 0.9) {\small{Loss rate}};

        \draw[->, line width=1,green!80!black] (1.25, 2.3) -- node[right] {$\Omega_2$} (1.25, 4.3);
        \draw[dashed] (0.5, 4.3) -- (2.85, 4.3);
        \node at (3, 0.2) {(B)};
      \end{scope}
    \end{tikzpicture}
  \end{center}
  \caption{Two photon coupling to a (lossy) continuum. (A) Off resonance measurement,
    (B) On resonance measurement.}
  \label{two-up-1}
\end{wrapfigure}

See figure \ref{two-up-1}A, the loss mechanism we suspect is a two photon coupling
to a lossy continuum with a frequency dependent loss rate
\footnote{The loss rate depends on the laser power, i.e. $\Omega_1$ and $\Omega_2$. Here
  ``loss rate'' essentially mean the loss rate per power squared, which is proportional to
  the square of the coupling from $|e\rangle$ and the density of state.
  Also note here since the second beam is coupling to a continuum rather than a single state
  in the generic case, $\Omega_2$ is not a single number but a function of frequency
  depending of the selection of normalization of the continuum.}.
Generically, the loss rate is not flat which may give us a frequency dependent
two photon loss rate. One way to see the difference between a detuned measurement
and a on-resonance (with the singe photon excited state $|e\rangle$)
measurement of the two photon decay rate is that the two process does not have the same
two photon frequencies. For the on-resonance measurement using the same laser that couples
$|e\rangle$ to the higher excited state (figure \ref{two-up-1}B),
the loss rate experienced by the two photon resonance can be different.
Since the loss rate can have an arbitrary dependency on frequency, one shouldn't expect
the difference in the loss rate to be included in the detuning factor when calculating
the scattering in the off-resonance case.\\

\begin{wrapfigure}{R}{5cm}
  \begin{center}
    \begin{tikzpicture}
      \draw[->, line width=2] (0, -0.5) -- (0, 5.5);

      \draw[dashed] (-0.05, 0) node[left] {$0$} -- (1.5, 0);
      \draw[dashed] (-0.05, 2.3) node[left] {$\omega_0$} -- (1.5, 2.3);
      \draw[dashed] (-0.05, 4.3) node[left] {$\omega_0 + \omega$} -- (1.5, 4.3);

      \draw[line width=1] (1.5, 0) -- (2, 0) node[right] {$|g\rangle$};
      \draw[line width=1] (1.5, 2.3) -- (2, 2.3) node[right] {$|e_1\rangle$};
      \draw[line width=1] (1.5, 4.3) -- (2, 4.3) node[right] {$|e_2\rangle$};
      \draw[line width=1] (0.2, 0) -- node[below] {$|g'\rangle$} (0.7, 0);
      \draw[snakearrow, line width=1, red] (1.75, 4.3) -- node[below left=0.2cm]
      {\large $\Gamma$} (0.45, 0);

      \draw[->, line width=1,blue] (1.75, 0) -- node[right] {$\omega$, $\Omega_1$} (1.75, 2);
      \draw[->, line width=1,green!80!black] (1.75, 2) --
      node[right] {$\omega$, $\Omega_2$} (1.75, 4);

      \draw[dashed] (0.5, 4) -- (2, 4);
      \draw[<->] (1, 4.3) -- (1, 4);
      \node[below] at (0.8, 4) {\footnotesize $\Delta\!\equiv\!\omega_0\!-\!\omega$};

      \draw[dashed] (1.3, 2) -- (2, 2);
      \draw[<->] (1.32, 2.3) -- (1.32, 2);
      \node[right] at (1.33, 2.13) {\footnotesize $\Delta$};
    \end{tikzpicture}
  \end{center}
  \caption{Four level model.}
  \label{two-up-2}
\end{wrapfigure}

As a more concrete example, we'll consider a single two photon excited state with
a stable single photon excited state.\\

See figure \ref{two-up-2}. The ground state $|g\rangle$ is coupled to $|e_1\rangle$,
$|e_1\rangle$ is coupled to $|e_2\rangle$ and $|e_2\rangle$ can decay to another ground
state $|g'\rangle$ with decay rate $\Gamma$ and a $100\%$ branching ratio.
The energy of the two ground states $|g\rangle$ and $|g'\rangle$ are both $0$,
the energy of $|e_1\rangle$ is $\omega_0$ and the energy of $|e_2\rangle$ is $\omega_0+\omega$.\\

The states are coupled by a single external laser with frequency $\omega$.
The laser drives the $|g\rangle\leftrightarrow|e_1\rangle$ transition
with a Rabi frequency $\Omega_1$ and the $|g\rangle\leftrightarrow|e_1\rangle$ transition
with a Rabi frequency $\Omega_2$. The single photon detuning on the $|e_1\rangle$ and
the two photon detuning on $|e_2\rangle$ are both $-\Delta\equiv\omega-\omega_0$.\\

For simplicity we have ignored the decay of the $|e_1\rangle$ state
and we assume the laser frequency is the same as the $|e_1\rangle$ $|e_2\rangle$
energy splitting. We will also assume that $\Omega_{1,2}\ll\Delta\ll\omega<\omega_0$
as well as $\Gamma\ll\Delta$. Some of the conditions may not be true in the real experiment
but this example should be enough to demonstrate the correct way to calculate
the scattering rate.\\

We'll calculate the scattering rate of the state $|g\rangle$ using a few different methods.
\footnote{All of the calculation below ignores the frequency dependency of the $|g'\rangle$
  continuum $|e_2\rangle$ decays to. All the scattering rate should be scaled by
  a factor of $\paren{\dfrac{2\omega}{\omega+\omega_0}}^3$.}

\subsubsection{Off resonance Raman scattering}

We can treat this as an off-resonance Raman scattering process.
The Raman Rabi rate for the $|g\rangle\rightarrow|e_1\rangle\rightarrow|e_2\rangle$ is

\[\Omega_R=\frac{\Omega_1\Omega_2}{2\Delta}\]

since $\Omega_R\ll\Omega_1\ll\Delta$, the off resonant scattering rate is,

\eqar{
  \gamma\approx&\frac{\Gamma}{4}\frac{\Omega_R^2}{\Delta^2}\\
  =&\frac{\Gamma}{16}\frac{\Omega_1^2\Omega_2^2}{\Delta^4}
}

\subsubsection{Single photon off resonance scattering from coupled excited state}
\begin{wrapfigure}{R}{7.3cm}
  \begin{center}
    \begin{tikzpicture}
      \draw[->, line width=2] (0, -0.5) -- (0, 3.5);

      \draw[dashed] (-0.05, 0) node[left] {$0$} -- (1.5, 0);
      \draw[dashed] (-0.05, 2.15) node[left] {$\omega_0 - \Omega_2/2$} -- (1.5, 2.15);
      \draw[dashed] (-0.05, 2.45) node[left] {$\omega_0 + \Omega_2/2$} -- (1.5, 2.45);

      \draw[line width=1] (1.5, 0) -- (2, 0) node[right] {$|g\rangle$};
      \draw[line width=1] (1.5, 2.15) -- (2, 2.15) node[right] {$|-\rangle$};
      \draw[line width=1] (1.5, 2.45) -- (2, 2.45) node[right] {$|+\rangle$};
      \draw[line width=1] (0.2, 0) -- node[below] {$|g'\rangle$} (0.7, 0);
      \draw[snakearrow, line width=1, red] (1.75, 2.15) -- node[below right]
      {\large $\Gamma/2$} (0.5, 0);
      \draw[snakearrow, line width=1, red] (1.75, 2.45) -- node[left=0.1cm]
      {\large $\Gamma/2$} (0.3, 0);

      \node[above] at (1, 2.6) {(A)};

      \begin{scope}[shift={(2.6, 0)}]
        \draw[line width=1] (1.5, 0) -- (2, 0) node[right] {$|g\rangle$};
        \draw[line width=1] (1.5, 2.15) -- (2, 2.15) node[right] {$|-\rangle$};
        \draw[line width=1] (1.5, 2.45) -- (2, 2.45) node[right] {$|+\rangle$};
        \draw[line width=1] (0.2, 0) -- node[below] {$|g'\rangle$} (0.7, 0);
        \draw[<->, line width=1, blue] (2, 0) -- node[right]
        {$\frac{D_1}{\sqrt2}$} (2, 2.15);
        \draw[<->, line width=1, blue] (1.9, 0) -- node[above left]
        {$\frac{D_1}{\sqrt2}$} (1.9, 2.45);
        \draw[<->, line width=1, red] (0.2, 0) -- node[left=0.1]
        {$\frac{D'}{\sqrt2}$} (1.5, 2.45);
        \draw[<->, line width=1, red] (0.43, 0) -- (1.6, 2.15);
        \node[red, below right] at (0.9, 1.075) {$\frac{-D'}{\sqrt2}$};

        \node[above] at (1, 2.6) {(B)};
      \end{scope}
    \end{tikzpicture}
  \end{center}
  \caption{Coupled excited states. (B) Dipole moments between new states.}
  \label{two-up-coupled}
\end{wrapfigure}

More directly related to the excited state linewidth measurement, we can first couple the
two excited states together with $\Omega_2$.\\

See figure \ref{two-up-coupled}A. The excited states becomes

\[|\pm\rangle=\frac{1}{\sqrt2}\paren{|e_1\rangle\pm|e_2\rangle}\]

with energy $\omega_0\pm\Omega_2/2$ and each with a lifetime of $\dfrac\Gamma2$.
The Rabi frequency between the two excited states and the ground state is
$\Omega_\pm=\dfrac{\Omega_1}{\sqrt2}$.\\
The scattering from the two excited states are
\eqar{
  \gamma_\pm\approx&\frac{\Gamma}{8}\frac{\Omega_\pm^2}{\paren{\Delta\mp\Omega_2/2}^2}\\
  \approx&\frac{\Gamma}{16}\frac{\Omega_1^2}{\Delta^2}
}
and the total scattering rate
\eqar{
  \gamma=&\gamma_-+\gamma_+\\
  \approx&\frac{\Gamma}{8}\frac{\Omega_1^2}{\Delta^2}
}
which clearly doesn't agree with the result we calculated above.

\subsubsection{Single photon off resonance scattering from coupled excited state
  (the correct way)}

The apparent disagreement between the two result above suggests that one of them must be wrong
and it turns out that the second one is incorrect due to the wrong assumption it made.\\

In this case, the issue manifests itself as an incorrect off resonant scattering rate
from the excited state. There is not problem in coupling the excited state first to get
$|\pm\rangle$ but the decay process from the two states are not independent which give rise
to a different scattering rate away from resonance. (when $\Delta\gg\Omega_2$).
Instead, the scattering from the two excited states have to be treated as interfering
Raman processes and their amplitudes needs to be added, which will lead to cancellation
as we'll see soon, before squared to get the scattering rate.\\

The correct way to calculate this requires using the dipole moment between the states
\footnote{And the formula also correctly takes into account the density of state of
  the scattered photon and gives the correct
  $\paren{\dfrac{2\omega}{\omega+\omega_0}}^3$ factor}.
For the purpose of this note, however, it is enough to only calculate the interference term
and scales our previous result correspondingly.\\

Let the $D_1=\langle g|\hat d|e_1\rangle$ and $D_2=\langle g'|\hat d|e_2\rangle$ be
the dipole moment coupling $e_1$ and $e_2$ to the corresponding ground states $g$ and $g'$.
The dipole moments between the new states are shown in figure \ref{two-up-coupled}B
\footnote{The sign of the dipole moment depends on the selection of the phase for
  $|\pm\rangle$ but the relative sign of the product does not depend on the phase selection}.\\

The wrong way to calculate the scattering gives a factor of
\eqar{
  &\paren{\frac{D'}{\sqrt2}\frac{D_1}{\sqrt2}\frac{1}{\Delta+\Omega_2/2}}^2+\paren{\frac{-D'}{\sqrt2}\frac{D_1}{\sqrt2}\frac{1}{\Delta-\Omega_2/2}}^2\\
  \approx&2\paren{\frac{D'}{\sqrt2}\frac{D_1}{\sqrt2}\frac{1}{\Delta}}^2\\
  =&\frac{D'^2D_1^2}{2\Delta^2}
}
The correct way gives,
\eqar{
  &\paren{\frac{D'}{\sqrt2}\frac{D_1}{\sqrt2}\frac{1}{\Delta+\Omega_2/2}+\frac{-D'}{\sqrt2}\frac{D_1}{\sqrt2}\frac{1}{\Delta-\Omega_2/2}}^2\\
  =&\frac{D'^2D_1^2}{4}\paren{\frac{1}{\Delta+\Omega_2/2}-\frac{1}{\Delta-\Omega_2/2}}^2\\
  =&\frac{D'^2D_1^2}{4}\paren{\frac{\Delta-\Omega_2/2-\Delta-\Omega_2/2}{\Delta^2-\Omega_2^2/4}}^2\\
  \approx&\frac{D'^2D_1^2\Omega_2^2}{4\Delta^4}
}
which means that we should scale the incorrect result above by
\[\frac{D'^2D_1^2\Omega_2^2}{4\Delta^4}\frac{2\Delta^2}{D'^2D_1^2}=\frac{\Omega_2^2}{2\Delta^2}\]
and the correct answer is
\eqar{
  \gamma=&\frac{\Gamma}{8}\frac{\Omega_1^2}{\Delta^2}\frac{\Omega_2^2}{2\Delta^2}\\
  =&\frac{\Gamma}{16}\frac{\Omega_1^2\Omega_2^2}{\Delta^4}
}
which is exactly what we got above.

\subsubsection{Summary for two photon coupling down}

In the particular case (two photon excite to a single state) we calculated,
the naive calculation using coupled excited state (i.e. direct result from PA measurement)
gives the wrong result because of the interference from different scattering channels.
More generally, to avoid this problem one need to realize,
\begin{enumerate}
\item The scattering is never a one photon process, one need to calculate the rate
  after considering all photon involved.
\item It is OK to couple some intermediate states first and use a dressed state picture
  to eliminate the number of photon/coupling/intermediate states involved. However,
  the coupled result may not be a single state and the scattering from the resulting
  states may not follow a simple $\Delta^{-2}$ rule.
\end{enumerate}

\subsection{Two photon coupling down}

% Ground state coupling calculation

\end{document}
