\documentclass[10pt,fleqn]{article}
% \usepackage[journal=rsc]{chemstyle}
% \usepackage{mhchem}
\usepackage{amsmath}
\usepackage{amssymb}
\usepackage{amsfonts}
\usepackage{esint}
\usepackage{bbm}
\usepackage{amscd}
\usepackage{picinpar}
\usepackage{graphicx}
\usepackage{tikz}
\usepackage{tikz-3dplot}
\usepackage{indentfirst}
\usepackage{wrapfig}
\usepackage{units}
\usepackage{textcomp}
\usepackage[utf8x]{inputenc}
% \usepackage{feyn}
\usepackage{feynmp}
\usepackage{xkeyval}
\usepackage{xargs}
\usepackage{verbatim}
\usepackage{pgfplots}
\usepackage{hyperref}
\usetikzlibrary{
  arrows,
  calc,
  decorations.pathmorphing,
  decorations.pathreplacing,
  decorations.markings,
  fadings,
  positioning,
  shapes
}

\DeclareGraphicsRule{*}{mps}{*}{}
\newcommand{\ud}{\mathrm{d}}
\newcommand{\ue}{\mathrm{e}}
\newcommand{\ui}{\mathrm{i}}
\newcommand{\res}{\mathrm{Res}}
\newcommand{\Tr}{\mathrm{Tr}}
\newcommand{\dsum}{\displaystyle\sum}
\newcommand{\dprod}{\displaystyle\prod}
\newcommand{\dlim}{\displaystyle\lim}
\newcommand{\dint}{\displaystyle\int}
\newcommand{\fsno}[1]{{\!\not\!{#1}}}
\newcommand{\eqar}[1]
{
  \begin{align*}
    #1
  \end{align*}
}
\newcommand{\texp}[2]{\ensuremath{{#1}\times10^{#2}}}
\newcommand{\dexp}[2]{\ensuremath{{#1}\cdot10^{#2}}}
\newcommand{\eval}[2]{{\left.{#1}\right|_{#2}}}
\newcommand{\paren}[1]{{\left({#1}\right)}}
\newcommand{\lparen}[1]{{\left({#1}\right.}}
\newcommand{\rparen}[1]{{\left.{#1}\right)}}
\newcommand{\abs}[1]{{\left|{#1}\right|}}
\newcommand{\sqr}[1]{{\left[{#1}\right]}}
\newcommand{\crly}[1]{{\left\{{#1}\right\}}}
\newcommand{\angl}[1]{{\left\langle{#1}\right\rangle}}
\newcommand{\tpdiff}[4][{}]{{\paren{\frac{\partial^{#1} {#2}}{\partial {#3}{}^{#1}}}_{#4}}}
\newcommand{\tpsdiff}[4][{}]{{\paren{\frac{\partial^{#1}}{\partial {#3}{}^{#1}}{#2}}_{#4}}}
\newcommand{\pdiff}[3][{}]{{\frac{\partial^{#1} {#2}}{\partial {#3}{}^{#1}}}}
\newcommand{\diff}[3][{}]{{\frac{\ud^{#1} {#2}}{\ud {#3}{}^{#1}}}}
\newcommand{\psdiff}[3][{}]{{\frac{\partial^{#1}}{\partial {#3}{}^{#1}} {#2}}}
\newcommand{\sdiff}[3][{}]{{\frac{\ud^{#1}}{\ud {#3}{}^{#1}} {#2}}}
\newcommand{\tpddiff}[4][{}]{{\left(\dfrac{\partial^{#1} {#2}}{\partial {#3}{}^{#1}}\right)_{#4}}}
\newcommand{\tpsddiff}[4][{}]{{\paren{\dfrac{\partial^{#1}}{\partial {#3}{}^{#1}}{#2}}_{#4}}}
\newcommand{\pddiff}[3][{}]{{\dfrac{\partial^{#1} {#2}}{\partial {#3}{}^{#1}}}}
\newcommand{\ddiff}[3][{}]{{\dfrac{\ud^{#1} {#2}}{\ud {#3}{}^{#1}}}}
\newcommand{\psddiff}[3][{}]{{\frac{\partial^{#1}}{\partial{}^{#1} {#3}} {#2}}}
\newcommand{\sddiff}[3][{}]{{\frac{\ud^{#1}}{\ud {#3}{}^{#1}} {#2}}}
\usepackage{fancyhdr}
\usepackage{multirow}
\usepackage{fontenc}
% \usepackage{tipa}
\usepackage{ulem}
\usepackage{color}
\usepackage{cancel}
\newcommand{\hcancel}[2][black]{\setbox0=\hbox{#2}%
  \rlap{\raisebox{.45\ht0}{\textcolor{#1}{\rule{\wd0}{1pt}}}}#2}
\pagestyle{fancy}
\setlength{\headheight}{67pt}
\fancyhead{}
\fancyfoot{}
\fancyfoot[C]{\thepage}
\fancyhead[R]{}
\renewcommand{\footruleskip}{0pt}
\renewcommand{\headrulewidth}{0.4pt}
\renewcommand{\footrulewidth}{0pt}

\newcommand\pgfmathsinandcos[3]{%
  \pgfmathsetmacro#1{sin(#3)}%
  \pgfmathsetmacro#2{cos(#3)}%
}
\newcommand\LongitudePlane[3][current plane]{%
  \pgfmathsinandcos\sinEl\cosEl{#2} % elevation
  \pgfmathsinandcos\sint\cost{#3} % azimuth
  \tikzset{#1/.estyle={cm={\cost,\sint*\sinEl,0,\cosEl,(0,0)}}}
}
\newcommand\LatitudePlane[3][current plane]{%
  \pgfmathsinandcos\sinEl\cosEl{#2} % elevation
  \pgfmathsinandcos\sint\cost{#3} % latitude
  \pgfmathsetmacro\yshift{\cosEl*\sint}
  \tikzset{#1/.estyle={cm={\cost,0,0,\cost*\sinEl,(0,\yshift)}}} %
}
\newcommand\DrawLongitudeCircle[2][1]{
  \LongitudePlane{\angEl}{#2}
  \tikzset{current plane/.prefix style={scale=#1}}
  % angle of "visibility"
  \pgfmathsetmacro\angVis{atan(sin(#2)*cos(\angEl)/sin(\angEl))} %
  \draw[current plane] (\angVis:1) arc (\angVis:\angVis+180:1);
  \draw[current plane,dashed] (\angVis-180:1) arc (\angVis-180:\angVis:1);
}
\newcommand\DrawLatitudeCircleArrow[2][1]{
  \LatitudePlane{\angEl}{#2}
  \tikzset{current plane/.prefix style={scale=#1}}
  \pgfmathsetmacro\sinVis{sin(#2)/cos(#2)*sin(\angEl)/cos(\angEl)}
  % angle of "visibility"
  \pgfmathsetmacro\angVis{asin(min(1,max(\sinVis,-1)))}
  \draw[current plane,decoration={markings, mark=at position 0.6 with {\arrow{<}}},postaction={decorate},line width=.6mm] (\angVis:1) arc (\angVis:-\angVis-180:1);
  \draw[current plane,dashed,line width=.6mm] (180-\angVis:1) arc (180-\angVis:\angVis:1);
}
\newcommand\DrawLatitudeCircle[2][1]{
  \LatitudePlane{\angEl}{#2}
  \tikzset{current plane/.prefix style={scale=#1}}
  \pgfmathsetmacro\sinVis{sin(#2)/cos(#2)*sin(\angEl)/cos(\angEl)}
  % angle of "visibility"
  \pgfmathsetmacro\angVis{asin(min(1,max(\sinVis,-1)))}
  \draw[current plane] (\angVis:1) arc (\angVis:-\angVis-180:1);
  \draw[current plane,dashed] (180-\angVis:1) arc (180-\angVis:\angVis:1);
}
\newcommand\coil[1]{
  {\rh * cos(\t * pi r)}, {\apart * (2 * #1 + \t) + \rv * sin(\t * pi r)}
}
\makeatletter
\define@key{DrawFromCenter}{style}[{->}]{
  \tikzset{DrawFromCenterPlane/.style={#1}}
}
\define@key{DrawFromCenter}{r}[1]{
  \def\@R{#1}
}
\define@key{DrawFromCenter}{center}[(0, 0)]{
  \def\@Center{#1}
}
\define@key{DrawFromCenter}{theta}[0]{
  \def\@Theta{#1}
}
\define@key{DrawFromCenter}{phi}[0]{
  \def\@Phi{#1}
}
\presetkeys{DrawFromCenter}{style, r, center, theta, phi}{}
\newcommand*\DrawFromCenter[1][]{
  \setkeys{DrawFromCenter}{#1}{
    \pgfmathsinandcos\sint\cost{\@Theta}
    \pgfmathsinandcos\sinp\cosp{\@Phi}
    \pgfmathsinandcos\sinA\cosA{\angEl}
    \pgfmathsetmacro\DX{\@R*\cost*\cosp}
    \pgfmathsetmacro\DY{\@R*(\cost*\sinp*\sinA+\sint*\cosA)}
    \draw[DrawFromCenterPlane] \@Center -- ++(\DX, \DY);
  }
}
\newcommand*\DrawFromCenterText[2][]{
  \setkeys{DrawFromCenter}{#1}{
    \pgfmathsinandcos\sint\cost{\@Theta}
    \pgfmathsinandcos\sinp\cosp{\@Phi}
    \pgfmathsinandcos\sinA\cosA{\angEl}
    \pgfmathsetmacro\DX{\@R*\cost*\cosp}
    \pgfmathsetmacro\DY{\@R*(\cost*\sinp*\sinA+\sint*\cosA)}
    \draw[DrawFromCenterPlane] \@Center -- ++(\DX, \DY) node {#2};
  }
}
\makeatother
\tikzstyle{snakearrow} = [decorate, decoration={pre length=0.2cm,
  post length=0.2cm, snake, amplitude=.4mm,
  segment length=2mm},thick, ->]
%% document-wide tikz options and styles
\tikzset{%
  >=latex, % option for nice arrows
  inner sep=0pt,%
  outer sep=2pt,%
  mark coordinate/.style={inner sep=0pt,outer sep=0pt,minimum size=3pt,
    fill=black,circle}%
}
\addtolength{\hoffset}{-1.3cm}
\addtolength{\voffset}{-2cm}
\addtolength{\textwidth}{3cm}
\addtolength{\textheight}{2.5cm}
\renewcommand{\footskip}{10pt}
\setlength{\headwidth}{\textwidth}
\setlength{\headsep}{20pt}
\setlength{\marginparwidth}{0pt}
\parindent=0pt
\title{Exact formula for quantum jump method in two level system with decay and coupling}
\begin{document}

\maketitle

\section{The problem}
In order to simulate the off-resonance scattering during Raman transition in a sideband cooling,
we need to simulate the coherent drive and incoherent decay at the same time.
However, this process can potentially couple to a very large number of state that needs to be
propagate at the same time so it'll not be very efficient. Naively using quantum jump method
does not help much since it doesn't change the number of states involved.

The problem can be solved by realizing that we only care about coherent drive between two states
(so off-resonance Raman coupling is ignored) and only the scattering process couples it to a
larger number of states. This means that we should be able to propagate Raman coupling
(i.e. coherent part of the quantum jump method) between only two states but includes a larger
number of states when we are doing the quantum jump. Both of these should be easy to do.
Furthermore, since the coherent part of the calculation has only very few states involved,
we should be able to speed up the calculation and precision by using the exact expression instead
of propagating it with a small time step.

\section{Modified Hamiltonian}
In quantum jump method, the coherent part of the propagation is done using
a non-Hermition Hamiltonian (set $\hbar=1$)
\eqar{
  H'=&H-\frac{\ui}{2}\sum_m C^\dagger_mC_m
}
where the $H$ is the original (Hermition) Hamiltonian and $C_m$s are the jump operator.
For our problem, in the energy basis, the matrix elements of the jump operators takes the form of
$C_{ij}=\sqrt\Gamma\delta_{ii_0}\delta_{jj_0}$
(here we drop the index of jump operator $m$ for simplicity),
i.e. they are jumping from one energy eigenstate ($j_0$)
to another energy eigenstate ($i_0$) at rate $\Gamma$.
(We can do this approximation because we ignore coherence between energy eigenstates.)
Therefore $(C^\dagger C)_{ij}=\Gamma\delta_{ij_0}\delta_{jj_0}$ which is a decay term on state $j_0$
independent of the state it is decaying into.
This is why we can ignore the present of other states when we propagate the coherent part
as long as we include the correct total decay rate and only worry about the other state when
we need to make the jump.

Now since we only need to consider two states, the Hamiltonian is just a two-by-two matrix. We
further assume that the drive is on-resonance, so there's no real diagonal term.
\eqar{
  H'=&-\frac{\ui}2\begin{pmatrix}
    \Gamma_1&\Omega\\
    -\Omega&\Gamma_2
  \end{pmatrix}
  \intertext{Define}
  \Gamma_1\equiv&\Gamma+\Delta\\
  \Gamma_2\equiv&\Gamma-\Delta\\
}
(note that the $\Delta$ is \textbf{not} detuning).

\section{Time evolution}

Formally the time evolution is
\eqar{
  &\exp\paren{-\ui H't}\\
  =&\exp\paren{-\frac t2\begin{pmatrix}
      \Gamma+\Delta&\Omega\\
      -\Omega&\Gamma-\Delta
    \end{pmatrix}}
}
Define $\Omega'=\sqrt{\Omega^2-\Delta^2}$
\eqar{
  &\exp\paren{-\ui H't}\\
  =&\frac{\ue^{-\Gamma t/2}}{2\ui\Omega'}\begin{pmatrix}
    \Delta\paren{\ue^{-\frac{\ui\Omega't}{2}}-\ue^{\frac{\ui\Omega't}{2}}}+\ui\Omega'\paren{\ue^{-\frac{\ui\Omega't}{2}}+\ue^{\frac{\ui\Omega't}{2}}}&\Omega\paren{\ue^{-\frac{\ui\Omega't}{2}}-\ue^{\frac{\ui\Omega't}{2}}}\\
    -\Omega\paren{\ue^{-\frac{\ui\Omega't}{2}}-\ue^{\frac{\ui\Omega't}{2}}}&-\Delta\paren{\ue^{-\frac{\ui\Omega't}{2}}-\ue^{\frac{\ui\Omega't}{2}}}+\ui\Omega'\paren{\ue^{-\frac{\ui\Omega't}{2}}+\ue^{\frac{\ui\Omega't}{2}}}
  \end{pmatrix}\\
  =&\frac{\ue^{-\Gamma t/2}}{2\ui\Omega'}\begin{pmatrix}
    -2\ui\Delta\sin\dfrac{\Omega't}{2}+2\ui\Omega'\cos\dfrac{\Omega't}{2}&-2\ui\Omega\sin\dfrac{\Omega't}{2}\\
    2\ui\Omega\sin\dfrac{\Omega't}{2}&2\ui\Delta\sin\dfrac{\Omega't}{2}+2\ui\Omega'\cos\dfrac{\Omega't}{2}
  \end{pmatrix}\\
  =&\frac{\ue^{-\Gamma t/2}}{\Omega'}\begin{pmatrix}
    -\Delta\sin\dfrac{\Omega't}{2}+\Omega'\cos\dfrac{\Omega't}{2}&-\Omega\sin\dfrac{\Omega't}{2}\\
    \Omega\sin\dfrac{\Omega't}{2}&\Delta\sin\dfrac{\Omega't}{2}+\Omega'\cos\dfrac{\Omega't}{2}
  \end{pmatrix}
}
Starting from the atom in state 1, the wave functions are
\eqar{
  \psi=&\frac{\ue^{-\Gamma t/2}}{\Omega'}\begin{pmatrix}
    -\Delta\sin\dfrac{\Omega't}{2}+\Omega'\cos\dfrac{\Omega't}{2}&-\Omega\sin\dfrac{\Omega't}{2}\\
    \Omega\sin\dfrac{\Omega't}{2}&\Delta\sin\dfrac{\Omega't}{2}+\Omega'\cos\dfrac{\Omega't}{2}
  \end{pmatrix}\begin{pmatrix}
    1\\
    0
  \end{pmatrix}\\
  =&\frac{\ue^{-\Gamma t/2}}{\Omega'}\begin{pmatrix}
    -\Delta\sin\dfrac{\Omega't}{2}+\Omega'\cos\dfrac{\Omega't}{2}\\
    \Omega\sin\dfrac{\Omega't}{2}
  \end{pmatrix}
}
\subsection{Verify the solution}
Time derivative
\eqar{
  \diff{\psi}{t}=&\diff{}{t}\frac{\ue^{-\Gamma t/2}}{\Omega'}\begin{pmatrix}
    -\Delta\sin\dfrac{\Omega't}{2}+\Omega'\cos\dfrac{\Omega't}{2}\\
    \Omega\sin\dfrac{\Omega't}{2}
  \end{pmatrix}\\
  =&\frac{-\Gamma\ue^{-\Gamma t/2}}{2\Omega'}\begin{pmatrix}
    -\Delta\sin\dfrac{\Omega't}{2}+\Omega'\cos\dfrac{\Omega't}{2}\\
    \Omega\sin\dfrac{\Omega't}{2}
  \end{pmatrix}+\frac{\ue^{-\Gamma t/2}}{2}\begin{pmatrix}
    -\Delta\cos\dfrac{\Omega't}{2}-\Omega'\sin\dfrac{\Omega't}{2}\\
    \Omega\cos\dfrac{\Omega't}{2}
  \end{pmatrix}\\
  =&\frac{\ue^{-\Gamma t/2}}{2\Omega'}\begin{pmatrix}
    \Gamma\Delta\sin\dfrac{\Omega't}{2}-\Gamma\Omega'\cos\dfrac{\Omega't}{2}-\Omega'\Delta\cos\dfrac{\Omega't}{2}-\Omega'^2\sin\dfrac{\Omega't}{2}\\
    -\Gamma\Omega\sin\dfrac{\Omega't}{2}+\Omega'\Omega\cos\dfrac{\Omega't}{2}
  \end{pmatrix}
}
Hamiltonian term,
\eqar{
  \ui H'\psi=&\frac{\ue^{-\Gamma t/2}}{2\Omega'}\begin{pmatrix}
    \Gamma+\Delta&\Omega\\
    -\Omega&\Gamma-\Delta
  \end{pmatrix}\begin{pmatrix}
    -\Delta\sin\dfrac{\Omega't}{2}+\Omega'\cos\dfrac{\Omega't}{2}\\
    \Omega\sin\dfrac{\Omega't}{2}
  \end{pmatrix}\\
  =&\frac{\ue^{-\Gamma t/2}}{2\Omega'}\begin{pmatrix}
    \paren{\Gamma+\Delta}\paren{-\Delta\sin\dfrac{\Omega't}{2}+\Omega'\cos\dfrac{\Omega't}{2}}+\Omega^2\sin\dfrac{\Omega't}{2}\\
    -\Omega\paren{-\Delta\sin\dfrac{\Omega't}{2}+\Omega'\cos\dfrac{\Omega't}{2}}+\paren{\Gamma-\Delta}\Omega\sin\dfrac{\Omega't}{2}
  \end{pmatrix}\\
  =&\frac{\ue^{-\Gamma t/2}}{2\Omega'}\begin{pmatrix}
    \paren{-\Delta\paren{\Gamma+\Delta}\sin\dfrac{\Omega't}{2}+\Omega'\paren{\Gamma+\Delta}\cos\dfrac{\Omega't}{2}}+\Omega^2\sin\dfrac{\Omega't}{2}\\
    \Omega\paren{\Gamma\sin\dfrac{\Omega't}{2}-\Omega'\cos\dfrac{\Omega't}{2}}
  \end{pmatrix}
}
\[\paren{\diff{}{t}+\ui H'}\psi=0\]


\section{Decay probabilities}
Now calculalte the decay probability from each states,
this is the main quantity we care about in the simulation.
The decay rates from each states are the expectation value of the jump operators
that jumps from the state.
The probability that decay has happened at a certain time is the integral of the rate.

The decay rate for state 1 is
\eqar{
  &\langle\psi|C_1^\dagger C_1|\psi\rangle\\
  =&\frac{\paren{\Gamma+\Delta}\ue^{-\Gamma t}}{\Omega'^2}\paren{-\Delta\sin\dfrac{\Omega't}{2}+\Omega'\cos\dfrac{\Omega't}{2}}^2\\
  =&\frac{\paren{\Gamma+\Delta}\ue^{-\Gamma t}}{\Omega^2-\Delta^2}\paren{
    \Delta^2\sin^2\dfrac{\Omega't}{2}+\paren{\Omega^2-\Delta^2}\cos^2\dfrac{\Omega't}{2}
    -\Delta\Omega'\sin\Omega't
  }\\
  =&\frac{\paren{\Gamma+\Delta}\ue^{-\Gamma t}}{\Omega^2-\Delta^2}\paren{
    \frac{\Omega^2}2+\paren{\frac{\Omega^2}2-\Delta^2}\cos\Omega't-\Delta\Omega'\sin\Omega't
  }
}
Total decay probability for state 1 is
\eqar{
  p_1(T)=&\int_0^T\ud t\frac{\paren{\Gamma+\Delta}\ue^{-\Gamma t}}{\Omega^2-\Delta^2}\paren{
    \frac{\Omega^2}2+\paren{\frac{\Omega^2}2-\Delta^2}\cos\Omega't-\Delta\Omega'\sin\Omega't
  }\\
  =&\frac{\paren{\Delta+\Gamma}}{2\Gamma\paren{\Omega^2-\Delta^2}\paren{\Omega^2+\Gamma^2-\Delta^2}}\\
  &\paren{\Omega^4+\Gamma^2\Omega^2-\Delta^2\Omega^2+2\Delta^3\Gamma-2\Delta^2\Gamma^2-2\Delta\Gamma\Omega^2+\Gamma^2\Omega^2}-\frac{\paren{\Delta+\Gamma}\ue^{-\Gamma t}}{2\Gamma\paren{\Omega^2-\Delta^2}\paren{\Omega^2+\Gamma^2-\Delta^2}}\\
  &\paren{\Omega^2\paren{\Omega^2+\Gamma^2-\Delta^2}+\Gamma\Omega'\paren{2\Delta^2-2\Delta\Gamma-\Omega^2}\sin\Omega't+\Gamma\paren{2\Delta^3-2\Delta^2\Gamma-2\Delta\Omega^2+\Gamma\Omega^2}\cos\Omega't}\\
  =&\frac{\paren{\Delta+\Gamma}\paren{\Omega^2+2\Gamma^2-2\Delta\Gamma}}{2\Gamma\paren{\Omega^2+\Gamma^2-\Delta^2}}-\frac{\paren{\Delta+\Gamma}\ue^{-\Gamma t}}{2\Gamma\paren{\Omega^2-\Delta^2}\paren{\Omega^2+\Gamma^2-\Delta^2}}\\
  &\paren{\Omega^2\paren{\Omega^2+\Gamma^2-\Delta^2}+\Gamma\Omega'\paren{2\Delta^2-2\Delta\Gamma-\Omega^2}\sin\Omega't+\Gamma\paren{2\Delta^3-2\Delta^2\Gamma-2\Delta\Omega^2+\Gamma\Omega^2}\cos\Omega't}
}
The decay rate for state 2 is
\eqar{
  &\langle\psi|C_2^\dagger C_2|\psi\rangle\\
  =&\frac{\paren{\Gamma-\Delta}\Omega^2\ue^{-\Gamma t}}{\Omega^2-\Delta^2}\sin^2\dfrac{\Omega't}{2}\\
  =&\frac{\paren{\Gamma-\Delta}\Omega^2}{2\paren{\Omega^2-\Delta^2}}\ue^{-\Gamma t}\paren{1-\cos\Omega't}
}
Total decay proobability for state 2 is
\eqar{
  p_2(T)=&\int_0^T\ud t\frac{\paren{\Gamma-\Delta}\Omega^2}{2\paren{\Omega^2-\Delta^2}}\ue^{-\Gamma t}\paren{1-\cos\Omega't}\\
  =&\frac{\Omega^2\paren{\Gamma-\Delta}}{2\Gamma\paren{\Omega^2+\Gamma^2-\Delta^2}}-\frac{\Omega^2\paren{\Gamma-\Delta}\ue^{-\Gamma t}}{2\Gamma\paren{\Omega^2-\Delta^2}\paren{\Omega^2+\Gamma^2-\Delta^2}}\paren{\Omega^2+\Gamma^2-\Delta^2-\Gamma^2\cos\Omega't+\Gamma\Omega'\sin\Omega't}
}
We also need the total decay probability in the simulation to decide the time
at which the decay happens. Total decay proobability for both states is
\eqar{
  &p_1(T)+p_2(T)\\
  =&\frac{\paren{\Delta+\Gamma}\paren{\Omega^2+2\Gamma^2-2\Delta\Gamma}}{2\Gamma\paren{\Omega^2+\Gamma^2-\Delta^2}}+\frac{\Omega^2\paren{\Gamma-\Delta}}{2\Gamma\paren{\Omega^2+\Gamma^2-\Delta^2}}\\
  &-\frac{\paren{\Delta+\Gamma}\ue^{-\Gamma t}}{2\Gamma\paren{\Omega^2-\Delta^2}\paren{\Omega^2+\Gamma^2-\Delta^2}}\\
  &\paren{\Omega^2\paren{\Omega^2+\Gamma^2-\Delta^2}+\Gamma\Omega'\paren{2\Delta^2-2\Delta\Gamma-\Omega^2}\sin\Omega't+\Gamma\paren{2\Delta^3-2\Delta^2\Gamma-2\Delta\Omega^2+\Gamma\Omega^2}\cos\Omega't}\\
  &-\frac{\Omega^2\paren{\Gamma-\Delta}\ue^{-\Gamma t}}{2\Gamma\paren{\Omega^2-\Delta^2}\paren{\Omega^2+\Gamma^2-\Delta^2}}\paren{\Omega^2+\Gamma^2-\Delta^2-\Gamma^2\cos\Omega't+\Gamma\Omega'\sin\Omega't}
  \intertext{}
  =&1-\frac{\Omega^2\paren{\Delta+\Gamma}\ue^{-\Gamma t}}{2\Gamma\paren{\Omega^2-\Delta^2}}\\
  &-\frac{\paren{\Delta+\Gamma}\paren{2\Delta^2-2\Delta\Gamma-\Omega^2}}{2\paren{\Omega^2-\Delta^2}\paren{\Omega^2+\Gamma^2-\Delta^2}}\ue^{-\Gamma t}\Omega'\sin\Omega't\\
  &-\frac{\paren{\Delta+\Gamma}\paren{2\Delta^3-2\Delta^2\Gamma-2\Delta\Omega^2+\Gamma\Omega^2}}{2\paren{\Omega^2-\Delta^2}\paren{\Omega^2+\Gamma^2-\Delta^2}}\ue^{-\Gamma t}\cos\Omega't\\
  &-\frac{\Omega^2\paren{\Gamma-\Delta}\ue^{-\Gamma t}}{2\Gamma\paren{\Omega^2-\Delta^2}}\\
  &-\frac{\Omega^2\paren{\Gamma-\Delta}}{2\paren{\Omega^2-\Delta^2}\paren{\Omega^2+\Gamma^2-\Delta^2}}\ue^{-\Gamma t}\Omega'\sin\Omega't\\
  &+\frac{\Omega^2\Gamma\paren{\Gamma-\Delta}}{2\paren{\Omega^2-\Delta^2}\paren{\Omega^2+\Gamma^2-\Delta^2}}\ue^{-\Gamma t}\cos\Omega't\\
  =&1-\frac{\Omega^2\ue^{-\Gamma t}}{\Omega^2-\Delta^2}+\frac{\Delta\ue^{-\Gamma t}\Omega'\sin\Omega't}{\Omega^2-\Delta^2}+\frac{\Delta^2\ue^{-\Gamma t}\cos\Omega't}{\Omega^2-\Delta^2}\\
  =&1-\frac{\ue^{-\Gamma t}\paren{\Omega^2-\Delta\Omega'\sin\Omega't-\Delta^2\cos\Omega't}}{\Omega^2-\Delta^2}
}
\subsection{Verification}
Compare this to the normalization of the wave function
\eqar{
  \langle\psi|\psi\rangle=&\frac{\ue^{-\Gamma t}}{\Omega^2-\Delta^2}\paren{
    \paren{-\Delta\sin\dfrac{\Omega't}{2}+\Omega'\cos\dfrac{\Omega't}{2}}^2+
    \paren{\Omega\sin\dfrac{\Omega't}{2}}^2
  }\\
  =&\frac{\ue^{-\Gamma t}}{\Omega^2-\Delta^2}\paren{
    \Delta^2\sin^2\dfrac{\Omega't}{2}+\Omega'^2\cos^2\dfrac{\Omega't}{2}
    -2\Delta\Omega'\sin\dfrac{\Omega't}{2}\cos\dfrac{\Omega't}{2}+
    \Omega^2\sin^2\dfrac{\Omega't}{2}
  }\\
  =&\frac{\ue^{-\Gamma t}}{\Omega^2-\Delta^2}\paren{
    \Delta^2\sin^2\dfrac{\Omega't}{2}+\paren{\Omega^2-\Delta^2}\cos^2\dfrac{\Omega't}{2}
    -\Delta\Omega'\sin\Omega't+
    \Omega^2\sin^2\dfrac{\Omega't}{2}
  }\\
  =&\frac{\ue^{-\Gamma t}}{\Omega^2-\Delta^2}\paren{
    \Omega^2-\Delta^2\cos\Omega't-\Delta\Omega'\sin\Omega't
  }
}
We have
\[p_1 + p_2+\langle\psi|\psi\rangle=1\]
\subsection{Derivative}
Calculate the derivative of $\langle\psi|\psi\rangle$ to help root finding with Newton's method.
\eqar{
  \diff{}{t}\langle\psi|\psi\rangle=&\diff{}{t}\frac{\ue^{-\Gamma t}}{\Omega^2-\Delta^2}\paren{
    \Omega^2-\Delta^2\cos\Omega't-\Delta\Omega'\sin\Omega't
  }\\
  =&-\frac{\Gamma\ue^{-\Gamma t}}{\Omega^2-\Delta^2}\paren{
    \Omega^2-\Delta^2\cos\Omega't-\Delta\Omega'\sin\Omega't
  }+\frac{\ue^{-\Gamma t}}{\Omega^2-\Delta^2}\paren{
    \Delta^2\Omega'\sin\Omega't-\Delta\Omega'^2\cos\Omega't
  }\\
  =&-\frac{\ue^{-\Gamma t}}{\Omega^2-\Delta^2}\paren{
    \Gamma\Omega^2
    -\Delta\Omega'\paren{\Delta+\Gamma}\sin\Omega't
    +\Delta\paren{\Omega'^2-\Gamma\Delta}\cos\Omega't
  }
}
Another way to calculate this
\eqar{
  \diff{}{t}\langle\psi|\psi\rangle=&-\langle\psi|C_1^\dagger C_1|\psi\rangle-\langle\psi|C_2^\dagger C_2|\psi\rangle\\
  =&-\frac{\paren{\Gamma+\Delta}\ue^{-\Gamma t}}{2\paren{\Omega^2-\Delta^2}}\paren{
    \Omega^2+\paren{\Omega^2-2\Delta^2}\cos\Omega't-2\Delta\Omega'\sin\Omega't
  }-\frac{\paren{\Gamma-\Delta}\Omega^2}{2\paren{\Omega^2-\Delta^2}}\ue^{-\Gamma t}\paren{1-\cos\Omega't}\\
  =&-\frac{\ue^{-\Gamma t}}{\paren{\Omega^2-\Delta^2}}\paren{
    \Gamma\Omega^2-\Delta\Omega'\paren{\Gamma+\Delta}\sin\Omega't+\Delta\paren{\Omega^2-\Delta^2-\Gamma\Delta}\cos\Omega't
  }
}

\section{Overdamp regime}
The results above should work for overdamping regime too.
However, the calculation will have imaginary numbers and is not very convenient.
We can reformulate the expressions using $\sinh$ and $\cosh$ functions so that it only involve
real numbers. There's also the critical case but that is unlikely to show up in real calculation
so it's ignored for now....

Define $\Delta'\equiv-\ui\Omega'=\sqrt{\Delta^2-\Omega^2}$

Wavefunctions
\eqar{
  \psi=&\frac{\ue^{-\Gamma t/2}}{\ui\Delta'}\begin{pmatrix}
    -\Delta\sin\dfrac{\ui\Delta't}{2}+\ui\Delta'\cos\dfrac{\ui\Delta't}{2}\\
    \Omega\sin\dfrac{\ui\Delta't}{2}
  \end{pmatrix}\\
  =&\frac{\ue^{-\Gamma t/2}}{\Delta'}\begin{pmatrix}
    -\Delta\sinh\dfrac{\Delta't}{2}+\Delta'\cosh\dfrac{\Delta't}{2}\\
    \Omega\sinh\dfrac{\Delta't}{2}
  \end{pmatrix}\\
  =&\frac{\ue^{-\Gamma t/2}}{2\Delta'}\begin{pmatrix}
    -\Delta\paren{\exp\paren{\dfrac{\Delta't}{2}}-\exp\paren{-\dfrac{\Delta't}{2}}}+\Delta'\paren{\exp\paren{\dfrac{\Delta't}{2}}+\exp\paren{-\dfrac{\Delta't}{2}}}\\
    \Omega\paren{\exp\paren{\dfrac{\Delta't}{2}}-\exp\paren{-\dfrac{\Delta't}{2}}}
  \end{pmatrix}\\
  =&\frac{\ue^{-\Gamma t/2}}{2\Delta'}\begin{pmatrix}
    \paren{\Delta'-\Delta}\exp\paren{\dfrac{\Delta't}{2}}+\paren{\Delta+\Delta'}\exp\paren{-\dfrac{\Delta't}{2}}\\
    \Omega\paren{\exp\paren{\dfrac{\Delta't}{2}}-\exp\paren{-\dfrac{\Delta't}{2}}}
  \end{pmatrix}
}
Decay probabilities
\eqar{
  \langle\psi|\psi\rangle=&\frac{\ue^{-\Gamma t}\paren{\Delta^2\cosh\Delta't-\Delta\Delta'\sinh\Delta't-\Omega^2}}{\Delta^2-\Omega^2}\\
  =&\frac{\ue^{-\Gamma t}\paren{\Delta\paren{\paren{\Delta-\Delta'}\ue^{\Delta't}+\paren{\Delta+\Delta'}\ue^{-\Delta't}}-2\Omega^2}}{2\paren{\Delta^2-\Omega^2}}
}
\eqar{
  \diff{}{t}\langle\psi|\psi\rangle=&\frac{\ue^{-\Gamma t}}{\Delta^2-\Omega^2}\paren{
    \Gamma\Omega^2+\Delta\Delta'\paren{\Gamma+\Delta}\sinh\Delta't+\Delta\paren{\Omega^2-\Delta^2-\Gamma\Delta}\cosh\Delta't
  }\\
  =&\frac{\ue^{-\Gamma t}}{2\paren{\Delta^2-\Omega^2}}\paren{
    2\Gamma\Omega^2+\Delta\Delta'\paren{\Gamma+\Delta}\paren{\ue^{\Delta't}-\ue^{-\Delta't}}+\Delta\paren{\Omega^2-\Delta^2-\Gamma\Delta}\paren{\ue^{\Delta't}+\ue^{-\Delta't}}
  }\\
  =&\frac{\ue^{-\Gamma t}}{2\paren{\Delta^2-\Omega^2}}\paren{
    2\Gamma\Omega^2
    -\Delta\paren{\Gamma-\Delta'}\paren{\Delta-\Delta'}\ue^{\Delta't}
    -\Delta\paren{\Gamma+\Delta'}\paren{\Delta+\Delta'}\ue^{-\Delta't}
  }
}
Decay probability from state 2
\eqar{
  &\langle\psi|C_2^\dagger C_2|\psi\rangle\\
  =&\frac{\paren{\Gamma-\Delta}\Omega^2}{2\paren{\Omega^2-\Delta^2}}\ue^{-\Gamma t}\paren{1-\cosh\Delta't}\\
  =&\frac{\ue^{-\Gamma t}}{2\paren{\Delta^2-\Omega^2}}\frac{\paren{\Gamma-\Delta}\Omega^2}{2}\paren{\ue^{\Delta't}+\ue^{-p\Delta't}-2}
}

\end{document}
