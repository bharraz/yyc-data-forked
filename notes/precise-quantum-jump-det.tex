\documentclass[10pt,fleqn]{article}
% \usepackage[journal=rsc]{chemstyle}
% \usepackage{mhchem}
\usepackage{amsmath}
\usepackage{amssymb}
\usepackage{amsfonts}
\usepackage{esint}
\usepackage{bbm}
\usepackage{amscd}
\usepackage{picinpar}
\usepackage{graphicx}
\usepackage{tikz}
\usepackage{tikz-3dplot}
\usepackage{indentfirst}
\usepackage{wrapfig}
\usepackage{units}
\usepackage{textcomp}
\usepackage[utf8x]{inputenc}
% \usepackage{feyn}
\usepackage{feynmp}
\usepackage{xkeyval}
\usepackage{xargs}
\usepackage{verbatim}
\usepackage{pgfplots}
\usepackage{hyperref}
\usetikzlibrary{
  arrows,
  calc,
  decorations.pathmorphing,
  decorations.pathreplacing,
  decorations.markings,
  fadings,
  positioning,
  shapes
}

\DeclareGraphicsRule{*}{mps}{*}{}
\newcommand{\ud}{\mathrm{d}}
\newcommand{\ue}{\mathrm{e}}
\newcommand{\ui}{\mathrm{i}}
\newcommand{\res}{\mathrm{Res}}
\newcommand{\Tr}{\mathrm{Tr}}
\newcommand{\dsum}{\displaystyle\sum}
\newcommand{\dprod}{\displaystyle\prod}
\newcommand{\dlim}{\displaystyle\lim}
\newcommand{\dint}{\displaystyle\int}
\newcommand{\fsno}[1]{{\!\not\!{#1}}}
\newcommand{\eqar}[1]
{
  \begin{align*}
    #1
  \end{align*}
}
\newcommand{\texp}[2]{\ensuremath{{#1}\times10^{#2}}}
\newcommand{\dexp}[2]{\ensuremath{{#1}\cdot10^{#2}}}
\newcommand{\eval}[2]{{\left.{#1}\right|_{#2}}}
\newcommand{\paren}[1]{{\left({#1}\right)}}
\newcommand{\lparen}[1]{{\left({#1}\right.}}
\newcommand{\rparen}[1]{{\left.{#1}\right)}}
\newcommand{\abs}[1]{{\left|{#1}\right|}}
\newcommand{\sqr}[1]{{\left[{#1}\right]}}
\newcommand{\crly}[1]{{\left\{{#1}\right\}}}
\newcommand{\angl}[1]{{\left\langle{#1}\right\rangle}}
\newcommand{\tpdiff}[4][{}]{{\paren{\frac{\partial^{#1} {#2}}{\partial {#3}{}^{#1}}}_{#4}}}
\newcommand{\tpsdiff}[4][{}]{{\paren{\frac{\partial^{#1}}{\partial {#3}{}^{#1}}{#2}}_{#4}}}
\newcommand{\pdiff}[3][{}]{{\frac{\partial^{#1} {#2}}{\partial {#3}{}^{#1}}}}
\newcommand{\diff}[3][{}]{{\frac{\ud^{#1} {#2}}{\ud {#3}{}^{#1}}}}
\newcommand{\psdiff}[3][{}]{{\frac{\partial^{#1}}{\partial {#3}{}^{#1}} {#2}}}
\newcommand{\sdiff}[3][{}]{{\frac{\ud^{#1}}{\ud {#3}{}^{#1}} {#2}}}
\newcommand{\tpddiff}[4][{}]{{\left(\dfrac{\partial^{#1} {#2}}{\partial {#3}{}^{#1}}\right)_{#4}}}
\newcommand{\tpsddiff}[4][{}]{{\paren{\dfrac{\partial^{#1}}{\partial {#3}{}^{#1}}{#2}}_{#4}}}
\newcommand{\pddiff}[3][{}]{{\dfrac{\partial^{#1} {#2}}{\partial {#3}{}^{#1}}}}
\newcommand{\ddiff}[3][{}]{{\dfrac{\ud^{#1} {#2}}{\ud {#3}{}^{#1}}}}
\newcommand{\psddiff}[3][{}]{{\frac{\partial^{#1}}{\partial{}^{#1} {#3}} {#2}}}
\newcommand{\sddiff}[3][{}]{{\frac{\ud^{#1}}{\ud {#3}{}^{#1}} {#2}}}
\usepackage{fancyhdr}
\usepackage{multirow}
\usepackage{fontenc}
% \usepackage{tipa}
\usepackage{ulem}
\usepackage{color}
\usepackage{cancel}
\newcommand{\hcancel}[2][black]{\setbox0=\hbox{#2}%
  \rlap{\raisebox{.45\ht0}{\textcolor{#1}{\rule{\wd0}{1pt}}}}#2}
\pagestyle{fancy}
\setlength{\headheight}{67pt}
\fancyhead{}
\fancyfoot{}
\fancyfoot[C]{\thepage}
\fancyhead[R]{}
\renewcommand{\footruleskip}{0pt}
\renewcommand{\headrulewidth}{0.4pt}
\renewcommand{\footrulewidth}{0pt}

\newcommand\pgfmathsinandcos[3]{%
  \pgfmathsetmacro#1{sin(#3)}%
  \pgfmathsetmacro#2{cos(#3)}%
}
\newcommand\LongitudePlane[3][current plane]{%
  \pgfmathsinandcos\sinEl\cosEl{#2} % elevation
  \pgfmathsinandcos\sint\cost{#3} % azimuth
  \tikzset{#1/.estyle={cm={\cost,\sint*\sinEl,0,\cosEl,(0,0)}}}
}
\newcommand\LatitudePlane[3][current plane]{%
  \pgfmathsinandcos\sinEl\cosEl{#2} % elevation
  \pgfmathsinandcos\sint\cost{#3} % latitude
  \pgfmathsetmacro\yshift{\cosEl*\sint}
  \tikzset{#1/.estyle={cm={\cost,0,0,\cost*\sinEl,(0,\yshift)}}} %
}
\newcommand\DrawLongitudeCircle[2][1]{
  \LongitudePlane{\angEl}{#2}
  \tikzset{current plane/.prefix style={scale=#1}}
  % angle of "visibility"
  \pgfmathsetmacro\angVis{atan(sin(#2)*cos(\angEl)/sin(\angEl))} %
  \draw[current plane] (\angVis:1) arc (\angVis:\angVis+180:1);
  \draw[current plane,dashed] (\angVis-180:1) arc (\angVis-180:\angVis:1);
}
\newcommand\DrawLatitudeCircleArrow[2][1]{
  \LatitudePlane{\angEl}{#2}
  \tikzset{current plane/.prefix style={scale=#1}}
  \pgfmathsetmacro\sinVis{sin(#2)/cos(#2)*sin(\angEl)/cos(\angEl)}
  % angle of "visibility"
  \pgfmathsetmacro\angVis{asin(min(1,max(\sinVis,-1)))}
  \draw[current plane,decoration={markings, mark=at position 0.6 with {\arrow{<}}},postaction={decorate},line width=.6mm] (\angVis:1) arc (\angVis:-\angVis-180:1);
  \draw[current plane,dashed,line width=.6mm] (180-\angVis:1) arc (180-\angVis:\angVis:1);
}
\newcommand\DrawLatitudeCircle[2][1]{
  \LatitudePlane{\angEl}{#2}
  \tikzset{current plane/.prefix style={scale=#1}}
  \pgfmathsetmacro\sinVis{sin(#2)/cos(#2)*sin(\angEl)/cos(\angEl)}
  % angle of "visibility"
  \pgfmathsetmacro\angVis{asin(min(1,max(\sinVis,-1)))}
  \draw[current plane] (\angVis:1) arc (\angVis:-\angVis-180:1);
  \draw[current plane,dashed] (180-\angVis:1) arc (180-\angVis:\angVis:1);
}
\newcommand\coil[1]{
  {\rh * cos(\t * pi r)}, {\apart * (2 * #1 + \t) + \rv * sin(\t * pi r)}
}
\makeatletter
\define@key{DrawFromCenter}{style}[{->}]{
  \tikzset{DrawFromCenterPlane/.style={#1}}
}
\define@key{DrawFromCenter}{r}[1]{
  \def\@R{#1}
}
\define@key{DrawFromCenter}{center}[(0, 0)]{
  \def\@Center{#1}
}
\define@key{DrawFromCenter}{theta}[0]{
  \def\@Theta{#1}
}
\define@key{DrawFromCenter}{phi}[0]{
  \def\@Phi{#1}
}
\presetkeys{DrawFromCenter}{style, r, center, theta, phi}{}
\newcommand*\DrawFromCenter[1][]{
  \setkeys{DrawFromCenter}{#1}{
    \pgfmathsinandcos\sint\cost{\@Theta}
    \pgfmathsinandcos\sinp\cosp{\@Phi}
    \pgfmathsinandcos\sinA\cosA{\angEl}
    \pgfmathsetmacro\DX{\@R*\cost*\cosp}
    \pgfmathsetmacro\DY{\@R*(\cost*\sinp*\sinA+\sint*\cosA)}
    \draw[DrawFromCenterPlane] \@Center -- ++(\DX, \DY);
  }
}
\newcommand*\DrawFromCenterText[2][]{
  \setkeys{DrawFromCenter}{#1}{
    \pgfmathsinandcos\sint\cost{\@Theta}
    \pgfmathsinandcos\sinp\cosp{\@Phi}
    \pgfmathsinandcos\sinA\cosA{\angEl}
    \pgfmathsetmacro\DX{\@R*\cost*\cosp}
    \pgfmathsetmacro\DY{\@R*(\cost*\sinp*\sinA+\sint*\cosA)}
    \draw[DrawFromCenterPlane] \@Center -- ++(\DX, \DY) node {#2};
  }
}
\makeatother
\tikzstyle{snakearrow} = [decorate, decoration={pre length=0.2cm,
  post length=0.2cm, snake, amplitude=.4mm,
  segment length=2mm},thick, ->]
%% document-wide tikz options and styles
\tikzset{%
  >=latex, % option for nice arrows
  inner sep=0pt,%
  outer sep=2pt,%
  mark coordinate/.style={inner sep=0pt,outer sep=0pt,minimum size=3pt,
    fill=black,circle}%
}
\addtolength{\hoffset}{-1.3cm}
\addtolength{\voffset}{-2cm}
\addtolength{\textwidth}{3cm}
\addtolength{\textheight}{2.5cm}
\renewcommand{\footskip}{10pt}
\setlength{\headwidth}{\textwidth}
\setlength{\headsep}{20pt}
\setlength{\marginparwidth}{0pt}
\parindent=0pt
\title{Exact formula for quantum jump method in two level system with decay and coupling including detuning}
\begin{document}

\maketitle

\section{The problem}
The derivation in \href{precise-quantum-jump.pdf}{the previous note} does not include
detuning caused by trap anharmonicity. The anharmonicity should be small in the small range
of states that is driven by any given orders but we would still like to include it in
the simulation to see how big the effect actually is.

Another effect that can be included is the decoherence. We should be able to simulate the
fast component with the decay term and the slow component
(e.g. B field drift that's slower than the experimental cycles) by a random overall detuning.

Since we are still only dealing with a single drive, the Hamiltonian will still
be time independent. We are also still ignoring the off-resonant scattering so the system
is still two-level. The main difference is that the diagnal part of the Hamiltonian will
now have a real part and the resulting state may be imaginary.

\section{Modified Hamiltonian}
With the detuning included, the effective Hamiltonian is now,
\eqar{
  H'=&-\frac{\ui}2\begin{pmatrix}
    \Gamma_1+\ui\delta&\Omega\\
    -\Omega&\Gamma_2-\ui\delta
  \end{pmatrix}
  \intertext{Define}
  \Gamma_1\equiv&\Gamma+\gamma\\
  \Gamma_2\equiv&\Gamma-\gamma
  \intertext{We have}
  H'=&-\frac{\ui}2\begin{pmatrix}
    \Gamma+\gamma+\ui\delta&\Omega\\
    -\Omega&\Gamma-\gamma-\ui\delta
  \end{pmatrix}
}

\section{Time evolution}

Formally the time evolution is
\eqar{
  &\exp\paren{-\ui H't}\\
  =&\exp\paren{-\frac t2\begin{pmatrix}
      \Gamma+\gamma+\ui\delta&\Omega\\
      -\Omega&\Gamma-\gamma-\ui\delta
    \end{pmatrix}}
}
% Define $\Omega'=\sqrt{\Omega^2-\gamma^2}$
% \eqar{
%   &\exp\paren{-\ui H't}\\
%   =&\frac{\ue^{-\Gamma t/2}}{2\ui\Omega'}\begin{pmatrix}
%     \gamma\paren{\ue^{-\frac{\ui\Omega't}{2}}-\ue^{\frac{\ui\Omega't}{2}}}+\ui\Omega'\paren{\ue^{-\frac{\ui\Omega't}{2}}+\ue^{\frac{\ui\Omega't}{2}}}&\Omega\paren{\ue^{-\frac{\ui\Omega't}{2}}-\ue^{\frac{\ui\Omega't}{2}}}\\
%     -\Omega\paren{\ue^{-\frac{\ui\Omega't}{2}}-\ue^{\frac{\ui\Omega't}{2}}}&-\gamma\paren{\ue^{-\frac{\ui\Omega't}{2}}-\ue^{\frac{\ui\Omega't}{2}}}+\ui\Omega'\paren{\ue^{-\frac{\ui\Omega't}{2}}+\ue^{\frac{\ui\Omega't}{2}}}
%   \end{pmatrix}\\
%   =&\frac{\ue^{-\Gamma t/2}}{2\ui\Omega'}\begin{pmatrix}
%     -2\ui\gamma\sin\dfrac{\Omega't}{2}+2\ui\Omega'\cos\dfrac{\Omega't}{2}&-2\ui\Omega\sin\dfrac{\Omega't}{2}\\
%     2\ui\Omega\sin\dfrac{\Omega't}{2}&2\ui\gamma\sin\dfrac{\Omega't}{2}+2\ui\Omega'\cos\dfrac{\Omega't}{2}
%   \end{pmatrix}\\
%   =&\frac{\ue^{-\Gamma t/2}}{\Omega'}\begin{pmatrix}
%     -\gamma\sin\dfrac{\Omega't}{2}+\Omega'\cos\dfrac{\Omega't}{2}&-\Omega\sin\dfrac{\Omega't}{2}\\
%     \Omega\sin\dfrac{\Omega't}{2}&\gamma\sin\dfrac{\Omega't}{2}+\Omega'\cos\dfrac{\Omega't}{2}
%   \end{pmatrix}
% }
% Starting from the atom in state 1, the wave functions are
% \eqar{
%   \psi=&\frac{\ue^{-\Gamma t/2}}{\Omega'}\begin{pmatrix}
%     -\gamma\sin\dfrac{\Omega't}{2}+\Omega'\cos\dfrac{\Omega't}{2}&-\Omega\sin\dfrac{\Omega't}{2}\\
%     \Omega\sin\dfrac{\Omega't}{2}&\gamma\sin\dfrac{\Omega't}{2}+\Omega'\cos\dfrac{\Omega't}{2}
%   \end{pmatrix}\begin{pmatrix}
%     1\\
%     0
%   \end{pmatrix}\\
%   =&\frac{\ue^{-\Gamma t/2}}{\Omega'}\begin{pmatrix}
%     -\gamma\sin\dfrac{\Omega't}{2}+\Omega'\cos\dfrac{\Omega't}{2}\\
%     \Omega\sin\dfrac{\Omega't}{2}
%   \end{pmatrix}
% }
% \subsection{Verify the solution}
% Time derivative
% \eqar{
%   \diff{\psi}{t}=&\diff{}{t}\frac{\ue^{-\Gamma t/2}}{\Omega'}\begin{pmatrix}
%     -\gamma\sin\dfrac{\Omega't}{2}+\Omega'\cos\dfrac{\Omega't}{2}\\
%     \Omega\sin\dfrac{\Omega't}{2}
%   \end{pmatrix}\\
%   =&\frac{-\Gamma\ue^{-\Gamma t/2}}{2\Omega'}\begin{pmatrix}
%     -\gamma\sin\dfrac{\Omega't}{2}+\Omega'\cos\dfrac{\Omega't}{2}\\
%     \Omega\sin\dfrac{\Omega't}{2}
%   \end{pmatrix}+\frac{\ue^{-\Gamma t/2}}{2}\begin{pmatrix}
%     -\gamma\cos\dfrac{\Omega't}{2}-\Omega'\sin\dfrac{\Omega't}{2}\\
%     \Omega\cos\dfrac{\Omega't}{2}
%   \end{pmatrix}\\
%   =&\frac{\ue^{-\Gamma t/2}}{2\Omega'}\begin{pmatrix}
%     \Gamma\gamma\sin\dfrac{\Omega't}{2}-\Gamma\Omega'\cos\dfrac{\Omega't}{2}-\Omega'\gamma\cos\dfrac{\Omega't}{2}-\Omega'^2\sin\dfrac{\Omega't}{2}\\
%     -\Gamma\Omega\sin\dfrac{\Omega't}{2}+\Omega'\Omega\cos\dfrac{\Omega't}{2}
%   \end{pmatrix}
% }
% Hamiltonian term,
% \eqar{
%   \ui H'\psi=&\frac{\ue^{-\Gamma t/2}}{2\Omega'}\begin{pmatrix}
%     \Gamma+\gamma&\Omega\\
%     -\Omega&\Gamma-\gamma
%   \end{pmatrix}\begin{pmatrix}
%     -\gamma\sin\dfrac{\Omega't}{2}+\Omega'\cos\dfrac{\Omega't}{2}\\
%     \Omega\sin\dfrac{\Omega't}{2}
%   \end{pmatrix}\\
%   =&\frac{\ue^{-\Gamma t/2}}{2\Omega'}\begin{pmatrix}
%     \paren{\Gamma+\gamma}\paren{-\gamma\sin\dfrac{\Omega't}{2}+\Omega'\cos\dfrac{\Omega't}{2}}+\Omega^2\sin\dfrac{\Omega't}{2}\\
%     -\Omega\paren{-\gamma\sin\dfrac{\Omega't}{2}+\Omega'\cos\dfrac{\Omega't}{2}}+\paren{\Gamma-\gamma}\Omega\sin\dfrac{\Omega't}{2}
%   \end{pmatrix}\\
%   =&\frac{\ue^{-\Gamma t/2}}{2\Omega'}\begin{pmatrix}
%     \paren{-\gamma\paren{\Gamma+\gamma}\sin\dfrac{\Omega't}{2}+\Omega'\paren{\Gamma+\gamma}\cos\dfrac{\Omega't}{2}}+\Omega^2\sin\dfrac{\Omega't}{2}\\
%     \Omega\paren{\Gamma\sin\dfrac{\Omega't}{2}-\Omega'\cos\dfrac{\Omega't}{2}}
%   \end{pmatrix}
% }
% \[\paren{\diff{}{t}+\ui H'}\psi=0\]


% \section{Decay probabilities}
% Now calculalte the decay probability from each states,
% this is the main quantity we care about in the simulation.
% The decay rates from each states are the expectation value of the jump operators
% that jumps from the state.
% The probability that decay has happened at a certain time is the integral of the rate.

% The decay rate for state 1 is
% \eqar{
%   &\langle\psi|C_1^\dagger C_1|\psi\rangle\\
%   =&\frac{\paren{\Gamma+\gamma}\ue^{-\Gamma t}}{\Omega'^2}\paren{-\gamma\sin\dfrac{\Omega't}{2}+\Omega'\cos\dfrac{\Omega't}{2}}^2\\
%   =&\frac{\paren{\Gamma+\gamma}\ue^{-\Gamma t}}{\Omega^2-\gamma^2}\paren{
%     \gamma^2\sin^2\dfrac{\Omega't}{2}+\paren{\Omega^2-\gamma^2}\cos^2\dfrac{\Omega't}{2}
%     -\gamma\Omega'\sin\Omega't
%   }\\
%   =&\frac{\paren{\Gamma+\gamma}\ue^{-\Gamma t}}{\Omega^2-\gamma^2}\paren{
%     \frac{\Omega^2}2+\paren{\frac{\Omega^2}2-\gamma^2}\cos\Omega't-\gamma\Omega'\sin\Omega't
%   }
% }
% Total decay probability for state 1 is
% \eqar{
%   p_1(T)=&\int_0^T\ud t\frac{\paren{\Gamma+\gamma}\ue^{-\Gamma t}}{\Omega^2-\gamma^2}\paren{
%     \frac{\Omega^2}2+\paren{\frac{\Omega^2}2-\gamma^2}\cos\Omega't-\gamma\Omega'\sin\Omega't
%   }\\
%   =&\frac{\paren{\gamma+\Gamma}}{2\Gamma\paren{\Omega^2-\gamma^2}\paren{\Omega^2+\Gamma^2-\gamma^2}}\\
%   &\paren{\Omega^4+\Gamma^2\Omega^2-\gamma^2\Omega^2+2\gamma^3\Gamma-2\gamma^2\Gamma^2-2\gamma\Gamma\Omega^2+\Gamma^2\Omega^2}-\frac{\paren{\gamma+\Gamma}\ue^{-\Gamma t}}{2\Gamma\paren{\Omega^2-\gamma^2}\paren{\Omega^2+\Gamma^2-\gamma^2}}\\
%   &\paren{\Omega^2\paren{\Omega^2+\Gamma^2-\gamma^2}+\Gamma\Omega'\paren{2\gamma^2-2\gamma\Gamma-\Omega^2}\sin\Omega't+\Gamma\paren{2\gamma^3-2\gamma^2\Gamma-2\gamma\Omega^2+\Gamma\Omega^2}\cos\Omega't}\\
%   =&\frac{\paren{\gamma+\Gamma}\paren{\Omega^2+2\Gamma^2-2\gamma\Gamma}}{2\Gamma\paren{\Omega^2+\Gamma^2-\gamma^2}}-\frac{\paren{\gamma+\Gamma}\ue^{-\Gamma t}}{2\Gamma\paren{\Omega^2-\gamma^2}\paren{\Omega^2+\Gamma^2-\gamma^2}}\\
%   &\paren{\Omega^2\paren{\Omega^2+\Gamma^2-\gamma^2}+\Gamma\Omega'\paren{2\gamma^2-2\gamma\Gamma-\Omega^2}\sin\Omega't+\Gamma\paren{2\gamma^3-2\gamma^2\Gamma-2\gamma\Omega^2+\Gamma\Omega^2}\cos\Omega't}
% }
% The decay rate for state 2 is
% \eqar{
%   &\langle\psi|C_2^\dagger C_2|\psi\rangle\\
%   =&\frac{\paren{\Gamma-\gamma}\Omega^2\ue^{-\Gamma t}}{\Omega^2-\gamma^2}\sin^2\dfrac{\Omega't}{2}\\
%   =&\frac{\paren{\Gamma-\gamma}\Omega^2}{2\paren{\Omega^2-\gamma^2}}\ue^{-\Gamma t}\paren{1-\cos\Omega't}
% }
% Total decay proobability for state 2 is
% \eqar{
%   p_2(T)=&\int_0^T\ud t\frac{\paren{\Gamma-\gamma}\Omega^2}{2\paren{\Omega^2-\gamma^2}}\ue^{-\Gamma t}\paren{1-\cos\Omega't}\\
%   =&\frac{\Omega^2\paren{\Gamma-\gamma}}{2\Gamma\paren{\Omega^2+\Gamma^2-\gamma^2}}-\frac{\Omega^2\paren{\Gamma-\gamma}\ue^{-\Gamma t}}{2\Gamma\paren{\Omega^2-\gamma^2}\paren{\Omega^2+\Gamma^2-\gamma^2}}\paren{\Omega^2+\Gamma^2-\gamma^2-\Gamma^2\cos\Omega't+\Gamma\Omega'\sin\Omega't}
% }
% We also need the total decay probability in the simulation to decide the time
% at which the decay happens. Total decay proobability for both states is
% \eqar{
%   &p_1(T)+p_2(T)\\
%   =&\frac{\paren{\gamma+\Gamma}\paren{\Omega^2+2\Gamma^2-2\gamma\Gamma}}{2\Gamma\paren{\Omega^2+\Gamma^2-\gamma^2}}+\frac{\Omega^2\paren{\Gamma-\gamma}}{2\Gamma\paren{\Omega^2+\Gamma^2-\gamma^2}}\\
%   &-\frac{\paren{\gamma+\Gamma}\ue^{-\Gamma t}}{2\Gamma\paren{\Omega^2-\gamma^2}\paren{\Omega^2+\Gamma^2-\gamma^2}}\\
%   &\paren{\Omega^2\paren{\Omega^2+\Gamma^2-\gamma^2}+\Gamma\Omega'\paren{2\gamma^2-2\gamma\Gamma-\Omega^2}\sin\Omega't+\Gamma\paren{2\gamma^3-2\gamma^2\Gamma-2\gamma\Omega^2+\Gamma\Omega^2}\cos\Omega't}\\
%   &-\frac{\Omega^2\paren{\Gamma-\gamma}\ue^{-\Gamma t}}{2\Gamma\paren{\Omega^2-\gamma^2}\paren{\Omega^2+\Gamma^2-\gamma^2}}\paren{\Omega^2+\Gamma^2-\gamma^2-\Gamma^2\cos\Omega't+\Gamma\Omega'\sin\Omega't}
%   \intertext{}
%   =&1-\frac{\Omega^2\paren{\gamma+\Gamma}\ue^{-\Gamma t}}{2\Gamma\paren{\Omega^2-\gamma^2}}\\
%   &-\frac{\paren{\gamma+\Gamma}\paren{2\gamma^2-2\gamma\Gamma-\Omega^2}}{2\paren{\Omega^2-\gamma^2}\paren{\Omega^2+\Gamma^2-\gamma^2}}\ue^{-\Gamma t}\Omega'\sin\Omega't\\
%   &-\frac{\paren{\gamma+\Gamma}\paren{2\gamma^3-2\gamma^2\Gamma-2\gamma\Omega^2+\Gamma\Omega^2}}{2\paren{\Omega^2-\gamma^2}\paren{\Omega^2+\Gamma^2-\gamma^2}}\ue^{-\Gamma t}\cos\Omega't\\
%   &-\frac{\Omega^2\paren{\Gamma-\gamma}\ue^{-\Gamma t}}{2\Gamma\paren{\Omega^2-\gamma^2}}\\
%   &-\frac{\Omega^2\paren{\Gamma-\gamma}}{2\paren{\Omega^2-\gamma^2}\paren{\Omega^2+\Gamma^2-\gamma^2}}\ue^{-\Gamma t}\Omega'\sin\Omega't\\
%   &+\frac{\Omega^2\Gamma\paren{\Gamma-\gamma}}{2\paren{\Omega^2-\gamma^2}\paren{\Omega^2+\Gamma^2-\gamma^2}}\ue^{-\Gamma t}\cos\Omega't\\
%   =&1-\frac{\Omega^2\ue^{-\Gamma t}}{\Omega^2-\gamma^2}+\frac{\gamma\ue^{-\Gamma t}\Omega'\sin\Omega't}{\Omega^2-\gamma^2}+\frac{\gamma^2\ue^{-\Gamma t}\cos\Omega't}{\Omega^2-\gamma^2}\\
%   =&1-\frac{\ue^{-\Gamma t}\paren{\Omega^2-\gamma\Omega'\sin\Omega't-\gamma^2\cos\Omega't}}{\Omega^2-\gamma^2}
% }
% \subsection{Verification}
% Compare this to the normalization of the wave function
% \eqar{
%   \langle\psi|\psi\rangle=&\frac{\ue^{-\Gamma t}}{\Omega^2-\gamma^2}\paren{
%     \paren{-\gamma\sin\dfrac{\Omega't}{2}+\Omega'\cos\dfrac{\Omega't}{2}}^2+
%     \paren{\Omega\sin\dfrac{\Omega't}{2}}^2
%   }\\
%   =&\frac{\ue^{-\Gamma t}}{\Omega^2-\gamma^2}\paren{
%     \gamma^2\sin^2\dfrac{\Omega't}{2}+\Omega'^2\cos^2\dfrac{\Omega't}{2}
%     -2\gamma\Omega'\sin\dfrac{\Omega't}{2}\cos\dfrac{\Omega't}{2}+
%     \Omega^2\sin^2\dfrac{\Omega't}{2}
%   }\\
%   =&\frac{\ue^{-\Gamma t}}{\Omega^2-\gamma^2}\paren{
%     \gamma^2\sin^2\dfrac{\Omega't}{2}+\paren{\Omega^2-\gamma^2}\cos^2\dfrac{\Omega't}{2}
%     -\gamma\Omega'\sin\Omega't+
%     \Omega^2\sin^2\dfrac{\Omega't}{2}
%   }\\
%   =&\frac{\ue^{-\Gamma t}}{\Omega^2-\gamma^2}\paren{
%     \Omega^2-\gamma^2\cos\Omega't-\gamma\Omega'\sin\Omega't
%   }
% }
% We have
% \[p_1 + p_2+\langle\psi|\psi\rangle=1\]
% \subsection{Derivative}
% Calculate the derivative of $\langle\psi|\psi\rangle$ to help root finding with Newton's method.
% \eqar{
%   \diff{}{t}\langle\psi|\psi\rangle=&\diff{}{t}\frac{\ue^{-\Gamma t}}{\Omega^2-\gamma^2}\paren{
%     \Omega^2-\gamma^2\cos\Omega't-\gamma\Omega'\sin\Omega't
%   }\\
%   =&-\frac{\Gamma\ue^{-\Gamma t}}{\Omega^2-\gamma^2}\paren{
%     \Omega^2-\gamma^2\cos\Omega't-\gamma\Omega'\sin\Omega't
%   }+\frac{\ue^{-\Gamma t}}{\Omega^2-\gamma^2}\paren{
%     \gamma^2\Omega'\sin\Omega't-\gamma\Omega'^2\cos\Omega't
%   }\\
%   =&-\frac{\ue^{-\Gamma t}}{\Omega^2-\gamma^2}\paren{
%     \Gamma\Omega^2
%     -\gamma\Omega'\paren{\gamma+\Gamma}\sin\Omega't
%     +\gamma\paren{\Omega'^2-\Gamma\gamma}\cos\Omega't
%   }
% }
% Another way to calculate this
% \eqar{
%   \diff{}{t}\langle\psi|\psi\rangle=&-\langle\psi|C_1^\dagger C_1|\psi\rangle-\langle\psi|C_2^\dagger C_2|\psi\rangle\\
%   =&-\frac{\paren{\Gamma+\gamma}\ue^{-\Gamma t}}{2\paren{\Omega^2-\gamma^2}}\paren{
%     \Omega^2+\paren{\Omega^2-2\gamma^2}\cos\Omega't-2\gamma\Omega'\sin\Omega't
%   }-\frac{\paren{\Gamma-\gamma}\Omega^2}{2\paren{\Omega^2-\gamma^2}}\ue^{-\Gamma t}\paren{1-\cos\Omega't}\\
%   =&-\frac{\ue^{-\Gamma t}}{\paren{\Omega^2-\gamma^2}}\paren{
%     \Gamma\Omega^2-\gamma\Omega'\paren{\Gamma+\gamma}\sin\Omega't+\gamma\paren{\Omega^2-\gamma^2-\Gamma\gamma}\cos\Omega't
%   }
% }

% \section{Overdamp regime}
% The results above should work for overdamping regime too.
% However, the calculation will have imaginary numbers and is not very convenient.
% We can reformulate the expressions using $\sinh$ and $\cosh$ functions so that it only involve
% real numbers. There's also the critical case but that is unlikely to show up in real calculation
% so it's ignored for now....

% Define $\gamma'\equiv-\ui\Omega'=\sqrt{\gamma^2-\Omega^2}$

% Wavefunctions
% \eqar{
%   \psi=&\frac{\ue^{-\Gamma t/2}}{\ui\gamma'}\begin{pmatrix}
%     -\gamma\sin\dfrac{\ui\gamma't}{2}+\ui\gamma'\cos\dfrac{\ui\gamma't}{2}\\
%     \Omega\sin\dfrac{\ui\gamma't}{2}
%   \end{pmatrix}\\
%   =&\frac{\ue^{-\Gamma t/2}}{\gamma'}\begin{pmatrix}
%     -\gamma\sinh\dfrac{\gamma't}{2}+\gamma'\cosh\dfrac{\gamma't}{2}\\
%     \Omega\sinh\dfrac{\gamma't}{2}
%   \end{pmatrix}\\
%   =&\frac{\ue^{-\Gamma t/2}}{2\gamma'}\begin{pmatrix}
%     -\gamma\paren{\exp\paren{\dfrac{\gamma't}{2}}-\exp\paren{-\dfrac{\gamma't}{2}}}+\gamma'\paren{\exp\paren{\dfrac{\gamma't}{2}}+\exp\paren{-\dfrac{\gamma't}{2}}}\\
%     \Omega\paren{\exp\paren{\dfrac{\gamma't}{2}}-\exp\paren{-\dfrac{\gamma't}{2}}}
%   \end{pmatrix}\\
%   =&\frac{\ue^{-\Gamma t/2}}{2\gamma'}\begin{pmatrix}
%     \paren{\gamma'-\gamma}\exp\paren{\dfrac{\gamma't}{2}}+\paren{\gamma+\gamma'}\exp\paren{-\dfrac{\gamma't}{2}}\\
%     \Omega\paren{\exp\paren{\dfrac{\gamma't}{2}}-\exp\paren{-\dfrac{\gamma't}{2}}}
%   \end{pmatrix}
% }
% Decay probabilities
% \eqar{
%   \langle\psi|\psi\rangle=&\frac{\ue^{-\Gamma t}\paren{\gamma^2\cosh\gamma't-\gamma\gamma'\sinh\gamma't-\Omega^2}}{\gamma^2-\Omega^2}\\
%   =&\frac{\ue^{-\Gamma t}\paren{\gamma\paren{\paren{\gamma-\gamma'}\ue^{\gamma't}+\paren{\gamma+\gamma'}\ue^{-\gamma't}}-2\Omega^2}}{2\paren{\gamma^2-\Omega^2}}
% }
% \eqar{
%   \diff{}{t}\langle\psi|\psi\rangle=&\frac{\ue^{-\Gamma t}}{\gamma^2-\Omega^2}\paren{
%     \Gamma\Omega^2+\gamma\gamma'\paren{\Gamma+\gamma}\sinh\gamma't+\gamma\paren{\Omega^2-\gamma^2-\Gamma\gamma}\cosh\gamma't
%   }\\
%   =&\frac{\ue^{-\Gamma t}}{2\paren{\gamma^2-\Omega^2}}\paren{
%     2\Gamma\Omega^2+\gamma\gamma'\paren{\Gamma+\gamma}\paren{\ue^{\gamma't}-\ue^{-\gamma't}}+\gamma\paren{\Omega^2-\gamma^2-\Gamma\gamma}\paren{\ue^{\gamma't}+\ue^{-\gamma't}}
%   }\\
%   =&\frac{\ue^{-\Gamma t}}{2\paren{\gamma^2-\Omega^2}}\paren{
%     2\Gamma\Omega^2
%     -\gamma\paren{\Gamma-\gamma'}\paren{\gamma-\gamma'}\ue^{\gamma't}
%     -\gamma\paren{\Gamma+\gamma'}\paren{\gamma+\gamma'}\ue^{-\gamma't}
%   }
% }
% Decay probability from state 2
% \eqar{
%   &\langle\psi|C_2^\dagger C_2|\psi\rangle\\
%   =&\frac{\paren{\Gamma-\gamma}\Omega^2}{2\paren{\Omega^2-\gamma^2}}\ue^{-\Gamma t}\paren{1-\cosh\gamma't}\\
%   =&\frac{\ue^{-\Gamma t}}{2\paren{\gamma^2-\Omega^2}}\frac{\paren{\Gamma-\gamma}\Omega^2}{2}\paren{\ue^{\gamma't}+\ue^{-p\gamma't}-2}
% }

\end{document}
