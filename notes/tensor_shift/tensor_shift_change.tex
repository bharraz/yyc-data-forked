\documentclass[10pt,fleqn]{article}
% \usepackage[journal=rsc]{chemstyle}
% \usepackage{mhchem}
\usepackage{amsmath}
\usepackage{amssymb}
\usepackage{amsfonts}
\usepackage{esint}
\usepackage{bbm}
\usepackage{amscd}
\usepackage{picinpar}
\usepackage{graphicx}
\usepackage{tikz}
\usepackage{tikz-3dplot}
\usepackage{indentfirst}
\usepackage{wrapfig}
\usepackage{units}
\usepackage{textcomp}
\usepackage[utf8x]{inputenc}
% \usepackage{feyn}
\usepackage{feynmp}
\usepackage{xkeyval}
\usepackage{xargs}
\usepackage{verbatim}
\usepackage{pgfplots}
\usepackage{hyperref}
\usetikzlibrary{
  arrows,
  calc,
  decorations.pathmorphing,
  decorations.pathreplacing,
  decorations.markings,
  fadings,
  positioning,
  shapes
}

\DeclareGraphicsRule{*}{mps}{*}{}
\newcommand{\ud}{\mathrm{d}}
\newcommand{\ue}{\mathrm{e}}
\newcommand{\ui}{\mathrm{i}}
\newcommand{\res}{\mathrm{Res}}
\newcommand{\Tr}{\mathrm{Tr}}
\newcommand{\dsum}{\displaystyle\sum}
\newcommand{\dprod}{\displaystyle\prod}
\newcommand{\dlim}{\displaystyle\lim}
\newcommand{\dint}{\displaystyle\int}
\newcommand{\fsno}[1]{{\!\not\!{#1}}}
\newcommand{\eqar}[1]
{
  \begin{align}
    #1
  \end{align}
}
\newcommand{\texp}[2]{\ensuremath{{#1}\times10^{#2}}}
\newcommand{\dexp}[2]{\ensuremath{{#1}\cdot10^{#2}}}
\newcommand{\eval}[2]{{\left.{#1}\right|_{#2}}}
\newcommand{\paren}[1]{{\left({#1}\right)}}
\newcommand{\lparen}[1]{{\left({#1}\right.}}
\newcommand{\rparen}[1]{{\left.{#1}\right)}}
\newcommand{\abs}[1]{{\left|{#1}\right|}}
\newcommand{\sqr}[1]{{\left[{#1}\right]}}
\newcommand{\crly}[1]{{\left\{{#1}\right\}}}
\newcommand{\angl}[1]{{\left\langle{#1}\right\rangle}}
\newcommand{\tpdiff}[4][{}]{{\paren{\frac{\partial^{#1} {#2}}{\partial {#3}{}^{#1}}}_{#4}}}
\newcommand{\tpsdiff}[4][{}]{{\paren{\frac{\partial^{#1}}{\partial {#3}{}^{#1}}{#2}}_{#4}}}
\newcommand{\pdiff}[3][{}]{{\frac{\partial^{#1} {#2}}{\partial {#3}{}^{#1}}}}
\newcommand{\diff}[3][{}]{{\frac{\ud^{#1} {#2}}{\ud {#3}{}^{#1}}}}
\newcommand{\psdiff}[3][{}]{{\frac{\partial^{#1}}{\partial {#3}{}^{#1}} {#2}}}
\newcommand{\sdiff}[3][{}]{{\frac{\ud^{#1}}{\ud {#3}{}^{#1}} {#2}}}
\newcommand{\tpddiff}[4][{}]{{\left(\dfrac{\partial^{#1} {#2}}{\partial {#3}{}^{#1}}\right)_{#4}}}
\newcommand{\tpsddiff}[4][{}]{{\paren{\dfrac{\partial^{#1}}{\partial {#3}{}^{#1}}{#2}}_{#4}}}
\newcommand{\pddiff}[3][{}]{{\dfrac{\partial^{#1} {#2}}{\partial {#3}{}^{#1}}}}
\newcommand{\ddiff}[3][{}]{{\dfrac{\ud^{#1} {#2}}{\ud {#3}{}^{#1}}}}
\newcommand{\psddiff}[3][{}]{{\frac{\partial^{#1}}{\partial{}^{#1} {#3}} {#2}}}
\newcommand{\sddiff}[3][{}]{{\frac{\ud^{#1}}{\ud {#3}{}^{#1}} {#2}}}
\usepackage{fancyhdr}
\usepackage{multirow}
\usepackage{fontenc}
% \usepackage{tipa}
\usepackage{ulem}
\usepackage{color}
\usepackage{cancel}
\newcommand{\hcancel}[2][black]{\setbox0=\hbox{#2}%
  \rlap{\raisebox{.45\ht0}{\textcolor{#1}{\rule{\wd0}{1pt}}}}#2}
\pagestyle{fancy}
\setlength{\headheight}{67pt}
\fancyhead{}
\fancyfoot{}
\fancyfoot[C]{\thepage}
\fancyhead[R]{}
\renewcommand{\footruleskip}{0pt}
\renewcommand{\headrulewidth}{0.4pt}
\renewcommand{\footrulewidth}{0pt}

\newcommand\pgfmathsinandcos[3]{%
  \pgfmathsetmacro#1{sin(#3)}%
  \pgfmathsetmacro#2{cos(#3)}%
}
\newcommand\LongitudePlane[3][current plane]{%
  \pgfmathsinandcos\sinEl\cosEl{#2} % elevation
  \pgfmathsinandcos\sint\cost{#3} % azimuth
  \tikzset{#1/.estyle={cm={\cost,\sint*\sinEl,0,\cosEl,(0,0)}}}
}
\newcommand\LatitudePlane[3][current plane]{%
  \pgfmathsinandcos\sinEl\cosEl{#2} % elevation
  \pgfmathsinandcos\sint\cost{#3} % latitude
  \pgfmathsetmacro\yshift{\cosEl*\sint}
  \tikzset{#1/.estyle={cm={\cost,0,0,\cost*\sinEl,(0,\yshift)}}} %
}
\newcommand\DrawLongitudeCircle[2][1]{
  \LongitudePlane{\angEl}{#2}
  \tikzset{current plane/.prefix style={scale=#1}}
  % angle of "visibility"
  \pgfmathsetmacro\angVis{atan(sin(#2)*cos(\angEl)/sin(\angEl))} %
  \draw[current plane] (\angVis:1) arc (\angVis:\angVis+180:1);
  \draw[current plane,dashed] (\angVis-180:1) arc (\angVis-180:\angVis:1);
}
\newcommand\DrawLatitudeCircleArrow[2][1]{
  \LatitudePlane{\angEl}{#2}
  \tikzset{current plane/.prefix style={scale=#1}}
  \pgfmathsetmacro\sinVis{sin(#2)/cos(#2)*sin(\angEl)/cos(\angEl)}
  % angle of "visibility"
  \pgfmathsetmacro\angVis{asin(min(1,max(\sinVis,-1)))}
  \draw[current plane,decoration={markings, mark=at position 0.6 with {\arrow{<}}},postaction={decorate},line width=.6mm] (\angVis:1) arc (\angVis:-\angVis-180:1);
  \draw[current plane,dashed,line width=.6mm] (180-\angVis:1) arc (180-\angVis:\angVis:1);
}
\newcommand\DrawLatitudeCircle[2][1]{
  \LatitudePlane{\angEl}{#2}
  \tikzset{current plane/.prefix style={scale=#1}}
  \pgfmathsetmacro\sinVis{sin(#2)/cos(#2)*sin(\angEl)/cos(\angEl)}
  % angle of "visibility"
  \pgfmathsetmacro\angVis{asin(min(1,max(\sinVis,-1)))}
  \draw[current plane] (\angVis:1) arc (\angVis:-\angVis-180:1);
  \draw[current plane,dashed] (180-\angVis:1) arc (180-\angVis:\angVis:1);
}
\newcommand\coil[1]{
  {\rh * cos(\t * pi r)}, {\apart * (2 * #1 + \t) + \rv * sin(\t * pi r)}
}
\makeatletter
\define@key{DrawFromCenter}{style}[{->}]{
  \tikzset{DrawFromCenterPlane/.style={#1}}
}
\define@key{DrawFromCenter}{r}[1]{
  \def\@R{#1}
}
\define@key{DrawFromCenter}{center}[(0, 0)]{
  \def\@Center{#1}
}
\define@key{DrawFromCenter}{theta}[0]{
  \def\@Theta{#1}
}
\define@key{DrawFromCenter}{phi}[0]{
  \def\@Phi{#1}
}
\presetkeys{DrawFromCenter}{style, r, center, theta, phi}{}
\newcommand*\DrawFromCenter[1][]{
  \setkeys{DrawFromCenter}{#1}{
    \pgfmathsinandcos\sint\cost{\@Theta}
    \pgfmathsinandcos\sinp\cosp{\@Phi}
    \pgfmathsinandcos\sinA\cosA{\angEl}
    \pgfmathsetmacro\DX{\@R*\cost*\cosp}
    \pgfmathsetmacro\DY{\@R*(\cost*\sinp*\sinA+\sint*\cosA)}
    \draw[DrawFromCenterPlane] \@Center -- ++(\DX, \DY);
  }
}
\newcommand*\DrawFromCenterText[2][]{
  \setkeys{DrawFromCenter}{#1}{
    \pgfmathsinandcos\sint\cost{\@Theta}
    \pgfmathsinandcos\sinp\cosp{\@Phi}
    \pgfmathsinandcos\sinA\cosA{\angEl}
    \pgfmathsetmacro\DX{\@R*\cost*\cosp}
    \pgfmathsetmacro\DY{\@R*(\cost*\sinp*\sinA+\sint*\cosA)}
    \draw[DrawFromCenterPlane] \@Center -- ++(\DX, \DY) node {#2};
  }
}
\makeatother
\tikzstyle{snakearrow} = [decorate, decoration={pre length=0.2cm,
  post length=0.2cm, snake, amplitude=.4mm,
  segment length=2mm},thick, ->]
%% document-wide tikz options and styles
\tikzset{%
  >=latex, % option for nice arrows
  inner sep=0pt,%
  outer sep=2pt,%
  mark coordinate/.style={inner sep=0pt,outer sep=0pt,minimum size=3pt,
    fill=black,circle}%
}
\addtolength{\hoffset}{-1.3cm}
\addtolength{\voffset}{-2cm}
\addtolength{\textwidth}{3cm}
\addtolength{\textheight}{2.5cm}
\renewcommand{\footskip}{10pt}
\setlength{\headwidth}{\textwidth}
\setlength{\headsep}{20pt}
\setlength{\marginparwidth}{0pt}
\parindent=0pt
\title{Effect of a slight change in polarization on light shift}

\ifpdf
  % Ensure reproducible output
  \pdfinfoomitdate=1
  \pdfsuppressptexinfo=-1
  \pdftrailerid{}
  \hypersetup{
    pdfcreator={},
    pdfproducer={}
  }
\fi

\begin{document}

\maketitle

\section{Goal}
Vector and tensor light shift from an optical tweezer can cause coupling
between different angular momentum states that were not allowed without it.
This effect can be different if the tweezer polarization changes between site.\\

While it's fairly easy to control when there's only vector light shift,
just use linear polarization, the effect can be much more complex
when tensor light shift terms. It's of course always possible to turn off
all of the coupling with perfect $\pi$ polarization (since it doesn't break
rotational symmetry around $z$ and therefore doesn't couple different $m_F$ states)
but non-$\pi$ linear polarization could still have nontrivial
effects in the experiment.\\

Questions to answer,
\begin{enumerate}
\item Can the solution for vector light shift, i.e. using pure linear polarization,
  eliminate all the unwanted coupling in the presence of
  tensor light shift~(section~\ref{lin-pol}).
\item For a single tweezer, can the effect from different source,
  e.g. circularity of polarization, angle of polarization, angle of B field,
  cancel out each other and to what level,
  i.e. is it possible to achieve ``optimal'' result
  without $\pi$ polarization~(section~\ref{general-pol}).
\item If yes, is it possible to have local optimal
  (e.g. cancellation of some terms at a condition that doesn't cancel all terms),
  and if that's the case, what is the best way to align
  everything~(section~\ref{general-pol}).
\item In general, what is the condition for the coupling to the wrong state
  being insensitive to certain type of change in the polarization.
  E.g. is there a magic certain polarization that is at least first order insensitive
  to arbitrary small angle rotation
  (i.e. linear polarization when there's only vector shift)~(section~\ref{summary}).
\end{enumerate}

We'll mainly (only) consider the state mixing for a single state.
This is fine since the system we want to study here, $^{171}\mathrm{Yb}$ neutral,
has a $J=0, I=1/2$ ground state that should have a very small (if any) vector
shift. Any existing vector shift should also be factored
into a slightly tweaked excited state vector shift (at least to the first order).\\

We'll also mostly consider the coupling for spontaneous emission.
For darkness of state during OP, one could almost always adjust the OP polarization
to cancel out any unwanted coupling. With the ground state
setting the quantization axis, the presence of unwanted coupling/decay path
is simply proportional to any non-diagonal element in the Hamiltonian.\\

\section{Linear polarization}
\label{lin-pol}

Assuming the polarization of the tweezer is linear and selecting the quantization axis
for the ground state, we'd like to see what coupling could be
caused by a linear polarization that is not pure-$\pi$. As mentioned above,
this can only be caused by the tensor term.
Tensor shift hamiltonian ignoring constant terms that causes no coupling,
\eqar{
  H_t\propto&\crly{\vec{u}^*\cdot\vec F,\vec{u}\cdot\vec F}
}

For linear polarization $\vec{u}=\vec{u}^*$
\eqar{
  H_t\propto&2\paren{\vec{u}\cdot\vec F}^2
}

If the polarization is very closed to $\pi$ but off by a small angle $\theta$
in the $x$ direction (can also be any other direction in the $x$-$y$ plane
without loss of generality)
\eqar{
  \vec{u}=&\cos\theta \hat z + \sin\theta \hat x\\
  \begin{split}
    H_t\propto&2\paren{\cos\theta F_z+\sin\theta F_x}^2\\
    =&2\paren{\cos^2\theta F_z^2+\sin\theta\cos\theta\paren{F_zF_x+F_xF_z}+\sin^2\theta F_x^2}\\
    \approx&2\paren{\cos^2\theta F_z^2+\frac{1}{2}\sin\theta\cos\theta\paren{F_zF_++F_zF_-+F_+F_z+F_-F_z}}\\
    =&2\paren{\cos^2\theta m_F^2+\frac{1}{2}\sin\theta\cos\theta\paren{\paren{2m_F+1}F_++\paren{2m_F-1}F_-}}\\
    =&2\cos^2\theta m_F^2+\sin\theta\cos\theta\paren{\paren{2m_F+1}F_++\paren{2m_F-1}F_-}
  \end{split}
}
Here we've ignored the second order term (in $\theta$)
and calculated the matrix element in the $z$ basis.\\

The result shows that the tensor light shift can cause a first order coupling between
the neighboring $m_F$ state when the polarization is not $\pi$
even if it remains linear. This is different from vector light shift
which only have non-trivial effect for circular polarization.\\

\section{General polarization}
\label{general-pol}
Tensor term for generic polarization,
\eqar{
  \begin{split}
    H_t\propto&\paren{u_x^*F_x+u_y^*F_y+u_z^*F_z}\paren{u_xF_x+u_yF_y+u_zF_z}+\paren{u_xF_x+u_yF_y+u_zF_z}\paren{u_x^*F_x+u_y^*F_y+u_z^*F_z}\\
    =&2\abs{u_x}^2F_x^2+2\abs{u_y}^2F_y^2+2\abs{u_z}^2F_z^2\\
              &+2\mathrm{Re}\paren{u_xu_y^*}\paren{F_xF_y+F_yF_x}
                +2\mathrm{Re}\paren{u_xu_z^*}\paren{F_xF_z+F_zF_x}
                +2\mathrm{Re}\paren{u_yu_z^*}\paren{F_yF_z+F_zF_y}\\
    =&\abs{u_x}^2\frac{\paren{F_++F_-}^2}{2}-\abs{u_y}^2\frac{\paren{F_+-F_-}^2}{2}+2\abs{u_z}^2F_z^2\\
              &+\frac{\mathrm{Re}\paren{u_xu_y^*}}{2\ui}\paren{\paren{F_++F_-}\paren{F_+-F_-}+\paren{F_+-F_-}\paren{F_++F_-}}\\
              &+\mathrm{Re}\paren{u_xu_z^*}\paren{\paren{F_++F_-}F_z+F_z\paren{F_++F_-}}
                -\ui\mathrm{Re}\paren{u_yu_z^*}\paren{\paren{F_+-F_-}F_z+F_z\paren{F_+-F_-}}\\
    =&\frac{\paren{\abs{u_x}^2-\abs{u_y}^2}\paren{F_-^2+F_+^2}
       +\paren{\abs{u_x}^2+\abs{u_y}^2}\paren{F_+F_-+F_-F_+}}{2}+2\abs{u_z}^2F_z^2
       +\frac{\mathrm{Re}\paren{u_xu_y^*}}{\ui}\paren{F_+^2+F_-^2}\\
              &+\mathrm{Re}\paren{u_xu_z^*}\paren{\paren{F_++F_-}F_z+F_z\paren{F_++F_-}}
                -\ui\mathrm{Re}\paren{u_yu_z^*}\paren{\paren{F_+-F_-}F_z+F_z\paren{F_+-F_-}}\\
  \end{split}
}
Ignoring all the diagonal terms in $z$ basis calculate the matrix element,
\eqar{
  \begin{split}
    H_t'=&\paren{\frac{\abs{u_x}^2-\abs{u_y}^2}{2}
           -\ui\mathrm{Re}\paren{u_xu_y^*}}\paren{F_+^2+F_-^2}\\
         &+\paren{\mathrm{Re}\paren{u_xu_z^*}-\ui\mathrm{Re}\paren{u_yu_z^*}}\paren{2m_F+1}F_+
           +\paren{\mathrm{Re}\paren{u_xu_z^*}+\ui\mathrm{Re}\paren{u_yu_z^*}}\paren{2m_F-1}F_-\\
  \end{split}
}
We can see that unless $\dfrac{\abs{u_x}^2-\abs{u_y}^2}{2}$,
$\mathrm{Re}\paren{u_xu_y^*}$, $\mathrm{Re}\paren{u_xu_z^*}$ and
$\mathrm{Re}\paren{u_yu_z^*}$ are all zero the effect of
the tensor shift will not be zero (which requires $u_x=u_y=0$
or pure $\pi$ polarization).\\

Moreover, since the term that contains a single $F_+$ or $F_-$
contains $m_F$ dependent pre-factor, they cannot be fully cancelled
by the vector term which only contains terms that are proportional to $F_+$ or $F_-$
without $m_F$ dependent pre-factor.
It is possible, however, for the effect to be cancelled for the stretched state
since they are coupled by only a single $F_\pm$ term and the coefficient can
in principle be cancelled by the vector term. For this to work in the experiment,
the second order (i.e. $F_+^2$ and $F_-^2$) terms from the tensor effect
still need to be zero (the vector term cannot cancel them out).
This requires both $\dfrac{\abs{u_x}^2-\abs{u_y}^2}{2}$ and
$\mathrm{Re}\paren{u_xu_y^*}$ to be $0$, or in another word $u_x=\pm\ui u_y$.\\

The vector term
\eqar{
  \begin{split}
    H_v\propto&\ui\paren{\vec{u}\times\vec{u}^*}\cdot\vec F\\
    =&2\mathrm{Im}\paren{u_y^*u_z}F_x
       +2\mathrm{Im}\paren{u_z^*u_x}F_y
       +2\mathrm{Im}\paren{u_x^*u_y}F_z\\
    =&\mathrm{Im}\paren{u_y^*u_z}\paren{F_++F_-}
       -\ui\mathrm{Im}\paren{u_z^*u_x}\paren{F_+-F_-}
       +2\mathrm{Im}\paren{u_x^*u_y}F_z\\
    =&\paren{\mathrm{Im}\paren{u_y^*u_z}-\ui\mathrm{Im}\paren{u_z^*u_x}}F_+
       +\paren{\mathrm{Im}\paren{u_y^*u_z}+\ui\mathrm{Im}\paren{u_z^*u_x}}F_-
       +2\mathrm{Im}\paren{u_x^*u_y}F_z
  \end{split}
}

Assuming $u_x=\pm\ui u_y$, define $\mathcal{R}\equiv\mathrm{Re}\paren{u_xu_z^*}$,
$\mathcal{I}\equiv\mathrm{Im}\paren{u_xu_z^*}$
\eqar{
  \mathrm{Re}\paren{u_yu_z^*}=&\pm\mathcal{I}\\
  \mathrm{Im}\paren{u_y^*u_z}=&\pm\mathcal{R}
}

Off diagonal terms,
\eqar{
  \begin{split}
    H_v'=&\pm\paren{\mathcal{R}\mp\ui\mathcal{I}}F_+
       \pm\paren{\mathcal{R}\pm\ui\mathcal{I}}F_-
  \end{split}\\
  \begin{split}
    H_t'=&\paren{\mathcal{R}\mp\ui\mathcal{I}}\paren{2m_F+1}F_+
           +\paren{\mathcal{R}\pm\ui\mathcal{I}}\paren{2m_F-1}F_-
  \end{split}
}

The coefficient in front of the $F_+$ and $F_-$ terms are the same
(or at least have the same polarization dependency).
This means that the effect of the two cannot be cancelled out
by adjusting the polarization (of the tweezer) but may be cancelled out
by selecting the right wavelength. It also means that unless the wavelength is magic
(for this particular coupling), the only good coupling for a single tweezer
is $\pi$ and there won't be any local optimal with non-linear polarization.

\section{Summary}
\label{summary}
In general, unless the ratio of vector and tensor polarizability
is some particular value, the only good polarization
(one that does not couple different $m_F$ states) is $\pi$
(linear along quantization axis). This means that
\begin{enumerate}
\item Just using linear polarization isn't good enough,
  the direction of the polarization also matters.
\item When aligning the polarization, there isn't any local optimal to get stuck on.
  If the coupling between $m_F$ is nulled, the polarization would have to be $\pi$.
\item Since the only good polarization is $\pi$,
  there isn't a polarization that is insensitive to any rotation.
  Only rotate it along $z$ is allowed (since it doesn't change the polarization...).
\end{enumerate}

\end{document}
