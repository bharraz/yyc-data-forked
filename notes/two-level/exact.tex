\documentclass[10pt,fleqn]{article}
% \usepackage[journal=rsc]{chemstyle}
% \usepackage{mhchem}
\usepackage{amsmath}
\usepackage{amssymb}
\usepackage{amsfonts}
\usepackage{esint}
\usepackage{bbm}
\usepackage{amscd}
\usepackage{picinpar}
\usepackage{graphicx}
\usepackage{tikz}
\usepackage{tikz-3dplot}
\usepackage{indentfirst}
\usepackage{wrapfig}
\usepackage{units}
\usepackage{textcomp}
\usepackage[utf8x]{inputenc}
\usepackage{xkeyval}
\usepackage{xargs}
\usepackage{verbatim}
\usepackage{pgfplots}
\usepackage{hyperref}
\usetikzlibrary{
  arrows,
  calc,
  decorations.pathmorphing,
  decorations.pathreplacing,
  decorations.markings,
  fadings,
  positioning,
  shapes
}

\DeclareGraphicsRule{*}{mps}{*}{}
\newcommand{\ud}{\mathrm{d}}
\newcommand{\ue}{\mathrm{e}}
\newcommand{\ui}{\mathrm{i}}
\newcommand{\res}{\mathrm{Res}}
\newcommand{\Tr}{\mathrm{Tr}}
\newcommand{\dsum}{\displaystyle\sum}
\newcommand{\dprod}{\displaystyle\prod}
\newcommand{\dlim}{\displaystyle\lim}
\newcommand{\dint}{\displaystyle\int}
\newcommand{\fsno}[1]{{\!\not\!{#1}}}
\newcommand{\eqar}[1]
{
  \begin{align*}
    #1
  \end{align*}
}
\newcommand{\texp}[2]{\ensuremath{{#1}\times10^{#2}}}
\newcommand{\dexp}[2]{\ensuremath{{#1}\cdot10^{#2}}}
\newcommand{\eval}[2]{{\left.{#1}\right|_{#2}}}
\newcommand{\paren}[1]{{\left({#1}\right)}}
\newcommand{\lparen}[1]{{\left({#1}\right.}}
\newcommand{\rparen}[1]{{\left.{#1}\right)}}
\newcommand{\abs}[1]{{\left|{#1}\right|}}
\newcommand{\sqr}[1]{{\left[{#1}\right]}}
\newcommand{\crly}[1]{{\left\{{#1}\right\}}}
\newcommand{\angl}[1]{{\left\langle{#1}\right\rangle}}
\newcommand{\tpdiff}[4][{}]{{\paren{\frac{\partial^{#1} {#2}}{\partial {#3}{}^{#1}}}_{#4}}}
\newcommand{\tpsdiff}[4][{}]{{\paren{\frac{\partial^{#1}}{\partial {#3}{}^{#1}}{#2}}_{#4}}}
\newcommand{\pdiff}[3][{}]{{\frac{\partial^{#1} {#2}}{\partial {#3}{}^{#1}}}}
\newcommand{\diff}[3][{}]{{\frac{\ud^{#1} {#2}}{\ud {#3}{}^{#1}}}}
\newcommand{\psdiff}[3][{}]{{\frac{\partial^{#1}}{\partial {#3}{}^{#1}} {#2}}}
\newcommand{\sdiff}[3][{}]{{\frac{\ud^{#1}}{\ud {#3}{}^{#1}} {#2}}}
\newcommand{\tpddiff}[4][{}]{{\left(\dfrac{\partial^{#1} {#2}}{\partial {#3}{}^{#1}}\right)_{#4}}}
\newcommand{\tpsddiff}[4][{}]{{\paren{\dfrac{\partial^{#1}}{\partial {#3}{}^{#1}}{#2}}_{#4}}}
\newcommand{\pddiff}[3][{}]{{\dfrac{\partial^{#1} {#2}}{\partial {#3}{}^{#1}}}}
\newcommand{\ddiff}[3][{}]{{\dfrac{\ud^{#1} {#2}}{\ud {#3}{}^{#1}}}}
\newcommand{\psddiff}[3][{}]{{\frac{\partial^{#1}}{\partial{}^{#1} {#3}} {#2}}}
\newcommand{\sddiff}[3][{}]{{\frac{\ud^{#1}}{\ud {#3}{}^{#1}} {#2}}}
\usepackage{fancyhdr}
\usepackage{multirow}
\usepackage{fontenc}
% \usepackage{tipa}
\usepackage{ulem}
\usepackage{color}
\usepackage{cancel}
\newcommand{\hcancel}[2][black]{\setbox0=\hbox{#2}%
  \rlap{\raisebox{.45\ht0}{\textcolor{#1}{\rule{\wd0}{1pt}}}}#2}
\pagestyle{fancy}
\setlength{\headheight}{67pt}
\fancyhead{}
\fancyfoot{}
\fancyfoot[C]{\thepage}
\fancyhead[R]{}
\renewcommand{\footruleskip}{0pt}
\renewcommand{\headrulewidth}{0.4pt}
\renewcommand{\footrulewidth}{0pt}

\newcommand\pgfmathsinandcos[3]{%
  \pgfmathsetmacro#1{sin(#3)}%
  \pgfmathsetmacro#2{cos(#3)}%
}
\newcommand\LongitudePlane[3][current plane]{%
  \pgfmathsinandcos\sinEl\cosEl{#2} % elevation
  \pgfmathsinandcos\sint\cost{#3} % azimuth
  \tikzset{#1/.estyle={cm={\cost,\sint*\sinEl,0,\cosEl,(0,0)}}}
}
\newcommand\LatitudePlane[3][current plane]{%
  \pgfmathsinandcos\sinEl\cosEl{#2} % elevation
  \pgfmathsinandcos\sint\cost{#3} % latitude
  \pgfmathsetmacro\yshift{\cosEl*\sint}
  \tikzset{#1/.estyle={cm={\cost,0,0,\cost*\sinEl,(0,\yshift)}}} %
}
\newcommand\DrawLongitudeCircle[2][1]{
  \LongitudePlane{\angEl}{#2}
  \tikzset{current plane/.prefix style={scale=#1}}
  % angle of "visibility"
  \pgfmathsetmacro\angVis{atan(sin(#2)*cos(\angEl)/sin(\angEl))} %
  \draw[current plane] (\angVis:1) arc (\angVis:\angVis+180:1);
  \draw[current plane,dashed] (\angVis-180:1) arc (\angVis-180:\angVis:1);
}
\newcommand\DrawLatitudeCircleArrow[2][1]{
  \LatitudePlane{\angEl}{#2}
  \tikzset{current plane/.prefix style={scale=#1}}
  \pgfmathsetmacro\sinVis{sin(#2)/cos(#2)*sin(\angEl)/cos(\angEl)}
  % angle of "visibility"
  \pgfmathsetmacro\angVis{asin(min(1,max(\sinVis,-1)))}
  \draw[current plane,decoration={markings, mark=at position 0.6 with {\arrow{<}}},postaction={decorate},line width=.6mm] (\angVis:1) arc (\angVis:-\angVis-180:1);
  \draw[current plane,dashed,line width=.6mm] (180-\angVis:1) arc (180-\angVis:\angVis:1);
}
\newcommand\DrawLatitudeCircle[2][1]{
  \LatitudePlane{\angEl}{#2}
  \tikzset{current plane/.prefix style={scale=#1}}
  \pgfmathsetmacro\sinVis{sin(#2)/cos(#2)*sin(\angEl)/cos(\angEl)}
  % angle of "visibility"
  \pgfmathsetmacro\angVis{asin(min(1,max(\sinVis,-1)))}
  \draw[current plane] (\angVis:1) arc (\angVis:-\angVis-180:1);
  \draw[current plane,dashed] (180-\angVis:1) arc (180-\angVis:\angVis:1);
}
\newcommand\coil[1]{
  {\rh * cos(\t * pi r)}, {\apart * (2 * #1 + \t) + \rv * sin(\t * pi r)}
}
\makeatletter
\define@key{DrawFromCenter}{style}[{->}]{
  \tikzset{DrawFromCenterPlane/.style={#1}}
}
\define@key{DrawFromCenter}{r}[1]{
  \def\@R{#1}
}
\define@key{DrawFromCenter}{center}[(0, 0)]{
  \def\@Center{#1}
}
\define@key{DrawFromCenter}{theta}[0]{
  \def\@Theta{#1}
}
\define@key{DrawFromCenter}{phi}[0]{
  \def\@Phi{#1}
}
\presetkeys{DrawFromCenter}{style, r, center, theta, phi}{}
\newcommand*\DrawFromCenter[1][]{
  \setkeys{DrawFromCenter}{#1}{
    \pgfmathsinandcos\sint\cost{\@Theta}
    \pgfmathsinandcos\sinp\cosp{\@Phi}
    \pgfmathsinandcos\sinA\cosA{\angEl}
    \pgfmathsetmacro\DX{\@R*\cost*\cosp}
    \pgfmathsetmacro\DY{\@R*(\cost*\sinp*\sinA+\sint*\cosA)}
    \draw[DrawFromCenterPlane] \@Center -- ++(\DX, \DY);
  }
}
\newcommand*\DrawFromCenterText[2][]{
  \setkeys{DrawFromCenter}{#1}{
    \pgfmathsinandcos\sint\cost{\@Theta}
    \pgfmathsinandcos\sinp\cosp{\@Phi}
    \pgfmathsinandcos\sinA\cosA{\angEl}
    \pgfmathsetmacro\DX{\@R*\cost*\cosp}
    \pgfmathsetmacro\DY{\@R*(\cost*\sinp*\sinA+\sint*\cosA)}
    \draw[DrawFromCenterPlane] \@Center -- ++(\DX, \DY) node {#2};
  }
}
\makeatother
\tikzstyle{snakearrow} = [decorate, decoration={pre length=0.2cm,
  post length=0.2cm, snake, amplitude=.4mm,
  segment length=2mm},thick, ->]
%% document-wide tikz options and styles
\tikzset{%
  >=latex, % option for nice arrows
  inner sep=0pt,%
  outer sep=2pt,%
  mark coordinate/.style={inner sep=0pt,outer sep=0pt,minimum size=3pt,
    fill=black,circle}%
}
\addtolength{\hoffset}{-1.3cm}
\addtolength{\voffset}{-2cm}
\addtolength{\textwidth}{3cm}
\addtolength{\textheight}{2.5cm}
\renewcommand{\footskip}{10pt}
\setlength{\headwidth}{\textwidth}
\setlength{\headsep}{20pt}
\setlength{\marginparwidth}{0pt}
\parindent=0pt
\title{Exact formula for two level system drive with fixed detunine and amplitude}

\ifpdf
  % Ensure reproducible output
  \pdfinfoomitdate=1
  \pdfsuppressptexinfo=-1
  \pdftrailerid{}
  \hypersetup{
    pdfcreator={},
    pdfproducer={}
  }
\fi

\begin{document}

\maketitle

Original Hamiltonian
\eqar{
  H_0=&\frac12\begin{pmatrix}
    \Delta&\Omega\ue^{\ui\paren{\delta t + \phi}}\\
    \Omega\ue^{-\ui\paren{\delta t + \phi}}&-\Delta
  \end{pmatrix}
}
Under a frame rotation
\eqar{
  U_1=&\begin{pmatrix}
    \ue^{-\ui\paren{\delta t + \phi}/2}&0\\
    0&\ue^{\ui\paren{\delta t + \phi}/2}
  \end{pmatrix}
}
The Hamiltonian becomes
\eqar{
  H_1=&U_1HU_1^\dagger+\ui\diff{U_1}{t}U_1^\dagger\\
  =&\frac12\begin{pmatrix}
    \Delta'&\Omega\\
    \Omega&-\Delta'
  \end{pmatrix}
}
where $\Delta'\equiv\Delta+\delta$. We can diagonalize it with
\eqar{
  U_2=&\begin{pmatrix}
    -\sqrt{\dfrac12-\dfrac{\Delta'}{2\Omega'}}&\sqrt{\dfrac12+\dfrac{\Delta'}{2\Omega'}}\\
    \sqrt{\dfrac12+\dfrac{\Delta'}{2\Omega'}}&\sqrt{\dfrac12-\dfrac{\Delta'}{2\Omega'}}
  \end{pmatrix}
}
where $\Omega'\equiv\sqrt{\Delta'^2+\Omega^2}$, to,
\eqar{
  H_2=&U_2H_1U_2^\dagger\\
  =&\begin{pmatrix}
    -\dfrac{\Omega'}{2}&0\\
    0&\dfrac{\Omega'}{2}
  \end{pmatrix}
}

Evolution under $H_2$,
\eqar{
  T_2=&\exp\paren{-\ui H_2t}\\
  =&\begin{pmatrix}
    \exp\paren{\ui\dfrac{\Omega'}{2}t}&0\\
    0&\exp\paren{-\ui\dfrac{\Omega'}{2}t}
  \end{pmatrix}
}
Evolution under $H_1$,
\eqar{
  &T_1\\
  =&U_2^\dagger T_2 U_2\\
  =&\begin{pmatrix}
    -\sqrt{\dfrac12-\dfrac{\Delta'}{2\Omega'}}\exp\paren{\ui\dfrac{\Omega'}{2}t}&\sqrt{\dfrac12+\dfrac{\Delta'}{2\Omega'}}\exp\paren{-\ui\dfrac{\Omega'}{2}t}\\
    \sqrt{\dfrac12+\dfrac{\Delta'}{2\Omega'}}\exp\paren{\ui\dfrac{\Omega'}{2}t}&\sqrt{\dfrac12-\dfrac{\Delta'}{2\Omega'}}\exp\paren{-\ui\dfrac{\Omega'}{2}t}
  \end{pmatrix}
  \begin{pmatrix}
    -\sqrt{\dfrac12-\dfrac{\Delta'}{2\Omega'}}&\sqrt{\dfrac12+\dfrac{\Delta'}{2\Omega'}}\\
    \sqrt{\dfrac12+\dfrac{\Delta'}{2\Omega'}}&\sqrt{\dfrac12-\dfrac{\Delta'}{2\Omega'}}
  \end{pmatrix}\\
  =&\begin{pmatrix}
    \paren{\dfrac12-\dfrac{\Delta'}{2\Omega'}}\exp\paren{\ui\dfrac{\Omega'}{2}t}+\paren{\dfrac12+\dfrac{\Delta'}{2\Omega'}}\exp\paren{-\ui\dfrac{\Omega'}{2}t}&
                                                                                                                                                                \dfrac{\Omega}{2\Omega'}\paren{\exp\paren{-\ui\dfrac{\Omega'}{2}t}-\exp\paren{\ui\dfrac{\Omega'}{2}t}}\\
    \dfrac{\Omega}{2\Omega'}\paren{\exp\paren{-\ui\dfrac{\Omega'}{2}t}-\exp\paren{\ui\dfrac{\Omega'}{2}t}}&\paren{\dfrac12+\dfrac{\Delta'}{2\Omega'}}\exp\paren{\ui\dfrac{\Omega'}{2}t}+\paren{\dfrac12-\dfrac{\Delta'}{2\Omega'}}\exp\paren{-\ui\dfrac{\Omega'}{2}t}
  \end{pmatrix}\\
  =&\begin{pmatrix}
    \cos\paren{\dfrac{\Omega'}{2}t}-\ui\dfrac{\Delta'}{\Omega'}\sin\paren{\dfrac{\Omega'}{2}t}&
                                                                                                -\ui\dfrac{\Omega}{\Omega'}\sin\paren{\dfrac{\Omega'}{2}t}\\
    -\ui\dfrac{\Omega}{\Omega'}\sin\paren{\dfrac{\Omega'}{2}t}&\cos\paren{\dfrac{\Omega'}{2}t}+\ui\dfrac{\Delta'}{\Omega'}\sin\paren{\dfrac{\Omega'}{2}t}
  \end{pmatrix}
}
Original time evolution
\eqar{
  &T_0\\
  =&U_1^\dagger T_1 U_1(t=0)\\
  =&\begin{pmatrix}
    \paren{\cos\paren{\dfrac{\Omega'}{2}t}-\ui\dfrac{\Delta'}{\Omega'}\sin\paren{\dfrac{\Omega'}{2}t}}\ue^{\ui\delta t/2}
    &-\ui\dfrac{\Omega}{\Omega'}\sin\paren{\dfrac{\Omega'}{2}t}\ue^{\ui\paren{\delta t/2 + \phi}}\\
    -\ui\dfrac{\Omega}{\Omega'}\sin\paren{\dfrac{\Omega'}{2}t}\ue^{-\ui\paren{\delta t/2 + \phi}}
    &\paren{\cos\paren{\dfrac{\Omega'}{2}t}+\ui\dfrac{\Delta'}{\Omega'}\sin\paren{\dfrac{\Omega'}{2}t}}\ue^{-\ui\delta t/2}
  \end{pmatrix}
}

If $\Delta=0$,
\eqar{
  T_0=&\begin{pmatrix}
    \paren{\cos\paren{\dfrac{\Omega'}{2}t}-\ui\dfrac{\delta}{\Omega'}\sin\paren{\dfrac{\Omega'}{2}t}}\ue^{\ui\delta t/2}
    &-\ui\dfrac{\Omega}{\Omega'}\sin\paren{\dfrac{\Omega'}{2}t}\ue^{\ui\paren{\delta t/2 + \phi}}\\
    -\ui\dfrac{\Omega}{\Omega'}\sin\paren{\dfrac{\Omega'}{2}t}\ue^{-\ui\paren{\delta t/2 + \phi}}
    &\paren{\cos\paren{\dfrac{\Omega'}{2}t}+\ui\dfrac{\delta}{\Omega'}\sin\paren{\dfrac{\Omega'}{2}t}}\ue^{-\ui\delta t/2}
  \end{pmatrix}
}
Now converting all the parameter to angles,
\eqar{
  \theta\equiv&\Omega t\\
  \alpha\equiv&\delta t\\
  \theta'\equiv&\sqrt{\theta^2+\alpha^2}
}
\eqar{
  T_0=&\begin{pmatrix}
    \paren{\cos\paren{\dfrac{\theta'}{2}}-\ui\dfrac{\alpha}{\theta'}\sin\paren{\dfrac{\theta'}{2}}}\ue^{\ui\alpha/2}
    &-\ui\dfrac{\theta}{\theta'}\sin\paren{\dfrac{\theta'}{2}}\ue^{\ui\paren{\alpha/2 + \phi}}\\
    -\ui\dfrac{\theta}{\theta'}\sin\paren{\dfrac{\theta'}{2}}\ue^{-\ui\paren{\alpha/2 + \phi}}
    &\paren{\cos\paren{\dfrac{\theta'}{2}}+\ui\dfrac{\alpha}{\theta'}\sin\paren{\dfrac{\theta'}{2}}}\ue^{-\ui\alpha/2}
  \end{pmatrix}\\
  =&\begin{pmatrix}
    \paren{\cos\paren{\dfrac{\theta'}{2}}-\ui\dfrac{\alpha}{2}\mathrm{sinc}\paren{\dfrac{\theta'}{2}}}\ue^{\ui\alpha/2}
    &-\ui\dfrac{\theta}{2}\mathrm{sinc}\paren{\dfrac{\theta'}{2}}\ue^{\ui\paren{\alpha/2 + \phi}}\\
    -\ui\dfrac{\theta}{2}\mathrm{sinc}\paren{\dfrac{\theta'}{2}}\ue^{-\ui\paren{\alpha/2 + \phi}}
    &\paren{\cos\paren{\dfrac{\theta'}{2}}+\ui\dfrac{\alpha}{2}\mathrm{sinc}\paren{\dfrac{\theta'}{2}}}\ue^{-\ui\alpha/2}
  \end{pmatrix}
}

\section*{Check}
Time derivative of $T_1$
\eqar{
  \ui\diff{T_1}{t}T_1^\dagger=&\ui\frac{\Omega'}{2}\begin{pmatrix}
    -\sin\paren{\dfrac{\Omega'}{2}t}-\ui\dfrac{\Delta'}{\Omega'}\cos\paren{\dfrac{\Omega'}{2}t}&-\ui\dfrac{\Omega}{\Omega'}\cos\paren{\dfrac{\Omega'}{2}t}\\
    -\ui\dfrac{\Omega}{\Omega'}\cos\paren{\dfrac{\Omega'}{2}t}&-\sin\paren{\dfrac{\Omega'}{2}t}+\ui\dfrac{\Delta'}{\Omega'}\cos\paren{\dfrac{\Omega'}{2}t}
  \end{pmatrix}\\
  &\begin{pmatrix}
    \cos\paren{\dfrac{\Omega'}{2}t}+\ui\dfrac{\Delta'}{\Omega'}\sin\paren{\dfrac{\Omega'}{2}t}&\ui\dfrac{\Omega}{\Omega'}\sin\paren{\dfrac{\Omega'}{2}t}\\
    \ui\dfrac{\Omega}{\Omega'}\sin\paren{\dfrac{\Omega'}{2}t}&\cos\paren{\dfrac{\Omega'}{2}t}-\ui\dfrac{\Delta'}{\Omega'}\sin\paren{\dfrac{\Omega'}{2}t}
  \end{pmatrix}\\
  =&\ui\frac{\Omega'}{2}\begin{pmatrix}
    -\ui\dfrac{\Delta'}{\Omega'}&-\ui\dfrac{\Omega}{\Omega'}\\
    -\ui\dfrac{\Omega}{\Omega'}&\ui\dfrac{\Delta'}{\Omega'}
  \end{pmatrix}\\
  =&\frac{1}{2}\begin{pmatrix}
    \Delta'&\Omega\\
    \Omega&-\Delta'
  \end{pmatrix}\\
  =&H_1
}

% Time derivative of $T_0$
% \eqar{
%   &\ui\diff{T_0}{t}T_0^\dagger\\
%   =&\begin{pmatrix}
%     -\dfrac{\Omega'}{2}\sin\paren{\dfrac{\Omega'}{2}t}-\ui\dfrac{\Delta'}{2}\cos\paren{\dfrac{\Omega'}{2}t}&\paren{\dfrac{\Omega\delta}{\Omega'}\sin\paren{\dfrac{\Omega'}{2}t}-\ui\dfrac{\Omega}{2}\cos\paren{\dfrac{\Omega'}{2}t}}\ue^{\ui\paren{\delta t + \phi}}\\
%     \paren{-\dfrac{\Omega\delta}{\Omega'}\sin\paren{\dfrac{\Omega'}{2}t}-\ui\dfrac{\Omega}{2}\cos\paren{\dfrac{\Omega'}{2}t}}\ue^{-\ui\paren{\delta t + \phi}}&-\dfrac{\Omega'}{2}\sin\paren{\dfrac{\Omega'}{2}t}+\ui\dfrac{\Delta'}{2}\cos\paren{\dfrac{\Omega'}{2}t}
%   \end{pmatrix}\\
%   &\begin{pmatrix}
%     \cos\paren{\dfrac{\Omega'}{2}t}+\ui\dfrac{\Delta'}{\Omega'}\sin\paren{\dfrac{\Omega'}{2}t}
%     &\ui\dfrac{\Omega}{\Omega'}\sin\paren{\dfrac{\Omega'}{2}t}\ue^{\ui\paren{\delta t + \phi}}\\
%     \ui\dfrac{\Omega}{\Omega'}\sin\paren{\dfrac{\Omega'}{2}t}\ue^{-\ui\paren{\delta t + \phi}}
%     &\cos\paren{\dfrac{\Omega'}{2}t}-\ui\dfrac{\Delta'}{\Omega'}\sin\paren{\dfrac{\Omega'}{2}t}
%   \end{pmatrix}\\
%   &-\ui\dfrac{\Delta'}{2}\\
%   &+\ui\dfrac{\Omega^2\delta}{\Omega'^2}\sin^2\paren{\dfrac{\Omega'}{2}t}
% }

\end{document}
