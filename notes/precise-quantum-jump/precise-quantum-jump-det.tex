\documentclass[10pt,fleqn]{article}
% \usepackage[journal=rsc]{chemstyle}
% \usepackage{mhchem}
\usepackage{amsmath}
\usepackage{amssymb}
\usepackage{amsfonts}
\usepackage{esint}
\usepackage{bbm}
\usepackage{amscd}
\usepackage{picinpar}
\usepackage{graphicx}
\usepackage{tikz}
\usepackage{tikz-3dplot}
\usepackage{indentfirst}
\usepackage{wrapfig}
\usepackage{units}
\usepackage{textcomp}
\usepackage[utf8x]{inputenc}
% \usepackage{feyn}
\usepackage{feynmp}
\usepackage{xkeyval}
\usepackage{xargs}
\usepackage{verbatim}
\usepackage{pgfplots}
\usepackage{hyperref}
\usetikzlibrary{
  arrows,
  calc,
  decorations.pathmorphing,
  decorations.pathreplacing,
  decorations.markings,
  fadings,
  positioning,
  shapes
}

\DeclareGraphicsRule{*}{mps}{*}{}
\newcommand{\ud}{\mathrm{d}}
\newcommand{\ue}{\mathrm{e}}
\newcommand{\ui}{\mathrm{i}}
\newcommand{\res}{\mathrm{Res}}
\newcommand{\Tr}{\mathrm{Tr}}
\newcommand{\dsum}{\displaystyle\sum}
\newcommand{\dprod}{\displaystyle\prod}
\newcommand{\dlim}{\displaystyle\lim}
\newcommand{\dint}{\displaystyle\int}
\newcommand{\fsno}[1]{{\!\not\!{#1}}}
\newcommand{\eqar}[1]
{
  \begin{align*}
    #1
  \end{align*}
}
\newcommand{\texp}[2]{\ensuremath{{#1}\times10^{#2}}}
\newcommand{\dexp}[2]{\ensuremath{{#1}\cdot10^{#2}}}
\newcommand{\eval}[2]{{\left.{#1}\right|_{#2}}}
\newcommand{\paren}[1]{{\left({#1}\right)}}
\newcommand{\lparen}[1]{{\left({#1}\right.}}
\newcommand{\rparen}[1]{{\left.{#1}\right)}}
\newcommand{\abs}[1]{{\left|{#1}\right|}}
\newcommand{\sqr}[1]{{\left[{#1}\right]}}
\newcommand{\crly}[1]{{\left\{{#1}\right\}}}
\newcommand{\angl}[1]{{\left\langle{#1}\right\rangle}}
\newcommand{\tpdiff}[4][{}]{{\paren{\frac{\partial^{#1} {#2}}{\partial {#3}{}^{#1}}}_{#4}}}
\newcommand{\tpsdiff}[4][{}]{{\paren{\frac{\partial^{#1}}{\partial {#3}{}^{#1}}{#2}}_{#4}}}
\newcommand{\pdiff}[3][{}]{{\frac{\partial^{#1} {#2}}{\partial {#3}{}^{#1}}}}
\newcommand{\diff}[3][{}]{{\frac{\ud^{#1} {#2}}{\ud {#3}{}^{#1}}}}
\newcommand{\psdiff}[3][{}]{{\frac{\partial^{#1}}{\partial {#3}{}^{#1}} {#2}}}
\newcommand{\sdiff}[3][{}]{{\frac{\ud^{#1}}{\ud {#3}{}^{#1}} {#2}}}
\newcommand{\tpddiff}[4][{}]{{\left(\dfrac{\partial^{#1} {#2}}{\partial {#3}{}^{#1}}\right)_{#4}}}
\newcommand{\tpsddiff}[4][{}]{{\paren{\dfrac{\partial^{#1}}{\partial {#3}{}^{#1}}{#2}}_{#4}}}
\newcommand{\pddiff}[3][{}]{{\dfrac{\partial^{#1} {#2}}{\partial {#3}{}^{#1}}}}
\newcommand{\ddiff}[3][{}]{{\dfrac{\ud^{#1} {#2}}{\ud {#3}{}^{#1}}}}
\newcommand{\psddiff}[3][{}]{{\frac{\partial^{#1}}{\partial{}^{#1} {#3}} {#2}}}
\newcommand{\sddiff}[3][{}]{{\frac{\ud^{#1}}{\ud {#3}{}^{#1}} {#2}}}
\usepackage{fancyhdr}
\usepackage{multirow}
\usepackage{fontenc}
% \usepackage{tipa}
\usepackage{ulem}
\usepackage{color}
\usepackage{cancel}
\newcommand{\hcancel}[2][black]{\setbox0=\hbox{#2}%
  \rlap{\raisebox{.45\ht0}{\textcolor{#1}{\rule{\wd0}{1pt}}}}#2}
\pagestyle{fancy}
\setlength{\headheight}{67pt}
\fancyhead{}
\fancyfoot{}
\fancyfoot[C]{\thepage}
\fancyhead[R]{}
\renewcommand{\footruleskip}{0pt}
\renewcommand{\headrulewidth}{0.4pt}
\renewcommand{\footrulewidth}{0pt}

\newcommand\pgfmathsinandcos[3]{%
  \pgfmathsetmacro#1{sin(#3)}%
  \pgfmathsetmacro#2{cos(#3)}%
}
\newcommand\LongitudePlane[3][current plane]{%
  \pgfmathsinandcos\sinEl\cosEl{#2} % elevation
  \pgfmathsinandcos\sint\cost{#3} % azimuth
  \tikzset{#1/.estyle={cm={\cost,\sint*\sinEl,0,\cosEl,(0,0)}}}
}
\newcommand\LatitudePlane[3][current plane]{%
  \pgfmathsinandcos\sinEl\cosEl{#2} % elevation
  \pgfmathsinandcos\sint\cost{#3} % latitude
  \pgfmathsetmacro\yshift{\cosEl*\sint}
  \tikzset{#1/.estyle={cm={\cost,0,0,\cost*\sinEl,(0,\yshift)}}} %
}
\newcommand\DrawLongitudeCircle[2][1]{
  \LongitudePlane{\angEl}{#2}
  \tikzset{current plane/.prefix style={scale=#1}}
  % angle of "visibility"
  \pgfmathsetmacro\angVis{atan(sin(#2)*cos(\angEl)/sin(\angEl))} %
  \draw[current plane] (\angVis:1) arc (\angVis:\angVis+180:1);
  \draw[current plane,dashed] (\angVis-180:1) arc (\angVis-180:\angVis:1);
}
\newcommand\DrawLatitudeCircleArrow[2][1]{
  \LatitudePlane{\angEl}{#2}
  \tikzset{current plane/.prefix style={scale=#1}}
  \pgfmathsetmacro\sinVis{sin(#2)/cos(#2)*sin(\angEl)/cos(\angEl)}
  % angle of "visibility"
  \pgfmathsetmacro\angVis{asin(min(1,max(\sinVis,-1)))}
  \draw[current plane,decoration={markings, mark=at position 0.6 with {\arrow{<}}},postaction={decorate},line width=.6mm] (\angVis:1) arc (\angVis:-\angVis-180:1);
  \draw[current plane,dashed,line width=.6mm] (180-\angVis:1) arc (180-\angVis:\angVis:1);
}
\newcommand\DrawLatitudeCircle[2][1]{
  \LatitudePlane{\angEl}{#2}
  \tikzset{current plane/.prefix style={scale=#1}}
  \pgfmathsetmacro\sinVis{sin(#2)/cos(#2)*sin(\angEl)/cos(\angEl)}
  % angle of "visibility"
  \pgfmathsetmacro\angVis{asin(min(1,max(\sinVis,-1)))}
  \draw[current plane] (\angVis:1) arc (\angVis:-\angVis-180:1);
  \draw[current plane,dashed] (180-\angVis:1) arc (180-\angVis:\angVis:1);
}
\newcommand\coil[1]{
  {\rh * cos(\t * pi r)}, {\apart * (2 * #1 + \t) + \rv * sin(\t * pi r)}
}
\makeatletter
\define@key{DrawFromCenter}{style}[{->}]{
  \tikzset{DrawFromCenterPlane/.style={#1}}
}
\define@key{DrawFromCenter}{r}[1]{
  \def\@R{#1}
}
\define@key{DrawFromCenter}{center}[(0, 0)]{
  \def\@Center{#1}
}
\define@key{DrawFromCenter}{theta}[0]{
  \def\@Theta{#1}
}
\define@key{DrawFromCenter}{phi}[0]{
  \def\@Phi{#1}
}
\presetkeys{DrawFromCenter}{style, r, center, theta, phi}{}
\newcommand*\DrawFromCenter[1][]{
  \setkeys{DrawFromCenter}{#1}{
    \pgfmathsinandcos\sint\cost{\@Theta}
    \pgfmathsinandcos\sinp\cosp{\@Phi}
    \pgfmathsinandcos\sinA\cosA{\angEl}
    \pgfmathsetmacro\DX{\@R*\cost*\cosp}
    \pgfmathsetmacro\DY{\@R*(\cost*\sinp*\sinA+\sint*\cosA)}
    \draw[DrawFromCenterPlane] \@Center -- ++(\DX, \DY);
  }
}
\newcommand*\DrawFromCenterText[2][]{
  \setkeys{DrawFromCenter}{#1}{
    \pgfmathsinandcos\sint\cost{\@Theta}
    \pgfmathsinandcos\sinp\cosp{\@Phi}
    \pgfmathsinandcos\sinA\cosA{\angEl}
    \pgfmathsetmacro\DX{\@R*\cost*\cosp}
    \pgfmathsetmacro\DY{\@R*(\cost*\sinp*\sinA+\sint*\cosA)}
    \draw[DrawFromCenterPlane] \@Center -- ++(\DX, \DY) node {#2};
  }
}
\makeatother
\tikzstyle{snakearrow} = [decorate, decoration={pre length=0.2cm,
  post length=0.2cm, snake, amplitude=.4mm,
  segment length=2mm},thick, ->]
%% document-wide tikz options and styles
\tikzset{%
  >=latex, % option for nice arrows
  inner sep=0pt,%
  outer sep=2pt,%
  mark coordinate/.style={inner sep=0pt,outer sep=0pt,minimum size=3pt,
    fill=black,circle}%
}
\addtolength{\hoffset}{-1.3cm}
\addtolength{\voffset}{-2cm}
\addtolength{\textwidth}{3cm}
\addtolength{\textheight}{2.5cm}
\renewcommand{\footskip}{10pt}
\setlength{\headwidth}{\textwidth}
\setlength{\headsep}{20pt}
\setlength{\marginparwidth}{0pt}
\parindent=0pt
\title{Exact formula for quantum jump method in two level system with decay and coupling including detuning}
\begin{document}

\maketitle

\section{The problem}
The derivation in \href{precise-quantum-jump.pdf}{the previous note} does not include
detuning caused by trap anharmonicity. The anharmonicity should be small in the small range
of states that is driven by any given orders but we would still like to include it in
the simulation to see how big the effect actually is.

Another effect that can be included is the decoherence. We should be able to simulate the
fast component with the decay term and the slow component
(e.g. B field drift that's slower than the experimental cycles) by a random overall detuning.

Since we are still only dealing with a single drive, the Hamiltonian will still
be time independent. We are also still ignoring the off-resonant scattering so the system
is still two-level. The main difference is that the diagnal part of the Hamiltonian will
now have a real part and the resulting state may be imaginary.

\section{Modified Hamiltonian}
With the detuning included, the effective Hamiltonian is now,
\eqar{
  H'=&-\frac{\ui}2\begin{pmatrix}
    \Gamma_1+\ui\delta&\Omega\\
    -\Omega&\Gamma_2-\ui\delta
  \end{pmatrix}
  \intertext{Define}
  \Gamma_1\equiv&\Gamma+\gamma\\
  \Gamma_2\equiv&\Gamma-\gamma
  \intertext{We have}
  H'=&-\frac{\ui}2\begin{pmatrix}
    \Gamma+\gamma+\ui\delta&\Omega\\
    -\Omega&\Gamma-\gamma-\ui\delta
  \end{pmatrix}
}

\section{Time evolution}

Formally the time evolution is
\eqar{
  &\exp\paren{-\ui H't}\\
  =&\exp\paren{-\frac t2\begin{pmatrix}
      \Gamma+\gamma+\ui\delta&\Omega\\
      -\Omega&\Gamma-\gamma-\ui\delta
    \end{pmatrix}}
}
Define $\Omega'=\sqrt{\Omega^2-\paren{\gamma+\ui\delta}^2}$ (complex)
\eqar{
  &\exp\paren{-\ui H't}\\
  =&\frac{\ue^{-\Gamma t/2}}{2\ui\Omega'}\begin{pmatrix}
    \gamma\paren{\ue^{-\frac{\ui\Omega't}{2}}-\ue^{\frac{\ui\Omega't}{2}}}+\ui\Omega'\paren{\ue^{-\frac{\ui\Omega't}{2}}+\ue^{\frac{\ui\Omega't}{2}}}&\Omega\paren{\ue^{-\frac{\ui\Omega't}{2}}-\ue^{\frac{\ui\Omega't}{2}}}\\
    -\Omega\paren{\ue^{-\frac{\ui\Omega't}{2}}-\ue^{\frac{\ui\Omega't}{2}}}&-\gamma\paren{\ue^{-\frac{\ui\Omega't}{2}}-\ue^{\frac{\ui\Omega't}{2}}}+\ui\Omega'\paren{\ue^{-\frac{\ui\Omega't}{2}}+\ue^{\frac{\ui\Omega't}{2}}}
  \end{pmatrix}\\
  =&\frac{\ue^{-\Gamma t/2}}{2\ui\Omega'}\begin{pmatrix}
    -2\ui\gamma\sin\dfrac{\Omega't}{2}+2\ui\Omega'\cos\dfrac{\Omega't}{2}&-2\ui\Omega\sin\dfrac{\Omega't}{2}\\
    2\ui\Omega\sin\dfrac{\Omega't}{2}&2\ui\gamma\sin\dfrac{\Omega't}{2}+2\ui\Omega'\cos\dfrac{\Omega't}{2}
  \end{pmatrix}\\
  =&\frac{\ue^{-\Gamma t/2}}{\Omega'}\begin{pmatrix}
    -\gamma\sin\dfrac{\Omega't}{2}+\Omega'\cos\dfrac{\Omega't}{2}&-\Omega\sin\dfrac{\Omega't}{2}\\
    \Omega\sin\dfrac{\Omega't}{2}&\gamma\sin\dfrac{\Omega't}{2}+\Omega'\cos\dfrac{\Omega't}{2}
  \end{pmatrix}
}
Starting from the atom in state 1, the wave functions are
\eqar{
  \psi=&\frac{\ue^{-\Gamma t/2}}{\Omega'}\begin{pmatrix}
    -\gamma\sin\dfrac{\Omega't}{2}+\Omega'\cos\dfrac{\Omega't}{2}&-\Omega\sin\dfrac{\Omega't}{2}\\
    \Omega\sin\dfrac{\Omega't}{2}&\gamma\sin\dfrac{\Omega't}{2}+\Omega'\cos\dfrac{\Omega't}{2}
  \end{pmatrix}\begin{pmatrix}
    1\\
    0
  \end{pmatrix}\\
  =&\frac{\ue^{-\Gamma t/2}}{\Omega'}\begin{pmatrix}
    -\gamma\sin\dfrac{\Omega't}{2}+\Omega'\cos\dfrac{\Omega't}{2}\\
    \Omega\sin\dfrac{\Omega't}{2}
  \end{pmatrix}
}

The quantities we care about are:
\begin{itemize}
\item Total decay probability ($1-\langle\psi|\psi\rangle$)
\item Instantaneous decay probability for each channel ($\langle\psi|C_m^\dagger C_m|\psi\rangle$ for $m=1,\ 2$)
\end{itemize}
These results will be different from the on-resonance case since we are taking absolute value
(probabilities).\\

Some equations useful for computing these values,
\eqar{
  \Omega'_r=&\Re\paren{\Omega'}\\
  \Omega'_i=&\Im\paren{\Omega'}\\
  \Omega'=&\Omega'_r+\ui\Omega'_i\\
  &\sin\paren{a+\ui b}\\
  =&\sin a\cos\ui b+\cos a\sin\ui b\\
  =&\sin a\cosh b+\ui\cos a\sinh b\\
  &\cos\paren{a+\ui b}\\
  =&\cos a\cos\ui b+\sin a\sin\ui b\\
  =&\cos a\cosh b+\ui\sin a\sinh b\\
  &\sin^2\dfrac{a}{2}\sinh^2\dfrac{b}{2}+\cos^2\dfrac{a}{2}\cosh^2\dfrac{b}{2}\\
  =&\frac{(1 - \cos a)(\cosh b - 1)+(1 + \cos a)(\cosh b + 1)}{4}\\
  =&\frac{\cosh b + \cos a}{2}\\
  &\sin^2\dfrac{a}{2}\cosh^2\dfrac{b}{2}+\cos^2\dfrac{a}{2}\sinh^2\dfrac{b}{2}\\
  =&\frac{(1 - \cos a)(\cosh b + 1)+(1 + \cos a)(\cosh b - 1)}{4}\\
  =&\frac{\cosh b - \cos a}{2}
}
\eqar{
  &-\gamma\sin\dfrac{\Omega't}{2}+\Omega'\cos\dfrac{\Omega't}{2}\\
  =&-\gamma\paren{\sin\dfrac{\Omega'_rt}{2}\cosh\dfrac{\Omega'_it}{2}-\ui\sinh\dfrac{\Omega'_it}{2}\cos\dfrac{\Omega'_rt}{2}}\\
  &+\paren{\Omega'_r+\ui\Omega'_i}\paren{\cos\dfrac{\Omega'_rt}{2}\cosh\dfrac{\Omega'_it}{2}+\ui\sin\dfrac{\Omega'_rt}{2}\sinh\dfrac{\Omega'_it}{2}}\\
  =&\paren{\Omega'_r\cos\dfrac{\Omega'_rt}{2}-\gamma\sin\dfrac{\Omega'_rt}{2}}\cosh\dfrac{\Omega'_it}{2}-\Omega'_i\sin\dfrac{\Omega'_rt}{2}\sinh\dfrac{\Omega'_it}{2}\\
  &+\ui\paren{\paren{\gamma\cos\dfrac{\Omega'_rt}{2}+\Omega'_r\sin\dfrac{\Omega'_rt}{2}}\sinh\dfrac{\Omega'_it}{2}+\Omega'_i\cos\dfrac{\Omega'_rt}{2}\cosh\dfrac{\Omega'_it}{2}}
  \intertext{}
  &\abs{-\gamma\sin\dfrac{\Omega't}{2}+\Omega'\cos\dfrac{\Omega't}{2}}^2\\
  =
  &{\Omega'_r}^2\cos^2\dfrac{\Omega'_rt}{2}\cosh^2\dfrac{\Omega'_it}{2}+\gamma^2\sin^2\dfrac{\Omega'_rt}{2}\cosh^2\dfrac{\Omega'_it}{2}+{\Omega'_i}^2\sin^2\dfrac{\Omega'_rt}{2}\sinh^2\dfrac{\Omega'_it}{2}\\
  &-2\gamma\Omega'_r\sin\dfrac{\Omega'_rt}{2}\cos\dfrac{\Omega'_rt}{2}\cosh^2\dfrac{\Omega'_it}{2}-2\Omega'_i\Omega'_r\sin\dfrac{\Omega'_rt}{2}\cos\dfrac{\Omega'_rt}{2}\sinh\dfrac{\Omega'_it}{2}\cosh\dfrac{\Omega'_it}{2}\\
  &+2\gamma\Omega'_i\sin^2\dfrac{\Omega'_rt}{2}\sinh\dfrac{\Omega'_it}{2}\cosh\dfrac{\Omega'_it}{2}\\
  &+\gamma^2\cos^2\dfrac{\Omega'_rt}{2}\sinh^2\dfrac{\Omega'_it}{2}+{\Omega'_r}^2\sin^2\dfrac{\Omega'_rt}{2}\sinh^2\dfrac{\Omega'_it}{2}+{\Omega'_i}^2\cos^2\dfrac{\Omega'_rt}{2}\cosh^2\dfrac{\Omega'_it}{2}\\
  &+2\gamma\Omega'_r\sin\dfrac{\Omega'_rt}{2}\cos\dfrac{\Omega'_rt}{2}\sinh^2\dfrac{\Omega'_it}{2}+2\gamma\Omega'_i\cos^2\dfrac{\Omega'_rt}{2}\sinh\dfrac{\Omega'_it}{2}\cosh\dfrac{\Omega'_it}{2}\\
  &+2\Omega'_i\Omega'_r\sin\dfrac{\Omega'_rt}{2}\cos\dfrac{\Omega'_rt}{2}\sinh\dfrac{\Omega'_it}{2}\cosh\dfrac{\Omega'_it}{2}\\
  =&\abs{\Omega'}^2\paren{\sin^2\dfrac{\Omega'_rt}{2}\sinh^2\dfrac{\Omega'_it}{2}+\cos^2\dfrac{\Omega'_rt}{2}\cosh^2\dfrac{\Omega'_it}{2}}\\
  &+2\gamma\Omega'_i\paren{\sinh\dfrac{\Omega'_it}{2}\cosh\dfrac{\Omega'_it}{2}-\sin\dfrac{\Omega'_rt}{2}\cos\dfrac{\Omega'_rt}{2}}\\
  &+\gamma^2\paren{\sin^2\dfrac{\Omega'_rt}{2}\cosh^2\dfrac{\Omega'_it}{2}+\cos^2\dfrac{\Omega'_rt}{2}\sinh^2\dfrac{\Omega'_it}{2}}\\
  =&
  \frac{\abs{\Omega'}^2-\gamma^2}2\cos\Omega'_rt
  +\frac{\abs{\Omega'}^2+\gamma^2}2\cosh\Omega'_it
  +\gamma\Omega'_i\paren{\sinh\Omega'_it-\sin\Omega'_rt}
  \intertext{}
  &\abs{\Omega\sin\dfrac{\Omega't}{2}}^2\\
  =&\Omega^2\abs{\sin\dfrac{\Omega'_rt}{2}\cosh\dfrac{\Omega'_it}{2}-\ui\sinh\dfrac{\Omega'_it}{2}\cos\dfrac{\Omega'_rt}{2}}^2\\
  =&\Omega^2\paren{
    \sin^2\dfrac{\Omega'_rt}{2}\cosh^2\dfrac{\Omega'_it}{2}
    +\cos^2\dfrac{\Omega'_rt}{2}\sinh^2\dfrac{\Omega'_it}{2}
  }\\
  =&\frac{\Omega^2}2\paren{\cosh\Omega'_it-\cos\Omega'_rt}
}

\section{Decay rates}
The decay rate for state 1 is
\eqar{
  &\langle\psi|C_1^\dagger C_1|\psi\rangle\\
  =&\frac{\Gamma_1\ue^{-\Gamma t}}{\abs{\Omega'}^2}\abs{-\gamma\sin\dfrac{\Omega't}{2}+\Omega'\cos\dfrac{\Omega't}{2}}^2\\
  =&\frac{\Gamma_1\ue^{-\Gamma t}}{2\abs{\Omega'}^2}\paren{
    \paren{\abs{\Omega'}^2-\gamma^2}\cos\Omega'_rt
    +\paren{\abs{\Omega'}^2+\gamma^2}\cosh\Omega'_it
    +2\gamma\Omega'_i\paren{\sinh\Omega'_it-\sin\Omega'_rt}
  }
}
The decay rate for state 2 is
\eqar{
  &\langle\psi|C_2^\dagger C_2|\psi\rangle\\
  =&\frac{\Gamma_2\ue^{-\Gamma t}}{\abs{\Omega'}^2}\abs{\Omega\sin\dfrac{\Omega't}{2}}^2\\
  =&\frac{\Gamma_2\Omega^2\ue^{-\Gamma t}}{2\abs{\Omega'}^2}\paren{\cosh\Omega'_it-\cos\Omega'_rt}\\
}

\subsection{Total decay probability}
The total probability of not decaying,
\eqar{
  \langle\psi|\psi\rangle=&\frac{\ue^{-\Gamma t}}{\abs{\Omega'}^2}\paren{
    \abs{-\gamma\sin\dfrac{\Omega't}{2}+\Omega'\cos\dfrac{\Omega't}{2}}^2
    +\abs{\Omega\sin\dfrac{\Omega't}{2}}^2
  }\\
  =&\frac{\ue^{-\Gamma t}}{\abs{\Omega'}^2}\paren{
    \frac{\abs{\Omega'}^2-\gamma^2-\Omega^2}2\cos\Omega'_rt
    +\frac{\abs{\Omega'}^2+\gamma^2+\Omega^2}2\cosh\Omega'_it
    +\gamma\Omega'_i\paren{\sinh\Omega'_it-\sin\Omega'_rt}
  }
}

\subsection{Total rate}
Calculate the derivative of $\langle\psi|\psi\rangle$ to help root finding with Newton's method.
\eqar{
  &\diff{}{t}\langle\psi|\psi\rangle\\
  =&-\langle\psi|C_1^\dagger C_1|\psi\rangle-\langle\psi|C_2^\dagger C_2|\psi\rangle\\
  =&-\frac{\ue^{-\Gamma t}}{2\abs{\Omega'}^2}\paren{
    \Gamma_1\paren{\abs{\Omega'}^2-\gamma^2}\cos\Omega'_rt
    +\Gamma_1\paren{\abs{\Omega'}^2+\gamma^2}\cosh\Omega'_it
    +2\Gamma_1\gamma\Omega'_i\paren{\sinh\Omega'_it-\sin\Omega'_rt}
  }\\
  &-\frac{\ue^{-\Gamma t}}{2\abs{\Omega'}^2}\paren{\Gamma_2\Omega^2\cosh\Omega'_it-\Gamma_2\Omega^2\cos\Omega'_rt}\\
  =&-\frac{\ue^{-\Gamma t}}{2\abs{\Omega'}^2}\Bigg(
  \paren{\Gamma_1\abs{\Omega'}^2-\Gamma_1\gamma^2-\Gamma_2\Omega^2}\cos\Omega'_rt
  +\paren{\Gamma_1\abs{\Omega'}^2+\Gamma_1\gamma^2+\Gamma_2\Omega^2}\cosh\Omega'_it\\
  &+2\Gamma_1\gamma\Omega'_i\paren{\sinh\Omega'_it-\sin\Omega'_rt}\Bigg)
}

\end{document}
