\documentclass[10pt,fleqn]{article}
% \usepackage[journal=rsc]{chemstyle}
% \usepackage{mhchem}
\usepackage{amsmath}
\usepackage{amssymb}
\usepackage{amsfonts}
\usepackage{esint}
\usepackage{bbm}
\usepackage{amscd}
\usepackage{picinpar}
\usepackage{graphicx}
\usepackage{tikz}
\usepackage{tikz-3dplot}
\usepackage{indentfirst}
\usepackage{wrapfig}
\usepackage{units}
\usepackage{textcomp}
\usepackage[utf8x]{inputenc}
% \usepackage{feyn}
% \usepackage{feynmp}
\usepackage{xkeyval}
\usepackage{xargs}
\usepackage{verbatim}
\usepackage{pgfplots}
\usepackage{hyperref}
\usetikzlibrary{
  arrows,
  calc,
  decorations.pathmorphing,
  decorations.pathreplacing,
  decorations.markings,
  fadings,
  positioning,
  shapes
}

\DeclareGraphicsRule{*}{mps}{*}{}
\newcommand{\ud}{\mathrm{d}}
\newcommand{\ue}{\mathrm{e}}
\newcommand{\ui}{\mathrm{i}}
\newcommand{\res}{\mathrm{Res}}
\newcommand{\Tr}{\mathrm{Tr}}
\newcommand{\dsum}{\displaystyle\sum}
\newcommand{\dprod}{\displaystyle\prod}
\newcommand{\dlim}{\displaystyle\lim}
\newcommand{\dint}{\displaystyle\int}
\newcommand{\fsno}[1]{{\!\not\!{#1}}}
\newcommand{\eqar}[1]
{
  \begin{align}
    #1
  \end{align}
}
\newcommand{\texp}[2]{\ensuremath{{#1}\times10^{#2}}}
\newcommand{\dexp}[2]{\ensuremath{{#1}\cdot10^{#2}}}
\newcommand{\eval}[2]{{\left.{#1}\right|_{#2}}}
\newcommand{\paren}[1]{{\left({#1}\right)}}
\newcommand{\lparen}[1]{{\left({#1}\right.}}
\newcommand{\rparen}[1]{{\left.{#1}\right)}}
\newcommand{\abs}[1]{{\left|{#1}\right|}}
\newcommand{\sqr}[1]{{\left[{#1}\right]}}
\newcommand{\crly}[1]{{\left\{{#1}\right\}}}
\newcommand{\angl}[1]{{\left\langle{#1}\right\rangle}}
\newcommand{\tpdiff}[4][{}]{{\paren{\frac{\partial^{#1} {#2}}{\partial {#3}{}^{#1}}}_{#4}}}
\newcommand{\tpsdiff}[4][{}]{{\paren{\frac{\partial^{#1}}{\partial {#3}{}^{#1}}{#2}}_{#4}}}
\newcommand{\pdiff}[3][{}]{{\frac{\partial^{#1} {#2}}{\partial {#3}{}^{#1}}}}
\newcommand{\diff}[3][{}]{{\frac{\ud^{#1} {#2}}{\ud {#3}{}^{#1}}}}
\newcommand{\psdiff}[3][{}]{{\frac{\partial^{#1}}{\partial {#3}{}^{#1}} {#2}}}
\newcommand{\sdiff}[3][{}]{{\frac{\ud^{#1}}{\ud {#3}{}^{#1}} {#2}}}
\newcommand{\tpddiff}[4][{}]{{\left(\dfrac{\partial^{#1} {#2}}{\partial {#3}{}^{#1}}\right)_{#4}}}
\newcommand{\tpsddiff}[4][{}]{{\paren{\dfrac{\partial^{#1}}{\partial {#3}{}^{#1}}{#2}}_{#4}}}
\newcommand{\pddiff}[3][{}]{{\dfrac{\partial^{#1} {#2}}{\partial {#3}{}^{#1}}}}
\newcommand{\ddiff}[3][{}]{{\dfrac{\ud^{#1} {#2}}{\ud {#3}{}^{#1}}}}
\newcommand{\psddiff}[3][{}]{{\frac{\partial^{#1}}{\partial{}^{#1} {#3}} {#2}}}
\newcommand{\sddiff}[3][{}]{{\frac{\ud^{#1}}{\ud {#3}{}^{#1}} {#2}}}
\usepackage{fancyhdr}
\usepackage{multirow}
\usepackage{fontenc}
% \usepackage{tipa}
\usepackage{ulem}
\usepackage{color}
\usepackage{cancel}
\newcommand{\hcancel}[2][black]{\setbox0=\hbox{#2}%
  \rlap{\raisebox{.45\ht0}{\textcolor{#1}{\rule{\wd0}{1pt}}}}#2}
\pagestyle{fancy}
\setlength{\headheight}{67pt}
\fancyhead{}
\fancyfoot{}
\fancyfoot[C]{\thepage}
\fancyhead[R]{}
\renewcommand{\footruleskip}{0pt}
\renewcommand{\headrulewidth}{0.4pt}
\renewcommand{\footrulewidth}{0pt}

\newcommand\pgfmathsinandcos[3]{%
  \pgfmathsetmacro#1{sin(#3)}%
  \pgfmathsetmacro#2{cos(#3)}%
}
\newcommand\LongitudePlane[3][current plane]{%
  \pgfmathsinandcos\sinEl\cosEl{#2} % elevation
  \pgfmathsinandcos\sint\cost{#3} % azimuth
  \tikzset{#1/.estyle={cm={\cost,\sint*\sinEl,0,\cosEl,(0,0)}}}
}
\newcommand\LatitudePlane[3][current plane]{%
  \pgfmathsinandcos\sinEl\cosEl{#2} % elevation
  \pgfmathsinandcos\sint\cost{#3} % latitude
  \pgfmathsetmacro\yshift{\cosEl*\sint}
  \tikzset{#1/.estyle={cm={\cost,0,0,\cost*\sinEl,(0,\yshift)}}} %
}
\newcommand\DrawLongitudeCircle[2][1]{
  \LongitudePlane{\angEl}{#2}
  \tikzset{current plane/.prefix style={scale=#1}}
  % angle of "visibility"
  \pgfmathsetmacro\angVis{atan(sin(#2)*cos(\angEl)/sin(\angEl))} %
  \draw[current plane] (\angVis:1) arc (\angVis:\angVis+180:1);
  \draw[current plane,dashed] (\angVis-180:1) arc (\angVis-180:\angVis:1);
}
\newcommand\DrawLatitudeCircleArrow[2][1]{
  \LatitudePlane{\angEl}{#2}
  \tikzset{current plane/.prefix style={scale=#1}}
  \pgfmathsetmacro\sinVis{sin(#2)/cos(#2)*sin(\angEl)/cos(\angEl)}
  % angle of "visibility"
  \pgfmathsetmacro\angVis{asin(min(1,max(\sinVis,-1)))}
  \draw[current plane,decoration={markings, mark=at position 0.6 with {\arrow{<}}},postaction={decorate},line width=.6mm] (\angVis:1) arc (\angVis:-\angVis-180:1);
  \draw[current plane,dashed,line width=.6mm] (180-\angVis:1) arc (180-\angVis:\angVis:1);
}
\newcommand\DrawLatitudeCircle[2][1]{
  \LatitudePlane{\angEl}{#2}
  \tikzset{current plane/.prefix style={scale=#1}}
  \pgfmathsetmacro\sinVis{sin(#2)/cos(#2)*sin(\angEl)/cos(\angEl)}
  % angle of "visibility"
  \pgfmathsetmacro\angVis{asin(min(1,max(\sinVis,-1)))}
  \draw[current plane] (\angVis:1) arc (\angVis:-\angVis-180:1);
  \draw[current plane,dashed] (180-\angVis:1) arc (180-\angVis:\angVis:1);
}
\newcommand\coil[1]{
  {\rh * cos(\t * pi r)}, {\apart * (2 * #1 + \t) + \rv * sin(\t * pi r)}
}
\makeatletter
\define@key{DrawFromCenter}{style}[{->}]{
  \tikzset{DrawFromCenterPlane/.style={#1}}
}
\define@key{DrawFromCenter}{r}[1]{
  \def\@R{#1}
}
\define@key{DrawFromCenter}{center}[(0, 0)]{
  \def\@Center{#1}
}
\define@key{DrawFromCenter}{theta}[0]{
  \def\@Theta{#1}
}
\define@key{DrawFromCenter}{phi}[0]{
  \def\@Phi{#1}
}
\presetkeys{DrawFromCenter}{style, r, center, theta, phi}{}
\newcommand*\DrawFromCenter[1][]{
  \setkeys{DrawFromCenter}{#1}{
    \pgfmathsinandcos\sint\cost{\@Theta}
    \pgfmathsinandcos\sinp\cosp{\@Phi}
    \pgfmathsinandcos\sinA\cosA{\angEl}
    \pgfmathsetmacro\DX{\@R*\cost*\cosp}
    \pgfmathsetmacro\DY{\@R*(\cost*\sinp*\sinA+\sint*\cosA)}
    \draw[DrawFromCenterPlane] \@Center -- ++(\DX, \DY);
  }
}
\newcommand*\DrawFromCenterText[2][]{
  \setkeys{DrawFromCenter}{#1}{
    \pgfmathsinandcos\sint\cost{\@Theta}
    \pgfmathsinandcos\sinp\cosp{\@Phi}
    \pgfmathsinandcos\sinA\cosA{\angEl}
    \pgfmathsetmacro\DX{\@R*\cost*\cosp}
    \pgfmathsetmacro\DY{\@R*(\cost*\sinp*\sinA+\sint*\cosA)}
    \draw[DrawFromCenterPlane] \@Center -- ++(\DX, \DY) node {#2};
  }
}
\makeatother
\tikzstyle{snakearrow} = [decorate, decoration={pre length=0.2cm,
  post length=0.2cm, snake, amplitude=.4mm,
  segment length=2mm},thick, ->]
%% document-wide tikz options and styles
\tikzset{%
  >=latex, % option for nice arrows
  inner sep=0pt,%
  outer sep=2pt,%
  mark coordinate/.style={inner sep=0pt,outer sep=0pt,minimum size=3pt,
    fill=black,circle}%
}
\addtolength{\hoffset}{-1.3cm}
\addtolength{\voffset}{-2cm}
\addtolength{\textwidth}{3cm}
\addtolength{\textheight}{2.5cm}
\renewcommand{\footskip}{10pt}
\setlength{\headwidth}{\textwidth}
\setlength{\headsep}{20pt}
\setlength{\marginparwidth}{0pt}
\parindent=0pt
\title{Magnus expansion with linearly changing Hamiltonian}

\ifpdf
  % Ensure reproducible output
  \pdfinfoomitdate=1
  \pdfsuppressptexinfo=-1
  \pdftrailerid{}
  \hypersetup{
    pdfcreator={},
    pdfproducer={}
  }
\fi

\begin{document}

\maketitle

With a Hamiltonian

\eqar{
  H(t)=&H_0+H_1 t
}

The commutators for the the leading order of Magnus expansion,

\eqar{
  \begin{split}
    \sqr{H(t_1),H(t_2)}=&\sqr{H_0+H_1 t_1,H_0+H_1 t_2}\\
    =&\sqr{H_0,H_1} t_2+\sqr{H_1,H_0} t_1\\
    =&\sqr{H_0,H_1} \paren{t_2-t_1}\\
  \end{split}\\
  \begin{split}
    \sqr{\sqr{H(t_1),H(t_2)},H(t_3)}=&\sqr{\sqr{H_0,H_1} \paren{t_2-t_1},H_0+H_1 t_3}\\
    =&\paren{\sqr{\sqr{H_0,H_1},H_0}+\sqr{\sqr{H_0,H_1},H_1} t_3}\paren{t_2-t_1}\\
  \end{split}\\
  \begin{split}
    &\sqr{\sqr{\sqr{H(t_1),H(t_2)},H(t_3)},H(t_4)}\\
    =&\sqr{\paren{\sqr{\sqr{H_0,H_1},H_0}+\sqr{\sqr{H_0,H_1},H_1} t_3},H_0+H_1 t_4}\paren{t_2-t_1}\\
    =&\sqr{\sqr{\sqr{H_0,H_1},H_0},H_0}\paren{t_2-t_1}+\sqr{\sqr{\sqr{H_0,H_1},H_0},H_1}\paren{t_2-t_1} t_4\\
     &+\sqr{\sqr{\sqr{H_0,H_1},H_1},H_0}t_3 \paren{t_2-t_1}+\sqr{\sqr{\sqr{H_0,H_1},H_1},H_1}t_3 t_4\paren{t_2-t_1}\\
  \end{split}
}

The terms in the time integral,
\eqar{
  \begin{split}
    &\sqr{H(t_1),\sqr{H(t_2),H(t_3)}}+\sqr{H(t_3),\sqr{H(t_2),H(t_1)}}\\
    =&\sqr{\sqr{H_0,H_1},H_0}\paren{t_2-t_3}+\sqr{\sqr{H_0,H_1},H_1} t_1\paren{t_2-t_3}\\
    &+\sqr{\sqr{H_0,H_1},H_0}\paren{t_2-t_1}+\sqr{\sqr{H_0,H_1},H_1} t_3\paren{t_2-t_1}\\
    =&\sqr{\sqr{H_0,H_1},H_0}\paren{2t_2-t_1-t_3}+\sqr{\sqr{H_0,H_1},H_1}\paren{t_1t_2+t_2t_3-2t_1t_3}\\
  \end{split}\\
  \begin{split}
    &\sqr{\sqr{\sqr{H(t_1),H(t_2)},H(t_3)},H(t_4)}+\sqr{\sqr{\sqr{H(t_3),H(t_2)},H(t_4)},H(t_1)}\\
    &+\sqr{\sqr{\sqr{H(t_3),H(t_4)},H(t_2)},H(t_1)}+\sqr{\sqr{\sqr{H(t_4),H(t_1)},H(t_3)},H(t_2)}\\
    =&\sqr{\sqr{\sqr{H_0,H_1},H_0},H_0}\paren{t_2-t_1+t_2-t_3+t_4-t_3+t_1-t_4}\\
    &+\sqr{\sqr{\sqr{H_0,H_1},H_0},H_1}\paren{t_2 t_4-t_1 t_4+t_2t_1-t_3t_1+t_4t_1-t_3t_1+t_1t_2-t_4t_2}\\
    &+\sqr{\sqr{\sqr{H_0,H_1},H_1},H_0}\paren{t_2t_3-t_1t_3+t_2t_4-t_3t_4+t_4t_2-t_3t_2+t_1t_3-t_4t_3}\\
    &+\sqr{\sqr{\sqr{H_0,H_1},H_1},H_1}\paren{t_2t_3t_4-t_1t_3t_4
      +t_2t_1t_4-t_3t_1t_4+t_4t_1t_2-t_3t_1t_2+t_1t_2t_3-t_4t_2t_3}\\
    =&\sqr{\sqr{\sqr{H_0,H_1},H_0},H_0}2\paren{t_2-t_3}+\sqr{\sqr{\sqr{H_0,H_1},H_0},H_1}2t_1\paren{t_2-t_3}\\
    &+\sqr{\sqr{\sqr{H_0,H_1},H_1},H_0}2t_4\paren{t_2-t_3}+\sqr{\sqr{\sqr{H_0,H_1},H_1},H_1}2t_1t_4\paren{t_2-t_3}\\
  \end{split}
}

Integrals,
\eqar{
  \begin{split}
    \Omega_1=&\int_{\tau_1}^{\tau_2}H(t_1)\ud t_1\\
    =&\int_{\tau_1}^{\tau_2}H_0+H_1 t_1\ud t_1\\
    =&\paren{\tau_2-\tau_1}\paren{H_0+\frac{\tau_2+\tau_1}{2}H_1}\\
  \end{split}\\
  \begin{split}
    \Omega_2=&\frac12\int_{\tau_1}^{\tau_2}\ud t_1\int_{\tau_1}^{t_1}\ud t_2 \sqr{H(t_1),H(t_2)}\\
    =&\sqr{H_0,H_1}\frac12\int_{\tau_1}^{\tau_2}\ud t_1\int_{\tau_1}^{t_1}\ud t_2 \paren{t_2-t_1}\\
    =&\sqr{H_0,H_1}\frac14\int_{\tau_1}^{\tau_2}\ud t_1\paren{t_1-\tau_1}^2\\
    =&-\frac{\paren{\tau_2-\tau_1}^3}{12}\sqr{H_0,H_1}\\
  \end{split}\\
  \begin{split}
    \Omega_3=&\frac16\int_{\tau_1}^{\tau_2}\!\!\ud t_1\int_{\tau_1}^{t_1}\!\!\ud t_2\int_{\tau_1}^{t_2}\!\!\ud t_3 \sqr{H(t_1),\sqr{H(t_2),H(t_3)}}+\sqr{H(t_3),\sqr{H(t_2),H(t_1)}}\\
    =&\frac16\int_{\tau_1}^{\tau_2}\!\!\ud t_1\int_{\tau_1}^{t_1}\!\!\ud t_2\int_{\tau_1}^{t_2}\!\!\ud t_3\sqr{\sqr{H_0,H_1},H_0}\paren{2t_2-t_1-t_3}+\sqr{\sqr{H_0,H_1},H_1}\paren{t_1t_2+t_2t_3-2t_1t_3}\\
    =&\frac{\paren{\tau_2-\tau_1}^5}{240}\sqr{\sqr{H_0,H_1},H_1}\\
  \end{split}\\
  \begin{split}
    \Omega_4=&\frac{1}{12}\int_{\tau_1}^{\tau_2}\!\!\ud t_1\int_{\tau_1}^{t_1}\!\!\ud t_2\int_{\tau_1}^{t_2}\!\!\ud t_3\int_{\tau_1}^{t_3}\!\!\ud t_4\sqr{\sqr{\sqr{H(t_1),H(t_2)},H(t_3)},H(t_4)}+\sqr{\sqr{\sqr{H(t_3),H(t_2)},H(t_4)},H(t_1)}\\
    &+\sqr{\sqr{\sqr{H(t_3),H(t_4)},H(t_2)},H(t_1)}+\sqr{\sqr{\sqr{H(t_4),H(t_1)},H(t_3)},H(t_2)}\\
    =&\frac{1}{6}\int_{\tau_1}^{\tau_2}\!\!\ud t_1\int_{\tau_1}^{t_1}\!\!\ud t_2\int_{\tau_1}^{t_2}\!\!\ud t_3\int_{\tau_1}^{t_3}\!\!\ud t_4\sqr{\sqr{\sqr{H_0,H_1},H_0},H_0}\paren{t_2-t_3}+\sqr{\sqr{\sqr{H_0,H_1},H_0},H_1}t_1\paren{t_2-t_3}\\
    &+\sqr{\sqr{\sqr{H_0,H_1},H_1},H_0}t_4\paren{t_2-t_3}+\sqr{\sqr{\sqr{H_0,H_1},H_1},H_1}t_1t_4\paren{t_2-t_3}\\
    =&\frac{\paren{\tau_2-\tau_1}^5}{720}\lparen{\sqr{\sqr{\sqr{H_0,H_1},H_0},H_0}+\frac{\tau_1+5\tau_2}{6}\sqr{\sqr{\sqr{H_0,H_1},H_0},H_1}}\\
    &+\rparen{\frac{5\tau_1+\tau_2}{6}\sqr{\sqr{\sqr{H_0,H_1},H_1},H_0}
      +\frac{\tau_1^2+5\tau_1\tau_2+\tau_2^2}{7}\sqr{\sqr{\sqr{H_0,H_1},H_1},H_1}}\\
  \end{split}
}

For Pauli,
\eqar{
  \begin{split}
    H_0=&\vec h_0\cdot\vec\sigma\\
  \end{split}\\
  \begin{split}
  H_1=&\vec h_1\cdot\vec\sigma\\
  \end{split}
}

Commutators,
\eqar{
  \begin{split}
    \sqr{H_0,H_1}=&\sqr{\vec h_0\cdot\vec\sigma,\vec h_1\cdot\vec\sigma}\\
    =&2\ui\paren{\vec h_0\times\vec h_1}\cdot\sigma\\
  \end{split}\\
  \begin{split}
    \sqr{\sqr{H_0,H_1},H_0}=&-4\paren{\paren{\vec h_0\times\vec h_1}\times\vec h_0}\cdot\sigma\\
    =&-4\paren{\vec h_1\abs{\vec h_0}^2-\vec h_0\paren{\vec h_0\cdot\vec h_1}}\cdot\sigma\\
  \end{split}\\
  \begin{split}
    \sqr{\sqr{H_0,H_1},H_1}=&-4\paren{\paren{\vec h_0\times\vec h_1}\times\vec h_1}\cdot\sigma\\
    =&-4\paren{\vec h_1\paren{\vec h_0\cdot\vec h_1}-\vec h_0\abs{\vec h_1}^2}\cdot\sigma\\
  \end{split}\\
  \begin{split}
    \sqr{\sqr{\sqr{H_0,H_1},H_0},H_0}=&-8\ui\paren{\paren{\vec h_1\abs{\vec h_0}^2-\vec h_0\paren{\vec h_0\cdot\vec h_1}}\times\vec h_0}\cdot\sigma\\
    =&8\ui\abs{\vec h_0}^2\paren{\vec h_0\times\vec h_1}\cdot\sigma\\
  \end{split}\\
  \begin{split}
    \sqr{\sqr{\sqr{H_0,H_1},H_0},H_1}=&-8\ui\paren{\paren{\vec h_1\abs{\vec h_0}^2-\vec h_0\paren{\vec h_0\cdot\vec h_1}}\times\vec h_1}\cdot\sigma\\
    =&8\ui\paren{\vec h_0\cdot\vec h_1}\paren{\vec h_0\times\vec h_1}\cdot\sigma\\
  \end{split}\\
  \begin{split}
    \sqr{\sqr{\sqr{H_0,H_1},H_1},H_0}=&8\ui\paren{\vec h_0\cdot\vec h_1}\paren{\vec h_0\times\vec h_1}\cdot\sigma\\
  \end{split}\\
  \begin{split}
    \sqr{\sqr{\sqr{H_0,H_1},H_1},H_1}=&8\ui\abs{\vec h_1}^2\paren{\vec h_0\times\vec h_1}\cdot\sigma\\
  \end{split}
}

Integrals,
\eqar{
  \begin{split}
    \Omega_1=&\paren{\tau_2-\tau_1}\paren{\vec h_0+\frac{\tau_2+\tau_1}{2}\vec h_1}\cdot\sigma\\
  \end{split}\\
  \begin{split}
    \Omega_2=&-\ui\frac{\paren{\tau_2-\tau_1}^3}{6}\paren{\vec h_0\times\vec h_1}\cdot\sigma\\
  \end{split}\\
  \begin{split}
    \Omega_3=&-\frac{\paren{\tau_2-\tau_1}^5}{60}\paren{\paren{\vec h_0\cdot\vec h_1}\vec h_1-\abs{\vec h_1}^2\vec h_0}\cdot\sigma\\
  \end{split}\\
  \begin{split}
    \Omega_4=&\ui\frac{\paren{\tau_2-\tau_1}^5}{90}
               \paren{
               \abs{\vec h_0}^2
               +\paren{\tau_1+\tau_2}\paren{\vec h_0\cdot\vec h_1}
               +\frac{\tau_1^2+5\tau_1\tau_2+\tau_2^2}{7}
               \abs{\vec h_1}^2}\paren{\vec h_0\times\vec h_1}\cdot\sigma\\
  \end{split}
}


\end{document}
