\documentclass[10pt,fleqn]{article}
%\usepackage[journal=rsc]{chemstyle}
%\usepackage{mhchem}
\usepackage{amsmath}
\usepackage{amssymb}
\usepackage{amsfonts}
\usepackage{esint}
\usepackage{bbm}
\usepackage{amscd}
\usepackage{picinpar}
\usepackage[pdftex]{graphicx}
\usepackage{indentfirst}
\usepackage{wrapfig}
\usepackage{units}
\usepackage{textcomp}
\usepackage[utf8x]{inputenc}
\usepackage{feyn}
\usepackage{feynmp}
\DeclareGraphicsRule{*}{mps}{*}{}
\newcommand{\ud}{\mathrm{d}}
\newcommand{\ue}{\mathrm{e}}
\newcommand{\ui}{\mathrm{i}}
\newcommand{\res}{\mathrm{Res}}
\newcommand{\Tr}{\mathrm{Tr}}
\newcommand{\dsum}{\displaystyle\sum}
\newcommand{\dprod}{\displaystyle\prod}
\newcommand{\dlim}{\displaystyle\lim}
\newcommand{\dint}{\displaystyle\int}
\newcommand{\fsno}[1]{{\!\not\!{#1}}}
\newcommand{\eqar}[1]
{
  \begin{align*}
    #1
  \end{align*}
}
\newcommand{\texp}[2]{\ensuremath{{#1}\times10^{#2}}}
\newcommand{\dexp}[2]{\ensuremath{{#1}\cdot10^{#2}}}
\newcommand{\eval}[2]{{\left.{#1}\right|_{#2}}}
\newcommand{\paren}[1]{{\left({#1}\right)}}
\newcommand{\lparen}[1]{{\left({#1}\right.}}
\newcommand{\rparen}[1]{{\left.{#1}\right)}}
\newcommand{\abs}[1]{{\left|{#1}\right|}}
\newcommand{\sqr}[1]{{\left[{#1}\right]}}
\newcommand{\crly}[1]{{\left\{{#1}\right\}}}
\newcommand{\angl}[1]{{\left\langle{#1}\right\rangle}}
\newcommand{\tpdiff}[4][{}]{{\paren{\frac{\partial^{#1} {#2}}{\partial {#3}{}^{#1}}}_{#4}}}
\newcommand{\tpsdiff}[4][{}]{{\paren{\frac{\partial^{#1}}{\partial {#3}{}^{#1}}{#2}}_{#4}}}
\newcommand{\pdiff}[3][{}]{{\frac{\partial^{#1} {#2}}{\partial {#3}{}^{#1}}}}
\newcommand{\diff}[3][{}]{{\frac{\ud^{#1} {#2}}{\ud {#3}{}^{#1}}}}
\newcommand{\psdiff}[3][{}]{{\frac{\partial^{#1}}{\partial {#3}{}^{#1}} {#2}}}
\newcommand{\sdiff}[3][{}]{{\frac{\ud^{#1}}{\ud {#3}{}^{#1}} {#2}}}
\newcommand{\tpddiff}[4][{}]{{\left(\dfrac{\partial^{#1} {#2}}{\partial {#3}{}^{#1}}\right)_{#4}}}
\newcommand{\tpsddiff}[4][{}]{{\paren{\dfrac{\partial^{#1}}{\partial {#3}{}^{#1}}{#2}}_{#4}}}
\newcommand{\pddiff}[3][{}]{{\dfrac{\partial^{#1} {#2}}{\partial {#3}{}^{#1}}}}
\newcommand{\ddiff}[3][{}]{{\dfrac{\ud^{#1} {#2}}{\ud {#3}{}^{#1}}}}
\newcommand{\psddiff}[3][{}]{{\frac{\partial^{#1}}{\partial{}^{#1} {#3}} {#2}}}
\newcommand{\sddiff}[3][{}]{{\frac{\ud^{#1}}{\ud {#3}{}^{#1}} {#2}}}
\usepackage{fancyhdr}
\usepackage{multirow}
\usepackage{fontenc}
%\usepackage{tipa}
\usepackage{ulem}
\usepackage{color}
\usepackage{cancel}
\newcommand{\hcancel}[2][black]{\setbox0=\hbox{#2}%
\rlap{\raisebox{.45\ht0}{\textcolor{#1}{\rule{\wd0}{1pt}}}}#2}
\pagestyle{fancy}
\setlength{\headheight}{67pt}
\fancyhead{}
\fancyfoot{}
\fancyfoot[C]{\thepage}
\renewcommand{\footruleskip}{0pt}
\renewcommand{\headrulewidth}{0.4pt}
\renewcommand{\footrulewidth}{0pt}
\addtolength{\hoffset}{-1.3cm}
\addtolength{\voffset}{-2cm}
\addtolength{\textwidth}{3cm}
\addtolength{\textheight}{2.5cm}
\renewcommand{\footskip}{10pt}
\setlength{\headwidth}{\textwidth}
\setlength{\headsep}{20pt}
\setlength{\marginparwidth}{0pt}
\parindent=0pt
\begin{document}
\section{Simplifies Optical Bloch Equation for Sideband Cooling Simulation.}
Rabi frequency between state $m$ and $n$ (assume to be real since the phase is not important for sidband cooling.): $\Omega_{mn}$\\
Pumping rate from state $n$ to $m$: $\Gamma_{mn}$\\

Diagonal terms,
\eqar{
  \diff{\rho_{mm}}{t}=&-\rho_{mm}\sum_k\Gamma_{km}+\sum_k\rho_{kk}\Gamma_{mk}+\ui\sum_k\paren{\rho_{mk}\Omega_{km}-\Omega_{mk}\rho_{km}}
}
Off-diagnal terms,
\eqar{
  \diff{\rho_{mn}}{t}=&-\frac{\rho_{mn}}{2}\sum_k\paren{\Gamma_{km}+\Gamma_{kn}}+\ui\sum_k\paren{\rho_{mk}\Omega_{kn}-\Omega_{mk}\rho_{kn}}
}

When only one sideband is driven,
\eqar{
  \Omega_{mn}=&\Omega_{m}\delta_{m,n-\Delta}+\Omega_{n}\delta_{n,m-\Delta}
}
where $Delta$ include both the change in vibrational level and internal level.\\
Define $p_n=\rho_{nn}$, the equations becomes,
\eqar{
  \diff{p_{m}}{t}=&\sum_k\paren{p_{k}\Gamma_{mk}-p_{m}\Gamma_{km}}+\ui\paren{\rho_{m,m-\Delta}\Omega_{m-\Delta}-\Omega_{m}\rho_{m+\Delta, m}+\rho_{m,m+\Delta}\Omega_{m}-\Omega_{m-\Delta}\rho_{m-\Delta,m}}\\
  \diff{\rho_{m,m+\Delta}}{t}=&-\frac{\rho_{m,m+\Delta}}{2}\sum_k\paren{\Gamma_{km}+\Gamma_{k,m+\Delta}}+\ui\Omega_{m}\paren{p_{m}-p_{m+\Delta}}
}
$\rho_{mn}$'s with $\abs{m-n}\neq0,\Delta$ are ignored since they are $0$. In particular, since $\Delta$ includes change of internal levels, elements with $\abs{m-n}\geqslant2\Delta$ does not exist.\\

Define $q_n=\ui\rho_{n,n+\Delta}$,
\eqar{
  \rho_{n,n+\Delta}=&-\ui q_n\\
  \rho_{n+\Delta,n}=&\ui q_n^*
}
\eqar{
  \diff{p_{m}}{t}=&\sum_k\paren{p_{k}\Gamma_{mk}-p_{m}\Gamma_{km}}+\ui\paren{\ui q_{m-\Delta}^*\Omega_{m-\Delta}-\ui\Omega_{m}q_{m}^*-\ui q_m\Omega_{m}+\ui\Omega_{m-\Delta}q_{m-\Delta}}\\
  -\ui\diff{q_m}{t}=&\ui\frac{q_{m}}{2}\sum_k\paren{\Gamma_{km}+\Gamma_{k,m+\Delta}}+\ui\Omega_{m}\paren{p_{m}-p_{m+\Delta}}\\
  \diff{p_{m}}{t}=&\sum_k\paren{p_{k}\Gamma_{mk}-p_{m}\Gamma_{km}}+\Omega_{m}\paren{q_{m}^*+q_m}-\Omega_{m-\Delta}\paren{q_{m-\Delta}^*+q_{m-\Delta}}\\
  \diff{q_m}{t}=&-\frac{q_{m}}{2}\sum_k\paren{\Gamma_{km}+\Gamma_{k,m+\Delta}}+\Omega_{m}\paren{p_{m+\Delta}-p_{m}}
}
For a process starting with $q_m=0$, $q_m$ will always remain real.
\eqar{
  \diff{p_{m}}{t}=&\sum_kp_{k}\Gamma_{mk}-p_{m}\Gamma_{m}+2\Omega_{m}q_m-2\Omega_{m-\Delta}q_{m-\Delta}\\
  \diff{q_m}{t}=&-\frac{q_{m}}{2}\paren{\Gamma_{m}+\Gamma_{m+\Delta}}+\Omega_{m}\paren{p_{m+\Delta}-p_{m}}
}
where $\Gamma_m\equiv\displaystyle\sum_k\Gamma_{km}$ is the decay rate of state $m$.\\

After writing the two internal states ($a$ and $b$) explicitly and adding a third state ($c$) to take into account $m_F$ pumping,
\eqar{
  \diff{p^a_{m}}{t}=&\sum_{\alpha=a,b,c;k}p^\alpha_{k}\Gamma^{a\alpha}_{mk}-p^a_{m}\Gamma^a_{m}+2\Omega_{m}q_m\\
  \diff{p^b_{m}}{t}=&\sum_{\alpha=a,b,c;k}p^\alpha_{k}\Gamma^{b\alpha}_{mk}-p^b_{m}\Gamma^b_{m}-2\Omega_{m-\delta}q_{m-\delta}\\
  \diff{p^c_{m}}{t}=&\sum_{\alpha=a,b,c;k}p^\alpha_{k}\Gamma^{c\alpha}_{mk}-p^c_{m}\Gamma^c_{m}\\
  \diff{q_m}{t}=&-\frac{q_{m}}{2}\paren{\Gamma^a_{m}+\Gamma^b_{m+\delta}}+\Omega_{m}\paren{p^b_{m+\delta}-p^a_{m}}
}
For sodium, states $a$, $b$ and $c$ corresponds to $|F=1, m_F=1\rangle$, $|F=2, m_F=2\rangle$ and  $|F=2, m_F=1\rangle$.\\

When the trapping frequency is much smaller than the line width, $\Gamma^\alpha_m$ is independent with $m$ and is proportional to the corresponding pumping power. $\Gamma^{\alpha\beta}_{mn}$ can be written as,
\eqar{
  \Gamma^{\alpha\beta}_{mn}=&\Gamma^{\alpha\beta}\gamma_{mn}\\
  =&\begin{pmatrix}
    \Gamma^{aa}&\Gamma^{ab}&\Gamma^{ac}\\
    \Gamma^{ba}&\Gamma^{bb}&\Gamma^{bc}\\
    \Gamma^{ca}&\Gamma^{cb}&\Gamma^{cc}
  \end{pmatrix}\gamma_{mn}\\
  =&\begin{pmatrix}
    \Gamma^aB^{aa}&\Gamma^bB^{ab}&\Gamma^cB^{ac}\\
    \Gamma^aB^{ba}&\Gamma^bB^{bb}&\Gamma^cB^{bc}\\
    \Gamma^aB^{ca}&\Gamma^bB^{cb}&\Gamma^cB^{cc}
  \end{pmatrix}\gamma_{mn}\\
  =&\begin{pmatrix}
    \Gamma_1B^{aa}&\varepsilon\Gamma_2B^{ab}&\Gamma_2B^{ac}\\
    \Gamma_1B^{ba}&\varepsilon\Gamma_2B^{bb}&\Gamma_2B^{bc}\\
    \Gamma_1B^{ca}&\varepsilon\Gamma_2B^{cb}&\Gamma_2B^{cc}
  \end{pmatrix}\gamma_{mn}
}
where $B^{\alpha\beta}$ is the branching ratio, $\Gamma_1$ is the $F$ pumping power ($(1, 1) \rightarrow (2, 2)'$ light), $\Gamma_2$ is the $m_F$ pumping power ($(2, 1) \rightarrow (2, 2)'$ light), $\varepsilon$ is the off-resonance coupling of the $m_F$ pumping light to the $(2, 2) \rightarrow (2, 3)'$ transition and $\gamma_{mn}$ is the state independent direction averaged recoil coupling between state $m$ and $n$.
\eqar{
  \gamma_{mn}=&\gamma_{nm}\\
  \sum_{m}\gamma_{mn}=&1
}

With $0.6\%$ polarization misalignment and for $\Gamma_1$ and $\Gamma_2$ normalized to the pumping rate of the $(1, 1)$ and $(2, 1)$ states.
\eqar{
  \Gamma^{\alpha\beta}=&\begin{pmatrix}
    0.500\Gamma_1&0.006\Gamma_2&0.500\Gamma_2\\
    0.333\Gamma_1&0.171\Gamma_2&0.333\Gamma_2\\
    0.167\Gamma_1&0.002\Gamma_2&0.167\Gamma_2
  \end{pmatrix}\\
  \Gamma^a_m=&\Gamma_1\\
  \Gamma^b_m=&0.179\Gamma_2\\
  \Gamma^c_m=&\Gamma_2
}
If we use the $D1$ transition to do the pumping instead the matrix becomes
\eqar{
  \Gamma^{\alpha\beta}=&\begin{pmatrix}
    0.500\Gamma_1&0.006\Gamma_2&0.500\Gamma_2\\
    0.333\Gamma_1&0.004\Gamma_2&0.333\Gamma_2\\
    0.167\Gamma_1&0.002\Gamma_2&0.167\Gamma_2
  \end{pmatrix}\\
  \Gamma^a_m=&\Gamma_1\\
  \Gamma^b_m=&0.012\Gamma_2\\
  \Gamma^c_m=&\Gamma_2
}


\end{document}
